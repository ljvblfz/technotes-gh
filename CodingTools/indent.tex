\section{Indent代码排版工具}

欲实现Cavium风格的代码,有
\begin{verbatim}
indent *.c -bli0 -i4 -npsl -cli4 -npcs
\end{verbatim}

其中,bli表示brace indentation, i选项表示indentation, npsl(dont-break-procedure-type)表示函数返回类型需与函数名称在同一行;cli表示case-label-indentation, 表示case语言缩进的距离。npcs表示no-sapce-after-function-call-names,函数调用时函数名称后无空格。

关于if等语句之后的\verb+{+括号是否在换行,有两种方式:br(braces-on-if-line)和bl(braces-after-if-line)。对于br选项,一般同时指定ce选项(cuddle-else)或nce选项,前者让\verb+}+和else在同一行。对于bl选项,一般同时指定bli选项,表示\verb+{+缩进距离,如不指定,GNU indent默认使用GNU风格,即缩进2个空格。类似于if语句,函数定义和struct定义也存在大括号是否在同一行的问题。对于函数,可以指定brf或blf(默认);对于struct,可以指定brs和blf。
