
\section{RHEL配置yum}

\begin{enumerate}
    \item 如果没有安装yum,先安装yum
    \item /etc/yum.repos.d目录下,删除原有的repo文件,替换为Centos的源文件
    \item /etc/yum.conf文件,添加一行\verb+timeout=120+,未验证必要性
    \item 使用rpm命令import选项导入密钥文件
    \item 运行yum update命令
\end{enumerate}

关于密钥文件,目前不是很懂其原理。网上提到的有:
\begin{verbatim}
http://ftp.sjtu.edu.cn/centos/5/os/i386/RPM-GPG-KEY-CentOS-5
http://centos.ustc.edu.cn/centos/RPM-GPG-KEY-CentOS-5
\end{verbatim}
如果不导入密钥,那么安装可能不能成功。yum有nogpgcheck选项,以及配置文件也有相关选项,可能会强制安装软件。未测试。

安装软件包时需要知道这个包在yum系统中的名字。使用yum search命令。例如,需要安装git时,键入
\begin{verbatim}
yum search git
\end{verbatim}
易知git在yum系统中可能叫做git.i386.

