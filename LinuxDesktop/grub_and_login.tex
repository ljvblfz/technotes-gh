\section{Grub与系统登录}
\subsection{修改Grub配置}
配置/etc/default/grub文件,然后运行update-grub,会自动生成/boot/grub/grub.cfg文件
\subsection{安装Grub}
如果安装了多个Linux系统,最后一个安装的系统的GRUB会覆盖之前安装的GRUB。如果这不是所希望的,可以进入心目中的默认系统重新安装GRUB。如果最后安装的是Windows,
则GRUB会被破坏,不会在开机时显示GRUB界面,也需要设法进入Linux后重新安装GRUB。
进入所希望的Linux系统后,执行grub-install. grub-install copies GRUB images into /boot/grub, and uses grub-setup to install grub into the boot sector.
\begin{verbatim}
sudo grub-install /dev/sda
\end{verbatim}
可以添加选项,\verb+boot-directory=DIR+
install GRUB images under the directory DIR/grub instead of the /boot/grub directory。
如果是从LiveCD进入的系统,执行grub-install前先chroot。
如果没有LiveCD,在进入系统前停留在了如下界面上:
\begin{verbatim}
GRUB loading
error:unknow filesystem
grub rescue>
\end{verbatim}
可以执行ls命令,找到Linux被安装在了哪个分区。如果确定了分区,比如,(hd0,5),则执行\verb+ls (hd0, 5)+时会看到许多mod文件。
安装normal.mod:
\begin{verbatim}
grub rescue>set root=(hd0,5)
grub rescue>set prefix=(hd0,5)/boot/grub
grub rescue>insmod /boot/grub/normal.mod
\end{verbatim}
执行normal命令,可以恢复Grub界面。进入Linux后重新安装GRUB即可。


\subsection{登陆密码恢复}
方法如下:

1、重新启动,按ESC键进入Boot Menu,选择recovery mode(一般是第二个选项)。

2、在\#号提示符下用cat /etc/shadow,看看用户名。

3、输入passwd "用户名"(引号要有的哦)。

4、输入新的密码.

5、重新启动,用新密码登录。 
