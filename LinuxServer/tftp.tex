\section{tftp Server安装与配置}

Ubuntu下,安装tftp-hpa。在/etc/default/tftp-hpa目录下配置。
RHEL/CentOS环境下,安装tftp-server。守护进程名称为in.tftpd。

tftp的配置文件在/etc/xinetd.d/tftp,可修改disable=no,默认路径等参数。然后执行:
\begin{lstlisting}
    service xinetd restart
\end{lstlisting}

可用netstat命令查看69号端口是否开启。测试发现xinetd开启若干分钟后,in.tftpd守护进程才启动。此时可在另一台主机上测试:

\begin{lstlisting}
    tftp (serverip)
    tftp>get (some_file_name)
\end{lstlisting}

如果结果是
\begin{lstlisting}
Error code 0: Permission denied
\end{lstlisting}
可能是selinux问题。运行
\begin{lstlisting}
ls -alZ
\end{lstlisting}
结果应包含:
\begin{lstlisting}
 user_u:object_r:tftpdir_t
\end{lstlisting}

解决办法是:
\begin{lstlisting}
restorecon -Rv /tftpboot 
\end{lstlisting}

