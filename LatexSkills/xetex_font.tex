\section{xelatex下的字体调节}

\subsection{字形调节}
一般有:
\begin{verbatim}
\textrm{...} \textbf{...}
\textsf{...} \textit{...} 
\texttt{...} \textsl{...}
\emph{...} \underline{...}
\textsc{...} 
\end{verbatim}
或者\verb+\itshape \sffamily+等。


\subsection{全局字体设置}
示例:
\begin{verbatim}
\setCJKmainfont[BoldFont=SimHei,ItalicFont=KaiTi]{SimSun}
\setCJKmainfont[ItalicFont={KaiTi_GB2312}]{SimSun}
\setCJKsansfont{SimHei}
\end{verbatim}
注意,\verb+KaiTi_GB2312+中的下划线不需要转义。需要事先查看系统已经安装了的字体:
\begin{verbatim}
fc-list :lang=zh
\end{verbatim}
以下模板从网上抄得:
\begin{verbatim}
\usepackage{xeCJK}
\CJKlanguage{zh-cn}
\setmainfont{DejaVu Sans}
\setCJKmainfont{AR PL UMing CN}
\setCJKsansfont[BoldFont=AR PL New Kai]{AR PL New Kai}
\setCJKmonofont{DejaVu Sans Mono}
\setCJKfamilyfont{song}{AR PL New Sung}
\setCJKfamilyfont{kai}{AR PL New Kai}
\setCJKfamilyfont{hei}{文泉驿正黑}
\end{verbatim}

\subsection{xeCJK字体调用}
在导言中设置hei字体为系统黑体:
\begin{verbatim}
\setCJKfamilyfont{hei}{SimHei}
\end{verbatim}
在正文中使用字体调用:
\begin{verbatim}
{\CJKfamily{hei}
这里对CDN有个要求,就是对同一个媒体资源子文件,相应的transferID必须保持一致
}
\end{verbatim}






