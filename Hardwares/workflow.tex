\documentclass[11pt,a4paper]{article}
\usepackage[top=1in,bottom=1in,left=1.2in,right=1.2in]{geometry}
\XeTeXlinebreaklocale "zh"
\XeTeXlinebreakskip = 0pt plus 1pt
\usepackage{indentfirst} 
\usepackage{fontspec}
\usepackage{titlesec}
\usepackage{xeCJK}
\usepackage{graphicx}
\usepackage{rotating}
\setCJKmainfont[ItalicFont={KaiTi}]{SimSun}
\setCJKsansfont{SimHei}
\setCJKfamilyfont{hei}{SimHei}

\usepackage{tikz}
\usetikzlibrary{calc}
\usetikzlibrary{fit}
\usetikzlibrary{positioning}
\usepgflibrary{plotmarks}

\usetikzlibrary{shapes.geometric}

\renewcommand{\today}{\number\year 年 \number\month 月 \number\day 日}
\renewcommand{\figurename}{图} 
\title{\sffamily 基于SE的B接口设计概要}
\author{李明哲}
\date{\today}



\begin{document}

\tikzset{ box/.style={
rectangle, minimum width=1.3cm,very thick, draw=gray!50!black!50, text centered,
text width=0.2\textwidth, top color=white, bottom color=gray!50!black!20},
diamondbox/.style={diamond, 
draw=gray!50!black!, top color=white, bottom color=gray!50!black!10,
minimum width=3cm, minimum height=1.5cm, very thick, inner sep=0pt, text centered},
ellipbox/.style={ellipse,minimum width=1.3cm, minimum height=.8cm, draw=gray!50!black!, very thick, text width=0.07\textwidth, inner sep=0pt},
every node/.style={text badly centered, font=\footnotesize}
}

%自定义命令:用于流程图判断语句
\newcommand{\abovelabel}[1]{node[midway, above, text width=20mm]{#1}}
\newcommand{\rightlabel}[1]{node[midway, right]{#1}}


\begin{figure}[ht]
    \centering
    \begin{tikzpicture}
	\matrix[row sep=8mm, column sep=30mm]
	{
            &\node[box](cpu){CPU};&\node[box](cache){高速缓存};\\
            \node[box](graphic card){显卡槽};&\node[box](north){北桥};&\node[box](mem){内存};\\
            \node(usb){USB设备};&\node(south1){};&\node(ide){IDE硬盘};\\
            \node(isa){ISA设备(过时)};&\node(south2){};&\node(sata){SATA硬盘};\\
            \node(pci slots){PCI槽};&\node(south3){};&\node(onboard graph){Onboard显卡控制器};\\
            \node(phy){以太网};&\node(south4){};&\node(audio){声卡};\\
            \node(cmos){CMOS内存};&\node(south5){};&\node[box](flash rom){Flash Rom};\\
            &\node[box](super io){Super I/O};\\
	};

	\begin{scope}[every path/.style={draw, thick}]
            \path (cpu) -- (north) \rightlabel{FSB}; 
            \path (cpu) -- (cache) \abovelabel{BSB}; 
            \path (graphic card) -- (north) \abovelabel{高速图形总线(PCI-E或AGP)} -- (mem) \abovelabel{内存总线}; 
            \path (north) -- (south1) \rightlabel{桥间总线}; 
	    \path (usb) -- (south1) \abovelabel{USB总线}-- (ide) \abovelabel{IDE总线};
	    \path (isa) -- (south2) \abovelabel{ISA总线}-- (sata) \abovelabel{?总线};
	    \path (pci slots) -- (south3) \abovelabel{PCI总线}-- (onboard graph) \abovelabel{PCI总线};
	    \path (phy) -- (south4) \abovelabel{PCI等总线}-- (audio) \abovelabel{总线};
	    \path (cmos) -- (south5) \abovelabel{PCI等总线} -- (flash rom) \abovelabel{LPC总线};
	    \path (south5) -- (super io) \rightlabel{LPC总线};
	\end{scope}
        
        \node[box, fit=(south1)(south5)](south){南桥};

    \end{tikzpicture}
    \caption{MainBoard}
\end{figure}


\end{document}

