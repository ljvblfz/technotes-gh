\section{Ubuntu/Deepin安装后的配置}

\subsection{配置网络和locale}
直接使用图形工具network-manager-gnome即可。

\subsection{硬盘设置}
设置/etc/fstab,使得一些硬盘分区开机自动挂载。
\begin{shellcmd}
UUID=4C0E	/media/D \	
ntfs-3g	defaults,nosuid,nodev,locale=zh_CN.UTF-8	0	0
UUID=077  /media/E12	ext4	defaults	0	0
\end{shellcmd}
有个图形工具叫做pysdm;另外对于ntfs分区,可以使用ntfs-config工具。
配置完/etc/fstab后,使用mount -a命令执行该文件。

\subsection{软件安装}
设置软件更新源,更新系统,并安装软件。对于deepin官方源,直接支持IPV6。对于USTC等其他源,可能需要将地址中的mirror改为mirror6。另外如果sources.list.d目录下有无法访问的地址,需要将其删除,可以将该目录重命名。

PPA清单:
\begin{verbatim}
UbuntuTweak
ppa:tualatrix/ppa
pidgin的WebQQ插件
ppa:lainme/lwqq
\end{verbatim}

包清单:
vim, vim-gnome, vim-scripts, vim-doc, vim-latexsuite,ntfs-config,pysdm,ubuntu-tweak(ppa:tualatrix/ppa),gnome-tweak-tool,nautilus-open-terminal,gdebi,Pidgin,chromium-browser, aMSN, sshpass, lcrt, texlive, texlive-xetex

需要手动下载的商用软件:
永中Office,Adobe Reader, Nutstore, Mendeley, Chrome

\subsubsection{系统工具}
Gnome桌面:gnome-shell和经典桌面gnome-panel。gnome-shell可以安装panel-dock插件。通过PPA可以安装cinnamon桌面,是LinuxMint开发的Gnome2风格的桌面。

Deb包管理器gdebi。

系统设置工具:ubuntu-tweak(ppa:tualatrix/ppa),gnome-tweak-tool。

nautilus辅助:nautilus-open-terminal

深度软件中心:deepin-softwarecenter(ppa:noobslab/deepin-sc)。

Adobe Air:官网下载安装即可。在Ubuntu12.04上安装会出错,可以执行:
\begin{verbatim}
sudo ln -s /usr/lib/i386-linux-gnu/\
libgnome-keyring.so.0.2.0 \
/usr/lib/libgnome-keyring.so.0
\end{verbatim}

\subsubsection{网络通信}
Pidgin,chromium-browser, aMSN, sshpass, lcrt, 网盘等。

\subsubsection{音视频播放}
视频播放器:Gnome-Mplayer和VLC。

视频网站客户端:ppstream。

音乐播放器:可选择banshee,audacious,rhythmbox。

歌词搜索程序:osdlyrics(ppa:osd-lyrics/ppa)为独立程序。执行osdlyrics命令,以将其打开。Ctrl+Shift+L快捷键使其解除锁定。

\subsubsection{办公软件}

VIM:vim-scripts,vim-latexsuite和vim-doc包含了Vim的一些有用插件。vim-addon-manager为Vim插件管理程序,安装该工具之后可以用vim-addons命令激活各种插件。

Latex:在texlive和texlive-xetex, texlive-latex-extra。

办公套件:永中2012青春版。

PDF文件阅读与编辑:acrobat, xournal, pdfmod。 

文献管理:mendeleydesktop。

制图工具:QtiPlot(仿制Origin), dia(模仿Visio)。论文制图也可使用办公套件中的电子表格程序,或GNUplot。 

词典: Goldendict

编程相关: ctags, cscope, doxygen, manpages-dev(Ubuntu可能已经预装), qtcreator, python-qt4, python-qt4-doc

\subsubsection{虚拟机}
KVM虚拟系统:qemu-kvm libvirt-bin virt-manager bridge-utils。
KVM制作的虚拟机保存路径一般是/var/lib/libvirt/images。
打开virt-manager可能需要root权限。


\subsection{系统设置工具一览}

系统设置:gnome-control-center

高级设置:gnome-tweak-tool

系统配置机制:gnome3用Gsettings机制取代了gconf机制。用gsettings取代了gconftool-2工具。dconf-editor是gsettings的众多后端工具之一。

\subsection{设置unity白名单}
如果是Unity桌面,设置unity白名单:
\begin{shellcmd}
gsettings set com.canonical.Unity.Panel systray-whitelist "['all']"
\end{shellcmd}

\subsection{字体配置}
文泉驿可能被默认安装了。安装少量Windows字体即可。如simsun,simfang,simhei,simkai
使用\verb+fc-list :lang=zh-cn+查看当前系统字体。

\subsection{主菜单配置}
使用alacarte创建的应用程序的快捷入口,创建在~/.local/share/applications目录下,并默认以 alacarte-made[-X].desktop 的格式命名,其中X是数字,用户可以随后重命名这个文件,菜单上显示的内容不会改变。而在alacarte工具中删除的快捷入口,也不会真的删除对应的 desktop 文件,而只是将对应文件中的Hide字段的值改为true。
以root权限安装的程序,其快捷入口大多创建在/usr/share/applications目录下,而以用户权限安装的程序,则只能将快捷入口创建在~/.local/share/applications目录下。

\subsection{默认程序配置}
有几个文件用于存储指定类型文件的关联程序,分别是 /etc/gnome/defaults.list, /usr/share/applications/defaults.list, /usr/share/applications/mimeinfo.cache, ~/.local/share/applications/mimeapps.list, ~/.local/share/applications/mimeinfo.cache。前面三个文件保存全局设置,后面两个保存用户设置。如果要修改某个类型文件的关联程序,可以通过直接修改这几个文件的方式实现。

\subsection{开机自启动程序}
gnome-session-properties工具,可以在主菜单->启动应用程序中找到。

\subsection{桌面启动器配置}
gnome-desktop-item-edit,可以创建或编辑.desktop类型的文件。创建名为my.desktop的启动器,执行:
\begin{verbatim}
gnome-desktop-item-edit --create-new my.desktop
\end{verbatim}










