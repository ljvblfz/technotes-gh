\section{实用计算、查询与搜索}

\subsection{算数}
Linux下一些CGI工具可以用作计算器,如Python,calc,bc等。

Python计算默认为整数计算,如需浮点数运算结果,可以在运算数后加.0;也可以在运行Python时使用命令行开关-Qnew

calc工具来自于apcalc软件包,可以实现任意精度的计算。calc默认就是浮点数计算,比较方便;而bc默认为整数计算。

bc其实是一种复杂的编程语言;如计算5/6, 设置精度为3位,则有:
\begin{verbatim}
	scale=3;5/6
\end{verbatim}

\subsection{进制转换}
可以用Python完成进制转换运算。

十进制转换为二进制、八进制、十六进制,可以分别使用函数bin,oct,hex。 
二进制,8进制,16进制转十进制容易,直接在互动界面上键入0xa3,0b101,0o33即可。

\subsection{ASCII码表}
可以查找man页:\verb+man ascii+
在Python下,十进制转ASCII可以用chr函数,反之使用ord。

\subsection{时间计算} 
Python的datetime包提供了日期与时间相关的操作。 参考\ref{sec:pythonTimeCalc}。

\subsection{日历查询}
cal工具能够打印某年或某月的日历,默认为本月。
\begin{verbatim}
cal #本月
cal 2012 #某年
cal -y 2012 #某年
cal -m 8 #今年某月
cal 8 2012 #某年某月
\end{verbatim}

calendar模块提供了查询平闰年和星期的功能,也能产生日历字符串。参考\ref{sec:pythonTimeCalc}。

\subsection{IP地址查询}
\begin{verbatim}
whois 202.38.95.110
\end{verbatim}

\subsection{Google汇率查询}
汇率查询,比如欲查询人民币和韩币汇率,Google:
\begin{verbatim}
cny in won
rmb in won
china in korea currency
\end{verbatim}

\subsection{Google单位换算}
\begin{verbatim}
mile to km
\end{verbatim}

\subsection{Google技巧}
\url{http://www.googleguide.com}.
布尔运算:与(空格默认为AND)、或(大写OR,|(vertical bar))、非(-)。\verb+()+可调整运算优先级。

域搜索:site, inurl, filetype。

精确搜索:双引号

前缀:allintitle,link

同时,Google提供了高级搜索的UI界面。

























