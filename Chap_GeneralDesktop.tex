\section{实用计算、查询与搜索}

\subsection{算数}
Linux下一些CGI工具可以用作计算器,如Python,calc,bc等。

Python计算默认为整数计算,如需浮点数运算结果,可以在运算数后加.0;也可以在运行Python时使用命令行开关-Qnew

calc工具来自于apcalc软件包,可以实现任意精度的计算。calc默认就是浮点数计算,比较方便;而bc默认为整数计算。

bc其实是一种复杂的编程语言;如计算5/6, 设置精度为3位,则有:
\begin{verbatim}
	scale=3;5/6
\end{verbatim}

\subsection{进制转换}
可以用Python完成进制转换运算。

十进制转换为二进制、八进制、十六进制,可以分别使用函数bin,oct,hex。 
二进制,8进制,16进制转十进制容易,直接在互动界面上键入0xa3,0b101,0o33即可。

\subsection{ASCII码表}
可以查找man页:\verb+man ascii+
在Python下,十进制转ASCII可以用chr函数,反之使用ord。

\subsection{时间计算} 
Python的datetime包提供了日期与时间相关的操作。 参考\ref{sec:pythonTimeCalc}。

\subsection{日历查询}
cal工具能够打印某年或某月的日历,默认为本月。
\begin{verbatim}
cal #本月
cal 2012 #某年
cal -y 2012 #某年
cal -m 8 #今年某月
cal 8 2012 #某年某月
\end{verbatim}

calendar模块提供了查询平闰年和星期的功能,也能产生日历字符串。参考\ref{sec:pythonTimeCalc}。

\subsection{IP地址查询}
\begin{verbatim}
whois 202.38.95.110
\end{verbatim}

\subsection{Google汇率查询}
汇率查询,比如欲查询人民币和韩币汇率,Google:
\begin{verbatim}
cny in won
rmb in won
china in korea currency
\end{verbatim}

\subsection{Google单位换算}
\begin{verbatim}
mile to km
\end{verbatim}

\subsection{Google技巧}
\url{http://www.googleguide.com}.
布尔运算:与(空格默认为AND)、或(大写OR,|(vertical bar))、非(-)。\verb+()+可调整运算优先级。

域搜索:site, inurl, filetype。

精确搜索:双引号

前缀:allintitle,link

同时,Google提供了高级搜索的UI界面。

























\section{ImageMagick工具用法}
ImageMagic包含display,convert, identify, mogrify, montage, compare等工具。
\label{sec:imagemagick}

display用来看图片,如观看当前路径下所有图片,有
\begin{verbatim}
display *
\end{verbatim}

identify用来显示图片信息。

convert用来转换图片:
\begin{verbatim}
convert input-file [options] output-file
\end{verbatim}

pdf和图片相互转换,有
\begin{verbatim}
convert *.png output.pdf
convert haha.pdf 1.png pdf文件转换为前缀为1的png文件
\end{verbatim}

旋转图片
\begin{verbatim}
convert -rotate 90 image.jpg image.png
convert -flip a.jpg b.jpg 上下翻转
convert -flop a.jpg b.jpg 左右翻转
\end{verbatim}

图像加框
\begin{verbatim}
convert -border 60*60 “#000000″ a.jpg b.jpg
60*60是表示边框的宽度,第一个是纵边框的宽度,第二个是横边框的宽度
#000000是RGB格式的边框色彩
\end{verbatim}



以下命令用于改变图片大小(宽×高):
\begin{verbatim}

1. 默认时,宽度和高度表示要最终需要转换图像的最大尺寸,同时Convert会控制图片的宽和高,保证图片按比例进行缩放。

如:convert -resize 600×600 src.jpg dst.jpg

转换后的dst.jpg的图片大小(宽度为600,而高度已经按比例调整为450).

2.如果需要转换成600×600,而图片无需保持原有比例,可以在宽高后面加上一个感叹号!.

如:convert -resize 600×600! src.jpg dst.jpg

3. 只指定高度,图片会转换成指定的高度值,而宽度会按原始图片比例进行转换。

如:convert -resize 400 src.jpg dst.jpg

转换后的dst.jpg的图片大小(宽度为400,而高度已经按比例调整为300),和例1有点类似。

4. 默认都是使用像素作为单位,也可以使用百分比来形象图片的缩放。

如:convert -resize 50%x100%! src.jpg dst.jpg 或者convert -resize 50%x100% src.jpg dst.jpg

此参数只会按你的比例计算后缩放,不保持原有比例。(结果尺寸为100×150)

5.使用 @ 来制定图片的像素个数。

如:convert -resize “10000@” src.jpg dst.jpg

此命令执行后,dst.jpg图片大小为(115×86),图片保持原有比例(115×86= 9080 < 10000)。

6.当原始文件大于指定的宽高时,才进行图片放大缩小,可使用>命令后缀。

如:convert -resize “100×50>” src.jpg dst.jpg

此命令执行后,dst.jpg图片大小为(67×50),图片保持原有比例。

如:convert -resize “100×50>!” src.jpg dst.jpg

此命令执行后,dst.jpg图片大小为(100×50),图片不保持原有比例。

7.当原始文件小于指定的宽高时,才进行图片放大转换,可使用<命令后缀。

如:convert -resize “100×500<” src.jpg dst.jpg 或者convert -resize “100×100<!” src.jpg dst.jpg

此命令执行后,dst.jpg和src.jpg大小相同,因为原始图片宽比100大。

如:convert -resize “600×600<” src.jpg dst.jpg

此命令执行后,dst.jpg图片大小为(600×450),图片保持原有比例。

如:convert -resize “600×600<!” src.jpg dst.jpg

此命令执行后,dst.jpg图片大小为(600×600),图片不保持原有比例。

8.使用^命令后缀可以使用宽高中较小的那个值作为尺寸

如:convert -resize “300×300^” src.jpg dst.jpg

此命令执行后,dst.jpg图片大小为(400×300),图片保持原有比例,(300:300 < 200:150,选择高作为最小尺寸)。

如:convert -resize “300×200^” src.jpg dst.jpg

此命令执行后,dst.jpg图片大小为(300×225),图片保持原有比例,(300:200 > 200:150,选择宽作为最小尺寸)。


\end{verbatim}
\section{音视频播放与编辑}

\subsection{视频播放控制}
\subsubsection{mplayer}
文件打开命令:mplayer [-af scaletempo [-speed 0.01~100]] filename

其中,-af设置音频过滤器,scaletempo实现变速不变调(pitch)功能。详情见其help输出和man页。

快捷键(man页:INTERACTIVE CONTROL):
速度调节: 	[和],以10\%为幅度。
进度调节:	左右方向键以10s为精度,上下方向键以分钟为精度,翻页键以10min为精度。

\begin{verbatim}

 p or SPACE       pause movie (press any key to continue)
 q or ESC         stop playing and quit program
 + or -           adjust audio delay by +/- 0.1 second
 o                cycle OSD mode:  none / seekbar / seekbar + timer
 * or /           increase or decrease PCM volume
 x or z           adjust subtitle delay by +/- 0.1 second
 r or t           adjust subtitle position up/down, also see -vf expand

backspace          Reset playback speed to normal.
 f                 Toggle fullscreen (also see -fs).
 T                 Toggle stay-on-top (also see -ontop).

v                Toggle subtitle visibility.
j and J          Cycle through the available subtitles.
\end{verbatim}

\subsubsection{VLC}
详情见其help输出和man页。VLC默认为变速不变调。

速度调节:[和],同mplayer一样的快捷键,只是调节幅度为0.1倍原速。

进度调节:shift/alt/ctrl+左右方向键。shift的精度为5s,alt为10s,ctrl为1min。


\subsection{音视频文件编辑}
Linux下有OpenShot Video Editor。曾用它来裁剪视频文件。

跨平台开源软件audacity,操作方式类似于Adobe Audition(CoolEditPro,不支持Linux),可以看见音频波形。支持批量导入音频文件。

\section{网络通信工具}

\subsection{邮件提醒}

\begin{description}
    \item[Pidgin插件]安装pidgin-guifications插件,并设置Pidgin接收MSN邮件。
    \item[checkgmail]checkgmail可以检查Gmail邮箱。将所关注的邮箱设置成自动转发到gmail,设置gmail的动作为打开邮件客户端如evolution。
    \item[其他方案]mail-notification;gnubiff。
\end{description}

\subsection{即时通信}

常见的聊天工具如QQ,MSN和飞信都有网页版,可以用浏览器制作Web App。除此之外,还有一些第三方客户端。

飞信: Web飞信,3G飞信,openfetion,cliofetion,pidgin-openfetion。

QQ:Web Q+, libqq-pidgin(ppa:lainme/libqq,基于QQ国际版,已经失效),Wine-qq。基于Web QQ协议,分别有人使用Python,GTK,Qt,Java、Pidgin开发了QQ客户端,包括:python-webqq(ppa:linux-deepin-team/linux-deepin), gtkqq(ppa:bill-zt/gtkqq),QtQQ, iQQ,pidgin-lwqq(ppa:lainme/pidgin-lwqq)。

MSN:Web Windows Live,pidgin,aMSN,kmess, kopete, emesene,empathy等。

\subsection{聊天消息提醒}
Web Q+:桌面通知功能,可以实现跨工作区提醒,直接显示聊天内容于桌面。标题栏提醒功能,不能跨工作区提醒,不暴露消息内容。消息走马灯功能,尚不知如何使用。

Pidgin:libnotify插件,会弹出消息,不够隐私;guification(软件包pidgin-guifications)插件,和notify(中文:消息提示)插件,可以做到跨工作区提醒,不直接显示消息。后来发现notify插件没什么作用(Linux Deepin 12.06)。好友千里眼,添加对单独好友的新消息监视,一次性使用,也可以勾选“重复”,每条消息都会提示一次。最好的办法是右键面板图标,勾选“有新消息时闪烁”。

其他聊天工具具体情况有待补充。

\subsection{Chromium浏览器}
软件包叫做chromium-browser,默认不自带flash插件。需要安装flashplugin-installer。然后执行:
\begin{shellcmd}
    sudo cp /usr/lib/flashplugin-installer/\
    libflashplayer.so \ 
    /usr/lib/chromium-browser/
\end{shellcmd}

\subsection{远程登录}
ssh远程登录:sshpass,lcrt。lcrt是为数不多的能保存密码的ssh登录工具。

\section{PDF阅读、编辑和转换}

\subsection{PDF阅读与批注}
acroread软件包包含了Adobe Reader, 兼容性好,可以调节背景色。但功能相对于Acrobat十分受限, 没有批注、书签等功能,运行效率也低下。

okular可以实现pdf的批注(F6或tools->review)和背景色改变。其缺点是,如果在gnome环境下安装需要下载几十MB的包。

mendeley可以实现批注,但批注只是内部识别。不能调节背景色。

xournal添加文字注释,下划线,通过export功能保存修改结果;但是Xournal对于篇幅较长的PDF文档太耗尽内存,因为它会将PDF文档中所有的页面都转化为图像数据置于内存之中并且不再释放。我们可以首先对文档进行分割。据说只要别大范围拖动滚动条,内存占用便不大。

evince,Foxit4Linux,永中阅读器既不能标记,又不能改变背景色。

综上,对于pdf阅读,较好的有okular,wine上的foxit,cajviewer。

pdfgrep(CLI)可以从pdf中查找正则表达式,用法类似于grep。diffpdf可以比较两个pdf文件的不同。

\subsection{PDF书签编辑}
pdfmod是目前发现的唯一一个制作书签的开源工具,但编辑不便,无法调整书签的显示顺序。Windows下的Foxit Reader可以制作pdf书签,linux版本的则不可。evince制作的书签似乎只是内部识别。因此,在Linux下最好wine一个Windows版的Foxit Reader。

\subsection{PDF页面级编辑}

pdfshuffler可以合并、分裂、排序页面,在precise pangolion尝试出错,显示没有EOF标记。

pdftk为CLI工具,功能包括合并,分裂,删页,反序,旋转,加解密,其中合并pdf的方式包括连接和互插。pdfchain是pdftk的一个GUI前端。

pdfmod可以用于删页、插入其他文件、导出页面、修改各页相对顺序(通过鼠标拖动,有时比pdftk方便)、编辑索引(书签),修改文件属性。

\subsection{PDF元素级编辑}
pdfedit添加文字注释(英文),划线,删除文档元素,删加页面;感觉不太稳定,运行十分缓慢,经常在打开文件时内核转储。

可以使用inscape或gimp提取PDF中的一页,进行复杂的修改,然后使用pdf删除合并工具恢复成完整的PDF文件。

openoffice.org-pdfimport包让LibreOffice能够直接导入PDF文件,进行文字修改,再保存为odg图形文件,或者选择导出为pdf。但导入PDF时文件内的图片常常会丢失,排版可能会被破坏。所以LibreOffice目前还不能算作PDF阅读器或者编辑器。

flpsed可以添加英文文字,但可能会损坏PDF,使其不能被其他阅读器打开。

pdfstudio功能强大,但系付费软件。

\subsection{PDF元数据}
pdfmod可以修改文件元数据,如作者、标题等。pdfinfo命令行工具可以显示元数据。
pdffonts显示文件字体信息。

\subsection{PDF格式转换}
ImageMagick可以实现pdf和图片的相互转换, 使用convert命令。
\begin{verbatim}
 convert [input-options] input-file [output-options] output-file
\end{verbatim}

\begin{verbatim}
convert -density 700 -quality 100 draft.pdf draft.jpg
\end{verbatim}
详细内容参\ref{sec:imagemagick}。

cups-pdf用于将其他格式的文件如图片打印为pdf。cups-pdf打印保存位置由/etc/cups/cups-pdf.conf文件配置,一般为~/PDF,或者/var/spool/cups-pdf。

pdftotext实现pdf到文本的转换,效果一般不理想。

gpdftext是一款编辑器,可以直接导入pdf文件进行文字编辑,再保存成文本或pdf格式,但是会丢失所有除文字内容之外的信息,包括格式、排版、分页信息。

pdftohtml实现pdf到html的转换,效果往往不理想。可能需要指定编码格式,如
\verb+pdftohtml -enc GBK haha.pdf +

pdfimages提取pdf中的图片,默认保存为ppm格式。
\verb+pdfimages -j haha.pdf+。
j选项指定保存为jpg格式。

pdf2ps和ps2pdf实现pdf和ps之间的转换,基于ghostscript机制。当前ps2pdf默认使用ps2pdf14,即pdf为1.4版本。可以直接使用ps2pdf15.pdftops也能实现pdf到ps之间的转换。

pdf2djvu,pdf2dsc,pdftoppm, pdf2svg分别实现pdf到djvu,dsc,ppm, svg的转换。SVG可缩放矢量图形(Scalable Vector Graphics)是基于  svg logo可扩展标记语言(XML),用于描述二维矢量图形的一种图形格式。

chm2pdf可以将chm转换为pdf。



\section{访问铁道部网站}
网购火车票的网址是http://www.12306.cn.sixxs.org/mormhweb/kyfw/,这个是http的普通网页,里面有个iframe,也就是嵌套了另一个网页,地址是https://dynamic.12306.cn/otsweb/,这里就是https的了。因为dynamic.12306.cn使用的是SRCA颁发的证书,这个证书在我们的计算机中是默认不被信任的,也就是不安全的。


火狐下解决办法:
\begin{verbatim}
1.在authorities导入铁道部根证书,编辑->首选项->高级->加密->查看证书->导入
2.打开https://dynamic.12306.cn/otsweb/,然后有提示将该网站添加例外即可
\end{verbatim}
\section{wget下载网站}
例如:
\begin{shellcmd}
wget -r -p -np -k http://www.21cn.com
wget --ftp-user=... --ftp-password=PASS  -r -p -np -k ftp://www.21cn.com
wget -r -p -np -k http://192.168.1.199/svn/OCTEON-SDK/branches/2012-03-27/OCTEON-SDK/docs/ --user=limz --password=123456
\end{shellcmd}


\section{Windows的U盘启动与U盘安装}
在百度百科上对相关主题有详细解释。

常用的工具有:老毛桃WinPE,一键U盘装系统,微软发布的Windows7 USB/DVD Download tool,U速启。

UltraISO也可以制作启动盘。 

HopedotVOS提供一个可以在U盘上运行的虚拟操作系统。
