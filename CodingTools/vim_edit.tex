\section{Vim文本编辑}


\subsection{快速键入}
本节C表示Ctrl键。
在insert模式下:\\
C-W 删除当前单词(同Bash) \\
C-U 删除当前句子(同Bash)\\
C-P,C-N  自动补全,补全内容分别从前方或后方搜索\\
C-A 键入上次在INSERT模式下键入的内容,并进入INSERT模式\\
C-Y 键入上次在INSERT模式下键入的内容\\
C-X C-F 自动补全,文件名\\
C-X C-L 自动补全,整行\\
C-X C-D  自动补全,宏定义\\

在Normal模式下,CTRL A 和CTRL X分别将光标下的数字加1或减1

\subsection{快速配对区域操作}
快速处理\verb+''、""、()、[]、{}、<> +等配对标点符号中的文本内容,包括更改、删除、复制等。
\begin{verbatim}
ci(  快速修改()内的内容,即删除后进入insert模式
di(  快速删除()内的内容
yi(  快速复制()内的内容
\end{verbatim}
将上述i替换为a,()本身也会被选取。


\subsection{跨文档复制粘贴}
一般地,复制到指定寄存器的方法为:

普通模式下\verb|"+寄存器名+y|。

插入某寄存器内容的方法为:

普通模式下\verb|"+寄存器名+p|或编辑模式下\verb|Ctrl+R+寄存器名|。

有两个特殊的寄存器: 选择寄存器(寄存器``)为可视模式下选择的内容,剪贴板(寄存器+)为用图形界面选择的内容。


\subsection{跨文件字符替换}
每个文件在vim被称为缓冲区。

方法一:命令录制。
\begin{verbatim}
qq,:wnext,q.
\end{verbatim}

方法二:bufdo, argdo, windo等。
\begin{verbatim}
:wa
:bufdo %s/foo/bar/ge |update
\end{verbatim}

\subsection{文档统计}
显示当前文件名称与行数
\begin{verbatim}
:f 或 Ctrl+G
\end{verbatim}
显示某单词word出现的次数
\begin{verbatim}
%s/word//gn
\end{verbatim}
中文字数统计(近似方法)
\begin{verbatim}
%s/\S//gn
缺点是将非中文字符也算做一个字,如英文单词、标点等。
:%s/[^[:print:][:cntrl:]]//gn
缺点是中文标点也算作了汉字
:%s/[\u4e00-\u9fa5]//gn
使用unicode匹配,不成功,可能是因为utf8编码的文件不支持unicode匹配
\end{verbatim}


注意wc -m命令也可以统计字符数,但是把空白字符也算作内了。

\subsection{基于空格的列操作}
\verb+\S,\s+分别表示非空格和空格,后带\verb%\+%表示至少为一。再结合括号和数字引用,可以基于空格进行各种列操作。

例1:在非空行前后添加内容:
\verb#:%s/^\s*\(\S\+.*\)$/haha;\1wawa;#

例2:只保留第一列,其余删除:
\verb%5,12s/\(\S\+\)\s\+.*/\1/g%

\subsection{折行的开与关}
zR 打开所有折行;zr打开下一级折行;zM关闭所有折行;zm 关闭上一级折行
zo 打开折行;zc 重新关闭折行
zf 创建折行;zfap zf命令作用于ap(一个段落)

\subsection{屏幕错乱}
Ctrl+L即可。

\subsection{键入特殊字符}
外国人姓名之间的点号:fcitx激活时键入[即可。
\begin{verbatim}
:digraphs 查看VIM支持的特殊字符
\end{verbatim}
Insert模式使用CTRL-K {key1 key2}键入特殊字符。
如输入拼音a的四个声调,分别为\verb+a-(或a~),a‘,a<,a`(或a!)+。希腊字符\[\alpha,\beta\]分别为\verb+a*, b*+。

\subsection{键入Unicode字符}
Insert模式使用CTRL-V {digits}来插入一个由{digits}指定其ASCII码的字符. 

用这种方法你可以插入0到255的所有字符.如果你键入的数字少于两个, 那么Vim会在遇到一个非数字字符时终止这个命令. 要避免非得键入一个非数字字符才能让这个命令结束,你可以在数字前加上一个或两个0来凑足3个数.

\begin{verbatim}
CTRL-V 009.
\end{verbatim}

要用十六进制来表示你的ASCII, 在CTRL-V后面附加一个"x":
\begin{verbatim}
CTRL-V x7f
\end{verbatim}

接下来的两个方法还可以让你键入一个16bit或32bit的数字(比如,用来指定一个Unicode字符):
\begin{verbatim}
CTRL-V o123
CTRL-V u1234
CTRL-V U12345678
\end{verbatim}

\subsection{快速插入日期}
可以定义如下键盘映射,使得在Insert模式下,F8可以键入日期:
\begin{verbatim}
imap <F8> <Esc>:read !date +\%F<CR>i
使用了date系统命令,\是为了转义%号,在Bash中不需要
格式%F相当于%Y-%m-%d
\end{verbatim}

\subsection{大小写替换}
在替换命令中,一个字符前面加\verb+\u+会被转为大写,\verb+\l+会被转为小写。


\subsection{列选择模式}
Normal模式下CtrlV会进入列选择模式,可以用来对表格的列进行选择操作。


\subsection{缩写}
iab命令定义在Insert模式下的缩写,如

\begin{verbatim}
iab ssend SSEND_HAHA
\end{verbatim}
iunab取消缩写,iab列出缩写。

\subsection{C程序缩进}
设置缩进为4,并用空格代替TAB:
\begin{verbatim}
set softtabstop=4
set shiftwidth=4
set expandtab
\end{verbatim}
== 为当前行整理缩进
=a{ 为当前代码块整理缩进
=G 从当前行到文件结尾,整理缩进
gg=G 为整个文件整理缩进


\subsection{英文拼写检查}
\begin{verbatim}
:setlocal spell spelllang=en_us 
:set spell
:setlocal nospell
:set nospell
\end{verbatim}


\subsection{vim选项的值}
\begin{verbatim}
:se[t] 显示所有不等于默认值的选项
:set all 显示所有选项
:set option? 显示选项的值
:set option 对于Toggle类型的选项,将其值设置为on;其他类型的选项,显示其值
\end{verbatim}

\subsection{重新载入vimrc}
\begin{verbatim}
:source ~/.vimrc
:so ~/.vimrc
\end{verbatim}


