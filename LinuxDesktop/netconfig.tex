\section{主机网络配置}
\label{netconfig}
\subsection{配置IP地址和DNS}
可以使用图形工具network-manager-gnome.重启网络:系统设置-网络-关闭
\begin{shellcmd}
sudo vi /etc/network/interfaces 
\end{shellcmd}
\begin{verbatim}
auto lo
iface lo inet loopback
auto eth0 eth0:1
iface eth0 inet static
address 210.75.225.165
netmask 255.255.255.0
gateway 210.75.225.254
dns-nameservers 159.226.59.158 159.226.8.6
iface eth0:1 inet static
address 192.168.1.224
netmask 255.255.255.0
gateway 192.168.1.1
\end{verbatim}

新系统版本不希望手动更改/etc/resolv.conf文件,重启后会重置。修改/etc/resolvconf/tail文件,然后执行/etc/init.d/resolveconf restart,会自动生成/etc/resolve.conf文件。或者在上述/etc/network/interfaces添加dns-nameservers配置。

对于旧版系统,编辑/etc/resolv.conf文件,加入2行: 
\begin{verbatim}
nameserver 159.226.59.158
nameserver 159.226.8.6
\end{verbatim}
其他的DNS服务器包括:
谷歌:8.8.8.8
中国科大:202.38.64.1。


重启网络
\begin{shellcmd}
sudo /etc/init.d/networking restart
或者 sudo ifdown -a;sudo ifup -a;
\end{shellcmd}
如果配置了interfaces文件,那么network-manager-gnome无用,可以卸载

\subsection{代理设置}
使用tsocks这个软件。配置文件在/etc/tsocks.conf,修改local使其包含202.38.95.110。
其他工具还包括proxychains



