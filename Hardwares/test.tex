

\section{硬件测试相关知识}
\subsection{JTAG}
JTAG是联合测试工作组(Joint Test Action Group)的简称,是在名为标准测试访问端口和边界扫描结构的IEEE的标准1149.1的常用名称。此标准用于测试访问端口,使用边界扫描的方法来测试印刷电路板。

1990年JTAG正式由IEEE的1149.1-1990号文档标准化,在1994年,加入了补充文档对边界扫描描述语言(BSDL)进行了说明。从那时开始,这个标准被全球的电子企业广泛采用。边界扫描几乎成为了JTAG的同义词。

在设计印刷电路版时,目前最主要用在测试集成电路的副区块,而且也提供一个在嵌入式系统很有用的调试机制,提供一个在系统中方便的``后门''。当使用一些调试工具像电路内模拟器用JTAG当做讯号传输的机制,使得程式设计师可以经由JTAG去读取整合在CPU上的调试模组。调试模组可以让程式设计师调试嵌入式系统中的软件 。

\subsection{DUT}
被测器件(英语:device under test,DUT)或被测装置,又称在测单元或被测部件(unit under test,UUT),常用于表示正处于测试阶段的工业产品。
在半导体测试中,DUT表示晶圆或最终封装部件上的特定管芯小片。利用连接系统将封装部件连接到手动或自动测试设备(ATE),ATE会为其施加电源,提供模拟信号,然后测量和估计器件得到的输出,以这种方式测定特定被测器件的好坏。

对于晶圆来说,使用者需要将ATE用一组显微针连接到一个个独立的DUT(晶圆小片)。若晶圆已被切割成小片并封装,我们可以用ZIF插座(零插拔力插座)将ATE连接到DUT(管壳)上。

更多的情况下,DUT用于表示任何被测电子装置。例如,装配线下线的手机中的每一芯片都会被测试,而手机整机会以同样的方式进行最终的测试,这里的每一部手机都可以被称作DUT。

DUT常以测试针组成的针床测试台连接到ATE。

被测系统(System under test,SUT)表示正在被测试的系统,目的是测试系统是否能正确操作。这一词语常用于软件测试中。

软件系统测试的一个特例是对应用软件的测试,称为被测应用程序(application under test,AUT)。

SUT也表明软件已经到了成熟期,因为系统测试在测试周期中是集成测试的后一阶段。
