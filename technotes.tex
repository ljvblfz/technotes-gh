\documentclass[11pt,a4paper]{report}
\usepackage[top=1in,bottom=1in,left=1.2in,right=1.2in]{geometry}
\XeTeXlinebreaklocale "zh"
\XeTeXlinebreakskip = 0pt plus 1pt
\usepackage{indentfirst} 
\usepackage{fontspec}
\usepackage{titlesec}
\usepackage{xeCJK}
\usepackage{graphicx}
\usepackage{hyperref}
\usepackage{ifthen}
\usepackage{color,fancyvrb}
\usepackage{listings}
\usepackage{syntonly}
\usepackage{makeidx}

\usepackage{tikz}
\usetikzlibrary{calc}
\usetikzlibrary{fit}
\usetikzlibrary{positioning}
\usepgflibrary{plotmarks}

\usetikzlibrary{shapes.geometric}

\CustomVerbatimEnvironment{shellcmd}{Verbatim}
{frame=single,rulecolor=\color{blue},framerule=3pt,framesep=1pc,fillcolor=\color{yellow}}

\setCJKmainfont[BoldFont=SimHei,ItalicFont=KaiTi]{SimSun}
\setCJKsansfont{SimHei}
\renewcommand{\baselinestretch}{1.3}
\renewcommand{\figurename}{图} 
\renewcommand{\bibname}{\begin{center} \sffamily 参考文献 \end{center}}
\renewcommand{\today}{\number\year 年 \number\month 月 \number\day 日}
\renewcommand{\contentsname}{\sffamily 目录}
\newcommand{\bookname}{技术笔记}
\title{\sffamily \bookname}
\author{李明哲}
\date{\today}
\setcounter{tocdepth}{1}

\begin{document}

\maketitle
\addcontentsline{toc}{part}{目录}
\tableofcontents
\listoffigures
\listoftables

\titleformat{\part}{\huge\raggedright\sffamily}{第\thepart 部分}{1em}{}
\titleformat{\chapter}{\Large\raggedright\sffamily}{第\thechapter 章}{1em}{}
\titleformat{\section}{\large\raggedright\sffamily}{\thesection}{1em}{}
\titleformat{\subsection}{\normalfont\raggedright\sffamily}{\thesubsection}{1em}{}
\titleformat{\subsubsection}{\normalfont\raggedright\sffamily}{\thesubsubsection}{1em}{}

\part{基础知识}
\chapter{硬件}
\input{Chap-Hardwares}
\chapter{系统软件}
\input{Chap-SystemPlatforms}
\chapter{网络基础}
\input{Chap-NetworkBasics}
\chapter{数据库系统}
\input{Chap-DataBases}

\part{系统操作}
\chapter{Linux Desktop}
\input{Chap-LinuxDesktop}
\chapter{RSPE相关操作}
\input{Chap-RSPE}
\chapter{Linux服务器}
\input{Chap-LinuxServer}

\part{桌面应用操作}

\chapter{Latex与办公套件}
\input{Chap-LatexSkills}
\chapter{编程工具}
\input{Chap-CodingTools}
\chapter{其他桌面工具和技巧}
\input{Chap-GeneralDesktop}

\part{计算机程序设计}
\chapter{编程方法杂谈}
\input{Chap-ProgramMethod}
\chapter{Shell编程}
\input{Chap-ShellPrograms}
\chapter{C编程}
\input{Chap-CProgram}
\chapter{Python编程}
\input{Chap-Python}
\chapter{算法}
\input{Chap-Algorithm}



\begin{thebibliography}{99}
	\addcontentsline{toc}{part}{参考文献}

    \bibitem {acp}Donald Knuth, \emph{计算机程序设计的艺术},Volume 3, 第二版,国防工业出版社.2003
    \bibitem {pp}Jon Bentley著, 黄倩译.\emph{编程珠玑},第二版,人民邮电出版社,2008
    \bibitem {ita}Thomas H. Cormen, Charles E. Leiserson, Ronald L. Rivest, and Clifford Stein.\emph{算法导论},中文第二版.机械工业出版社
    \bibitem {sedgewick}Robert Sedgewick.\emph{Algorithm}, Edition 4.Addison-Wesley
    \bibitem {wikipedia}Wikipedia. \url{http://en.wikipedia.org/wiki/}
    \bibitem {weijipedia}维基百科.\url{http://zh.wikipedia.org/wiki/}
    \bibitem {pie} Eric Giguere, John Mongan.\emph{Programming Interviews Exposed, 3rd Edition}. John Wileys \& Sons, Inc .
    \bibitem {pibible} 欧立奇,刘洋,段韬.\emph{程序员面试宝典, 第3版}.电子工业出版社,2011.4 
    \bibitem {bop} 微软编程之美小组.\emph{编程之美}.电子工业出版社.2008.
    \bibitem {ms100}结构之法算法之道博客.\emph{微软面试100题}
    \bibitem {sword}何海涛.\emph{剑指offer:名企面试官精讲典型编程题}, 第1版.电子工业出版社.2012
    \bibitem {refractor}Martin Fowler著, 熊节译. \emph{重构:改善既有代码的设计}. 人民邮电出版社. 2010.4
    \bibitem {self}李明哲. \bookname (本笔记). 2012 - \number\year 
    \bibitem {tcpipill}W.Richard Stevens. TCP/IP Illustrated. Volume 1
    \bibitem {krc}Brian W.Kernighan, Dennis M.Ritchie, 徐宝文译.\emph{C程序设计语言},第2版.机械工业出版社

\end{thebibliography}









\end{document}

