
\section{Latex中文环境安装}
\label{sec:LatexInstall}.
\subsection{基于CJK宏包的方法}
Debian下相关软件包:
texlive texlive-latex-extra latex-cjk-chinese
参\ref{sec:zhCJK}。
CJK 作为一个 LaTeX2e 宏包,其对字体的定义遵照 NFSS(新字体选择框架)。Dpkg环境只提供了gbsn,gkai两种简体中文字体。可以使用一些网上流传的工具(如gbkfonts)将ttl字体转换为CJK宏包需要的PK字体,很繁琐,勿尝试。

\subsection{基于xeCJK宏包的方法}
Debian下相关软件包:
texlive texlive-latex-extra texlive-xetex

xeCJK的好处就是能够直接使用系统安装的字体。

\subsection{CTEX}
Ubuntu原生软件包系统不提供对CTEX的安装,只能到网上下载相关文件,复制到目录\verb+/usr/share/texmf-texlive/tex/latex/ctex/fontset+下,然后执行sudo texhash命令。

字体配置路径可能是
\begin{verbatim}
/usr/local/texlive/tex/latex/ctex/fontset/
/usr/local/texlive/2012/texmf-dist/tex/latex/ctex/fontset
\end{verbatim}
相关文件包括: ctex-xecjk-winfonts.def

\subsection{从CTAN安装texlive}
\url{http://oss.ustc.edu.cn/CTAN/systems/texlive}

\url{http://mirrors.tuna.tsinghua.edu.cn/CTAN/systems/texlive}

\subsection{背景知识}
texhash又叫mktexlsr,用于刷新ls-R文件名数据库。

CTAN全称是Comprehensive Tex Archive Network.The TeX Archive Network is a set of fully-mirrored ftp sites providing the most complete, up-to-date TeX-related software possible.





