
\section{cut命令用法笔记}

cut是一个选取命令,就是将一段数据经过分析,取出我们想要的。一般来说,选取信息通常是针对“行”来进行分析的,并不是整篇信息分析的。

其语法格式为:
\begin{verbatim}
cut  [-bn] [file] 
cut [-c] [file]  
cut [-df] [file]
\end{verbatim}

使用说明
cut 命令从文件的每一行剪切字节、字符和字段并将这些字节、字符和字段写至标准输出。
如果不指定 File 参数,cut 命令将读取标准输入。必须指定 -b、-c 或 -f 标志之一。
主要参数

\begin{itemize}
    \item -b :以字节为单位进行分割。这些字节位置将忽略多字节字符边界,除非也指定了 -n 标志。
    \item -c :以字符为单位进行分割。
    \item -d :自定义分隔符,默认为制表符tab。
    \item -f  :与-d一起使用,指定显示哪个区域。
\end{itemize}

举例:
\begin{verbatim}
提取who命令输出每一行的第3个字节
who | cut -b 3

提取who命令输出每一行的第3,4,5,8字节
who|cut -b 3-5,8

paswd文件的内容是由冒号分割的域,可以用cut提取某些域。下面的命令提取前5行的某些域。
cat /etc/passwd|head -n 5|cut -d : -f 1,3-5,7

获取当前年份
date | cut -d ' ' -f 1

\end{verbatim}
