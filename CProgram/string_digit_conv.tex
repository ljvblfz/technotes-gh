\section{c语言字符串数字转换函数}
\begin{verbatim}
atof(将字符串转换成浮点型数)
atoi(将字符串转换成整型数)
atol(将字符串转换成长整型数)
strtod(将字符串转换成浮点数)
strtol(将字符串转换成长整型数)
strtoul(将字符串转换成无符号长整型数):经测试,如果字符串表示的整数大于unsigned long的范围,会转换成2^32-1
toascii(将整型数转换成合法的ASCII 码字符)
toupper(将小写字母转换成大写字母)
tolower(将大写字母转换成小写字母)


     atof(将字符串转换成浮点型数)
相关函数 atoi,atol,strtod,strtol,strtoul
表头文件 #include <stdlib.h>
定义函数 double atof(const char *nptr);
函数说明 atof()会扫描参数nptr字符串,跳过前面的空格字符,直到遇上数
     字或正负符号才开始做转换,而再遇到非数字或字符串结束时
     ('/0')才结束转换,并将结果返回。参数nptr字符串可包含正负
     号、小数点或E(e)来表示指数部分,如123.456或123e-2。
 返回值 返回转换后的浮点型数。
附加说明 atof()与使用strtod(nptr,(char**)NULL)结果相同。
  范例 /* 将字符串a 与字符串b转换成数字后相加*/
     #include<stdlib.h>
     main()
     {
     char *a=”-100.23”;
     char *b=”200e-2”;
     float c;
     c=atof(a)+atof(b);
     printf(“c=%.2f/n”,c);
     }
  执行 c=-98.23
     atoi(将字符串转换成整型数)
相关函数 atof,atol,atrtod,strtol,strtoul
表头文件 #include<stdlib.h>
定义函数 int atoi(const char *nptr);
函数说明 atoi()会扫描参数nptr字符串,跳过前面的空格字符,直到遇上数
     字或正负符号才开始做转换,而再遇到非数字或字符串结束时
     ('/0')才结束转换,并将结果返回。
 返回值 返回转换后的整型数。
附加说明 atoi()与使用strtol(nptr,(char**)NULL,10);结果相同。
  范例 /* 将字符串a 与字符串b转换成数字后相加*/
     #include<stdlib.h>
     mian()
     {
     char a[]=”-100”;
     char b[]=”456”;
     int c;
     c=atoi(a)+atoi(b);
     printf(c=%d/n”,c);
     }
  执行 c=356
     atol(将字符串转换成长整型数)
相关函数 atof,atoi,strtod,strtol,strtoul
表头文件 #include<stdlib.h>
定义函数 long atol(const char *nptr);
函数说明 atol()会扫描参数nptr字符串,跳过前面的空格字符,直到遇上数
     字或正负符号才开始做转换,而再遇到非数字或字符串结束时
     ('/0')才结束转换,并将结果返回。
 返回值 返回转换后的长整型数。
附加说明 atol()与使用strtol(nptr,(char**)NULL,10);结果相同。
  范例 /*将字符串a与字符串b转换成数字后相加*/
     #include<stdlib.h>
     main()
     {
     char a[]=”1000000000”;
     char b[]=” 234567890”;
     long c;
     c=atol(a)+atol(b);
     printf(“c=%d/n”,c);
     }
  执行 c=1234567890
     gcvt(将浮点型数转换为字符串,取四舍五入)
相关函数 ecvt,fcvt,sprintf
表头文件 #include<stdlib.h>
定义函数 char *gcvt(double number,size_t ndigits,char *buf);
函数说明 gcvt()用来将参数number转换成ASCII码字符串,参数ndigits表示
     显示的位数。gcvt()与ecvt()和fcvt()不同的地方在于,gcvt()所
     转换后的字符串包含小数点或正负符号。若转换成功,转换后的字
     符串会放在参数buf指针所指的空间。
 返回值 返回一字符串指针,此地址即为buf指针。
附加说明
  范例 #include<stdlib.h>
     main()
     {
     double a=123.45;
     double b=-1234.56;
     char *ptr;
     int decpt,sign;
     gcvt(a,5,ptr);
     printf(“a value=%s/n”,ptr);
     ptr=gcvt(b,6,ptr);
     printf(“b value=%s/n”,ptr);
     }
  执行 a value=123.45
     b value=-1234.56
     strtod(将字符串转换成浮点数)
相关函数 atoi,atol,strtod,strtol,strtoul
表头文件 #include<stdlib.h>
定义函数 double strtod(const char *nptr,char **endptr);
函数说明 strtod()会扫描参数nptr字符串,跳过前面的空格字符,直到遇上
     数字或正负符号才开始做转换,到出现非数字或字符串结束时
     ('/0')才结束转换,并将结果返回。若endptr不为NULL,则会将遇
     到不合条件而终止的nptr中的字符指针由endptr传回。参数nptr字
     符串可包含正负号、小数点或E(e)来表示指数部分。如123.456或
     123e-2。
 返回值 返回转换后的浮点型数。
附加说明 参考atof()。
  范例 /*将字符串a,b,c 分别采用10,2,16 进制转换成数字*/
     #include<stdlib.h>
     mian()
     {
     char a[]=”1000000000”;
     char b[]=”1000000000”;
     char c[]=”ffff”;
     printf(“a=%d/n”,strtod(a,NULL,10));
     printf(“b=%d/n”,strtod(b,NULL,2));
     printf(“c=%d/n”,strtod(c,NULL,16));
     }
  执行 a=1000000000
     b=512
     c=65535
     strtol(将字符串转换成长整型数)
相关函数 atof,atoi,atol,strtod,strtoul
表头文件 #include<stdlib.h>
定义函数 long int strtol(const char *nptr,char **endptr,int base);
函数说明 strtol()会将参数nptr字符串根据参数base来转换成长整型数。参
     数base范围从2至36,或0。参数base代表采用的进制方式,如base
     值为10则采用10进制,若base值为16则采用16进制等。当base值为0
     时则是采用10进制做转换,但遇到如'0x'前置字符则会使用16进制
     做转换。一开始strtol()会扫描参数nptr字符串,跳过前面的空格
     字符,直到遇上数字或正负符号才开始做转换,再遇到非数字或字
     符串结束时('/0')结束转换,并将结果返回。若参数endptr不为
     NULL,则会将遇到不合条件而终止的nptr中的字符指针由endptr返
     回。
 返回值 返回转换后的长整型数,否则返回ERANGE并将错误代码存入errno
     中。
附加说明 ERANGE指定的转换字符串超出合法范围。
  范例 /* 将字符串a,b,c 分别采用10,2,16进制转换成数字*/
     #include<stdlib.h>
     main()
     {
     char a[]=”1000000000”;
     char b[]=”1000000000”;
     char c[]=”ffff”;
     printf(“a=%d/n”,strtol(a,NULL,10));
     printf(“b=%d/n”,strtol(b,NULL,2));
     printf(“c=%d/n”,strtol(c,NULL,16));
     }
  执行 a=1000000000
     b=512
     c=65535
     strtoul(将字符串转换成无符号长整型数)
相关函数 atof,atoi,atol,strtod,strtol
表头文件 #include<stdlib.h>
定义函数 unsigned long int strtoul(const char *nptr,char
     **endptr,int base);

函数说明 strtoul()会将参数nptr字符串根据参数base来转换成无符号的长整
     型数。参数base范围从2至36,或0。参数base代表采用的进制方
     式,如base值为10则采用10进制,若base值为16则采用16进制数
     等。当base值为0时则是采用10进制做转换,但遇到如'0x'前置字符
     则会使用16进制做转换。一开始strtoul()会扫描参数nptr字符串,
     跳过前面的空格字符串,直到遇上数字或正负符号才开始做转换,
     再遇到非数字或字符串结束时('/0')结束转换,并将结果返回。若
     参数endptr不为NULL,则会将遇到不合条件而终止的nptr中的字符
     指针由endptr返回。
 返回值 返回转换后的长整型数,否则返回ERANGE并将错误代码存入errno
     中。
附加说明 ERANGE指定的转换字符串超出合法范围。
  范例 参考strtol()
     toascii(将整型数转换成合法的ASCII 码字符)
相关函数 isascii,toupper,tolower
表头文件 #include<ctype.h>
定义函数 int toascii(int c)
函数说明 toascii()会将参数c转换成7位的unsigned char值,第八位则会被
     清除,此字符即会被转成ASCII码字符。
 返回值 将转换成功的ASCII码字符值返回。
  范例 #include<stdlib.h>
     main()
     {
     int a=217;
     char b;
     printf(“before toascii () : a value =%d(%c)/n”,a,a);
     b=toascii(a);
     printf(“after toascii() : a value =%d(%c)/n”,b,b);
     }
  执行 before toascii() : a value =217()
     after toascii() : a value =89(Y)
     tolower(将大写字母转换成小写字母)
相关函数 isalpha,toupper
表头文件 #include<stdlib.h>
定义函数 int tolower(int c);
函数说明 若参数c为大写字母则将该对应的小写字母返回。
 返回值 返回转换后的小写字母,若不须转换则将参数c值返回。
附加说明
  范例 /* 将s字符串内的大写字母转换成小写字母*/
     #include<ctype.h>
     main()
     {
     char s[]=”aBcDeFgH12345;!#$”;
     int i;
     printf(“before tolower() : %s/n”,s);
     for(i=0;I<sizeof(s);i++)
     s[i]=tolower(s[i]);
     printf(“after tolower() : %s/n”,s);
     }
  执行 before tolower() : aBcDeFgH12345;!#$
     after tolower() : abcdefgh12345;!#$
     toupper(将小写字母转换成大写字母)
相关函数 isalpha,tolower
表头文件 #include<ctype.h>
定义函数 int toupper(int c);
函数说明 若参数c为小写字母则将该对映的大写字母返回。
 返回值 返回转换后的大写字母,若不须转换则将参数c值返回。
附加说明
  范例 /* 将s字符串内的小写字母转换成大写字母*/
     #include<ctype.h>
     main()
     {
     char s[]=”aBcDeFgH12345;!#$”;
     int i;
     printf(“before toupper() : %s/n”,s);
     for(i=0;I<sizeof(s);i++)
     s[i]=toupper(s[i]);
     printf(“after toupper() : %s/n”,s);
     }
  执行 before toupper() : aBcDeFgH12345;!#$
     after toupper() : ABCDEFGH12345;!#$



 
\end{verbatim}
