\section{Doxygen文档生成工具}
\begin{verbatim}
http://www.stack.nl/~dimitri/doxygen/index.html

http://sourceforge.net/projects/doxygen/
\end{verbatim}

\begin{lstlisting}
doxygen -g
doxygen Doxyfile
\end{lstlisting}

要点:
\begin{enumerate}
	\item 修改Doxyfile,可以指定项目名称,概要,语言(Chinese)等。Doxyfile中比较重要的字段还包括
	\begin{enumerate}
		\item OUTPUT\_LANGUAGE
		\item OPTIMIZE\_OUTPUT\_FOR\_C
		\item EXTRACT\_ALL, EXTRACT\_STATIC
	\end{enumerate}
	\item \verb+ \mainpage \author +等命令比较常用
	\item 无需学习过多内容
\end{enumerate}

在vim中安装doxygen-toolkit插件,常用命令包括
\begin{description}
    \item[:Dox]生成函数或类的注释
    \item[:DoxLic]生成版权信息,默认为GPL
    \item[:DoxAuthor]生成作者信息
\end{description}

为配置该插件,可在.vimrc中加以配置:
\begin{lstlisting}
"for doxygen-toolkit
let g:DoxygenToolkit_authorName="Li Mingzhe, limz@dsp.ac.cn" 
let s:licenseTag = "Copyright(C) National Network New Media Engineering Center(3NM)\<enter>\<enter>"
let s:licenseTag = s:licenseTag . "All rights reserved. 3NM’s source code is an unpublished work and the\<enter>"
let s:licenseTag = s:licenseTag . "use of a copyright notice does not imply otherwise. This source code contains\<enter>"
let s:licenseTag = s:licenseTag . "confidential, trade secret material of 3NM, Inc. Any attempt or participation\<enter>"
let s:licenseTag = s:licenseTag . "in deciphering, decoding, reverse engineering or in any way altering the source\<enter>"
let s:licenseTag = s:licenseTag . "code is strictly prohibited, unless the prior written consent of 3NM\<enter>"
let s:licenseTag = s:licenseTag . "is obtained.\<enter>"
\end{lstlisting}



