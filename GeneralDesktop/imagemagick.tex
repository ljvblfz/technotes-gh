\section{ImageMagick工具用法}
ImageMagic包含display,convert, identify, mogrify, montage, compare等工具。
\label{sec:imagemagick}

display用来看图片,如观看当前路径下所有图片,有
\begin{verbatim}
display *
\end{verbatim}

identify用来显示图片信息。

convert用来转换图片:
\begin{verbatim}
convert input-file [options] output-file
\end{verbatim}

pdf和图片相互转换,有
\begin{verbatim}
convert *.png output.pdf
convert haha.pdf 1.png pdf文件转换为前缀为1的png文件
\end{verbatim}

旋转图片
\begin{verbatim}
convert -rotate 90 image.jpg image.png
convert -flip a.jpg b.jpg 上下翻转
convert -flop a.jpg b.jpg 左右翻转
\end{verbatim}

图像加框
\begin{verbatim}
convert -border 60*60 “#000000″ a.jpg b.jpg
60*60是表示边框的宽度,第一个是纵边框的宽度,第二个是横边框的宽度
#000000是RGB格式的边框色彩
\end{verbatim}



以下命令用于改变图片大小(宽×高):
\begin{verbatim}

1. 默认时,宽度和高度表示要最终需要转换图像的最大尺寸,同时Convert会控制图片的宽和高,保证图片按比例进行缩放。

如:convert -resize 600×600 src.jpg dst.jpg

转换后的dst.jpg的图片大小(宽度为600,而高度已经按比例调整为450).

2.如果需要转换成600×600,而图片无需保持原有比例,可以在宽高后面加上一个感叹号!.

如:convert -resize 600×600! src.jpg dst.jpg

3. 只指定高度,图片会转换成指定的高度值,而宽度会按原始图片比例进行转换。

如:convert -resize 400 src.jpg dst.jpg

转换后的dst.jpg的图片大小(宽度为400,而高度已经按比例调整为300),和例1有点类似。

4. 默认都是使用像素作为单位,也可以使用百分比来形象图片的缩放。

如:convert -resize 50%x100%! src.jpg dst.jpg 或者convert -resize 50%x100% src.jpg dst.jpg

此参数只会按你的比例计算后缩放,不保持原有比例。(结果尺寸为100×150)

5.使用 @ 来制定图片的像素个数。

如:convert -resize “10000@” src.jpg dst.jpg

此命令执行后,dst.jpg图片大小为(115×86),图片保持原有比例(115×86= 9080 < 10000)。

6.当原始文件大于指定的宽高时,才进行图片放大缩小,可使用>命令后缀。

如:convert -resize “100×50>” src.jpg dst.jpg

此命令执行后,dst.jpg图片大小为(67×50),图片保持原有比例。

如:convert -resize “100×50>!” src.jpg dst.jpg

此命令执行后,dst.jpg图片大小为(100×50),图片不保持原有比例。

7.当原始文件小于指定的宽高时,才进行图片放大转换,可使用<命令后缀。

如:convert -resize “100×500<” src.jpg dst.jpg 或者convert -resize “100×100<!” src.jpg dst.jpg

此命令执行后,dst.jpg和src.jpg大小相同,因为原始图片宽比100大。

如:convert -resize “600×600<” src.jpg dst.jpg

此命令执行后,dst.jpg图片大小为(600×450),图片保持原有比例。

如:convert -resize “600×600<!” src.jpg dst.jpg

此命令执行后,dst.jpg图片大小为(600×600),图片不保持原有比例。

8.使用^命令后缀可以使用宽高中较小的那个值作为尺寸

如:convert -resize “300×300^” src.jpg dst.jpg

此命令执行后,dst.jpg图片大小为(400×300),图片保持原有比例,(300:300 < 200:150,选择高作为最小尺寸)。

如:convert -resize “300×200^” src.jpg dst.jpg

此命令执行后,dst.jpg图片大小为(300×225),图片保持原有比例,(300:200 > 200:150,选择宽作为最小尺寸)。


\end{verbatim}
