\section{awk语言}
Awk是功能完整的文本处理语言, 使用类似于C的语法. 它具有一整套操作符和能力集, 我们只在这里讲解一小部分 - 也就是在shell脚本中最有用的部分.

Awk将传递进来的每行输入都分割成域. 默认情况下, 一个域指的就是使用空白分隔的一个连续字符串, 不过我们可以修改属性来改变分隔符. Awk将会分析并操作每个分割域. 因为这种特性, 所以awk非常善于处理结构化的文本文件 -- 尤其是表 -- 将数据组织成统一的块, 比如说分成行和列.

强引用(单引号)和大括号用来包含shell脚本中的awk代码段.

\begin{verbatim}
echo one two | awk '{print $1}'
# one

echo one two | awk '{print $2}'
# two

awk '{print $3}' $filename
# 打印文件$filename的域#3, 到stdout. 
 
awk '{print $1 $5 $6}' $filename
# 打印文件$filename的域#1, #5, 和#6. 
\end{verbatim}


\section{cut命令用法笔记}

cut是一个选取命令,就是将一段数据经过分析,取出我们想要的。一般来说,选取信息通常是针对“行”来进行分析的,并不是整篇信息分析的。

其语法格式为:
\begin{verbatim}
cut  [-bn] [file] 
cut [-c] [file]  
cut [-df] [file]
\end{verbatim}

使用说明
cut 命令从文件的每一行剪切字节、字符和字段并将这些字节、字符和字段写至标准输出。
如果不指定 File 参数,cut 命令将读取标准输入。必须指定 -b、-c 或 -f 标志之一。
主要参数

\begin{itemize}
    \item -b :以字节为单位进行分割。这些字节位置将忽略多字节字符边界,除非也指定了 -n 标志。
    \item -c :以字符为单位进行分割。
    \item -d :自定义分隔符,默认为制表符tab。
    \item -f  :与-d一起使用,指定显示哪个区域。
\end{itemize}

举例:
\begin{verbatim}
提取who命令输出每一行的第3个字节
who | cut -b 3

提取who命令输出每一行的第3,4,5,8字节
who|cut -b 3-5,8

paswd文件的内容是由冒号分割的域,可以用cut提取某些域。下面的命令提取前5行的某些域。
cat /etc/passwd|head -n 5|cut -d : -f 1,3-5,7

获取当前年份
date | cut -d ' ' -f 1

\end{verbatim}
\section{find和xargs}

\subsection{xargs命令}
执行一条命令,其参数从标准输入获取。

其用途之一是用于将参数分成多行输入,类似于行末添加了反斜杠。例如输入xargs file,分多行输入file命令的参数,用CtrlD终止输入。

用途之二是构造管道,第一个命令的输入并不连接到第二个命令的输入,而是其命令行参数。如

\begin{verbatim}
ls | xargs file
find /tmp -name core -type f -print0 | xargs -0 /bin/rm -f
\end{verbatim}

\subsection{find命令}
\begin{verbatim}
find [-H] [-L] [-P] [-D debugopts] [-Olevel] [path...] [expression]
\end{verbatim}
H,L,P选项控制是否进入符号链接。D,O选项也不太常用。
expression包含选项、测试、行动三部分,因此find命令的使用方式通常为如下形式:

\verb+find 起始目录 寻找条件 操作+

寻找条件可以用and,or,not连接起来,分别写作:

\verb+ -a -o 和 !+

如

\verb+ find ! -name 'tmp'+

\verb+-prune选项相当与-maxdetph 0+,表示只对path参数包含的路径进行操作,不进行递归。

需要说明的是:当使用很多的逻辑选项时,可以用括号把这些选项括起来。为了避免Shell本身对括号引起误解,在话号前需要加转义字符来去除括号的意义。例:

\verb+find \(–name ’tmp’ -a type c -user ’inin’ \)+


\subsection{只显示文件夹}
只显示当前目录下的文件夹:

\verb+ls -l | grep ^d+
\verb+find * -type d -prune+
\verb+find . -maxdepth 1 -mindepth 1 -type d+

只显示当前目录下的非文件夹:

\verb+ls -l | grep -v ^d+

\section{sed的使用}

用法: sed [选项]\ldots {脚本(如果没有其他脚本)} [输入文件]\ldots

\begin{verbatim}
-n, --quiet, --silent
    取消自动打印模式空间
-e 脚本, --expression=脚本
    添加“脚本”到程序的运行列表
-f 脚本文件, --file=脚本文件
    添加“脚本文件”到程序的运行列表
-i[扩展名], --in-place[=扩展名]
   直接修改文件(如果指定扩展名就备份文件)
\end{verbatim}

下面主要谈论sed脚本语言。


sed命令主要有:
\begin{description}
    \item[s/match/replace]替换
    \item[i\textbackslash\ TEXT]前插(insert)文本
    \item[a\textbackslash\ TEXT]后插(append)文本
    \item[r filename]后插文件
    \item[d]删除行
    \item[p]打印行
\end{description}

sed命令之前需要有0~2个量,被称作''地址''。addr1,addr2形式为地址范围,闭区间。addr1,addr2!后面的叹号表示取反。addr1,+N形式为addr及其后N行。addr1~N为自addr1起以N为步长的所有行。由 /regexp/表示regexp匹配到的行。

替换命令 举例:
\begin{verbatim}
echo 123456|sed -e s/123/wolf/g > haha4
more haha3|sed -e s/123/wolf/g > haha4
sed -e 's/<[^>]*>//g' haha.xml     (删除haha.xml中所有的xml标记)
\end{verbatim}

删除行命令举例:
\begin{verbatim}
sed -e /^$/d haha >> haha2  //删除文件haha中的空行
sed -e /33/d haha > haha2  //删除haha中包含33的行
sed -e 2d haha > haha2  //删除第2行
sed -e 2,4d haha > haha2  //删除第2~4行
\end{verbatim}

提取行命令:
\begin{verbatim}
sed -n '3,10p' myfile > newfile //提取从第3行到第10行
\end{verbatim}

提取不定行:
\begin{verbatim}
i=2
sed -n ${i}p myfile
\end{verbatim}


\section{sort用法笔记}
比较重要的选项包括:

\begin{itemize}
    \item o 指定输出文件
    \item r 逆序
    \item k=pos 指定关键词在第几列,默认为第一列
    \item n 按照数值大小排序;默认为按字典序
    \item u 删除重复行
    \item t 指定field分隔符;默认为空格
    \item f 忽略大小写 
\end{itemize}


举例:
\begin{verbatim}
sort -n -t ‘ ‘ -k 3r -k 2 facebook.txt
sort -t ‘ ‘ -k 1 facebook.txt
\end{verbatim}

sort命令选项众多,十分强大,但我此时意识到不常用的命令和选项对我来说是毫无意义的。
\section{uniq命令用法笔记}

uniq命令用来检查或忽略文件中重复的行。重要选项有:

\begin{itemize}
    \item -u 只显示不重复的行
    \item -d 只显示重复的行
    \item -c 显示重复次数
\end{itemize}


