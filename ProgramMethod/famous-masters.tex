\section{大师辑录}
\begin{description}
\item[Donald E. Knuth(高德纳)]
 Tex作者,费时十年作Tex。《计算机程序设计艺术》(\emph{The Art of Computer Programming})的作者

\item[Guy Lewis Steele Jr.(盖伊·路易士·史提尔二世)]
(1954年10月2日-)绰号为The Great Quux,GLS,生于美国密苏里州,计算机科学家,曾与理查德·斯托曼共同开发了Emacs,也是Scheme的共同作者,在编程语言方面有很大的贡献。

\item[Leslie Lamport(莱斯利·兰伯特)]
LaTeX(,音译“拉泰赫”)是一种基于TeX的排版系统,由美国电脑学家莱斯利·兰伯特(Leslie Lamport)在20世纪80年代初期开发,利用这种格式,即使使用者没有排版和程序设计的知识也可以充分发挥由TEX所提供的强大功能,能在几天,甚至几小时内生成很多具有书籍质量的印刷品。
\item[Dimitri van Heesch]
doxygen作者
\item[Brian Wilson Kernighan 布莱恩·威尔森·柯林汉]
(1942年-)
生于加拿大多伦多,加拿大计算机科学家,曾服务于贝尔实验室,为普林斯顿大学教授。他曾参与Unix的研发,也是AMPL与AWK的共同创造者之一。与丹尼斯·里奇共同写作了C语言的第一本著作之后,他的名字开始为人所熟知。他也创作了许多Unix上的程式,包括在Version 7 Unix上的 ditroff 与 cron。

\item[Peter Jay Weinberger 彼得·杰·温伯格] 
(1942年8月6日-),美国著名计算机科学家,曾服务于贝尔实验室,现在google工作。他是AWK的共同作者之一。
\item[Alfred Aho 阿尔佛雷德·艾侯]
(1941年8月9日-),生于加拿大安大略省提明斯(Timmins),担任哥伦比亚大学的劳伦斯科斯曼计算机科学教授。阿尔佛雷德·艾侯最有名的著作,是与 彼得·温伯格和布莱恩·柯林汉合著的《AWK程式设计》,A就是其姓氏“Aho ”的缩写。另外还有他与 Ravi Sethi以及Jeffrey Ullman合著的《编译器:原理、技术、工具》(又称龙书或天龙宝典,松岗有出版中译本,分为上下两册)。他也写了Unix底下egrep和fgrep工具的最初版本。同时也与Ullman和John Hopcroft著作大量计算机科学领域的参考书,包括算法、数据结构以及计算机科学基础。

\item[Alan Curtis Kay 艾伦·凯]
(1940年5月17日—),英文原名Alan Curtis Kay,美国计算机科学家,在面向对象编程和窗口式图形用户界面方面作出了先驱性贡献。2003年获得图灵奖。目前担任Viewpoints研究院院长,加州大学伯克利分校兼职教授。曾任Apple公司院士,惠普公司资深院士。
\item[Edsger Wybe Dijkstra 艾兹赫尔·戴克斯特拉]
(Edsger Wybe Dijkstra,1930年5月11日-2002年8月6日),荷兰计算机科学家,毕业就职于荷兰莱顿大学,早年钻研物理及数学,而后转为计算学。曾在1972年获得过素有计算机科学界的诺贝尔奖之称的图灵奖,之后,他还获得过1974年AFIPS Harry Goode Memorial Award、1989年ACM SIGCSE计算机科学教育教学杰出贡献奖。他曾经提出“GOTO有害论”信号量和PV原语,解决了有趣的“哲学家就餐问题”。他的贡献包括:
提出了目前离散数学应用广泛的最短路径算法(Dijkstra's Shortest Path First Algorithm)
为解决操作系统中资源分配问题,提出银行家算法
\item[Jeffrey D. Ullman]
数据库理论、自动机理论、编译原理大师。他的《Automata Theory, Languages, and Computation》让我真正的进入了计算机基础理论的世界。《Compilers: Principles, Techniques, and Tools 》让编译器不再神奇,让我也能写出自己的编译器。《A First Course in Database Systems》让我对数据库的了解从应用进入了理论的深度,可以说Ullman是我在计算机理论方面的启蒙老师,他的书教给了我计算机世界最奇妙最基础最有趣的东西。
\item[Andrew S. Tanenbaum]
操作系统专家和网络专家,Minix的创始人,追根溯底也是Linux的祖师爷。他的《Modern Operating Systems》和《Operating Systems Design and Implementation》以及Minix系统让我了解了操作系统的原理并且有机会通过实践来学习操作系统原理,有了他的教导,使操作系统对我而言不再神秘。《Computer Networks》则让我对网络从感觉上的认识进入了理性的认识,从理论上了解了计算机网络的结构和理论。Tanenbaum让我知道了我天天使用的东西原来是这样的。

\item[Tim Berners-Lee(蒂姆·伯纳斯·李)]
 World Wide Web 的发明人,他为互联网带来的东西影响了全世界的人民。
\item[Dennis Ritchie(丹尼斯·里奇)]
C语言之父!Dennis Ritchie对许多领域都深有研究,包括:C, ALTRAN, B, BCPL和Unix.在C方面他尤其产生了重大的影响,他与Brian Kernighan合著的书:The C Programming Language无疑是历史上最好的编程书籍.
\item[Bram Cohen(布拉姆·科恩)]
BitTorrent 技术之父!Bram编写了BitTorrent协议允许文件共享。如今这项技术正应用在互联网的各个角落。
\item[Rasmus Lerdorf, Andi Gutmans\&Zeev Suraski]
PHP之父!如今PHP已运行于互联网上34\%的网页。
\item[Bjarne Stroustrup]
C++之父。没有他从C到C++引领的这一步,我们会怎么样呢?

\item[Jim Gray]



Douglas McIlroy
P. J. Plauger












\end{description}

