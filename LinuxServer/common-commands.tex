\section{常用命令详解}

\subsection{ls}
解释几个比较实用的选项。

\subsubsection{l选项}
会产生类似如下输出。
\begin{verbatim}
srw-rw-rw-   1 root root             0 11月  3 15:45 log=
crw-------   1 root root       10, 237 11月  3 15:45 loop-control
-rwsr-xr-x   1 root root 45420  7月 27 01:07 /usr/bin/passwd*
\end{verbatim}
其中第1个字符表示文件类型。s表示socket,p表示FIFO(命名管道),b表示块设备,l表示符号链接,-表示普通文件,?表示未知类型。

在文件权限中,s表示设置用户(组)id位为1,且文件可执行。t表示粘住位为1,且文件可执行。相应的S和T表示文件不可执行。

第2个位段表示硬链接数。第3、4位分别表示用户和组,-n选项让二者用数字而非名字表示。第5位表示size。第6位表示修改日期。最后一位为文件名。

\subsubsection{R选项}
递归输出。
\subsubsection{F选项}
文件名后面加上一个符号,表示文件类型。如/表示目录,@表示符号链接,|表示FIFO,=表示socket。普通文件后面什么也没有。
\subsubsection{ld选项}
-d选项表示显示的为目录本身的信息,而非其子目录或包含文件的信息。
\subsubsection{lt选项}
t选项表示安装修改时间排序。
\subsubsection{ltr选项}
表示按照t选项的排序的反序显示。
\subsubsection{c选项}
表示按照inode ctime值排序,而不是修改时间。
\subsubsection{i选项}
显示inode号。

\subsection{grep}

略
\subsection{find}


略



