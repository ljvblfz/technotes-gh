\section{Ubuntu/Deepin安装后的配置}


\subsection{修改配置文件}
配置网络参\ref{netconfig}.

磁盘分区配置文件包括/etc/fstab, 参\ref{diskpartition}.

其他配置文件包括.bashrc, .vimrc, /etc/hosts, 针对git的.netrc等.

\subsection{软件安装}

软件源编辑器:software-properties-gtk

包清单:
vim, vim-gnome, vim-scripts, vim-doc, vim-latexsuite,ntfs-config, pysdm,gnome-tweak-tool,nautilus-open-terminal,gdebi,Pidgin,chromium-browser, aMSN, sshpass, lcrt, texlive, texlive-xetex, texlive-latex-extra, git

Ubuntu源中未收录的软件:
yozooffice, wpsoffice, acroread, yunio, nautilus-nutstore, mendeleydesktop, google-chrome, ubuntu-tweak, pidgin-lwqq。

永中office,WPS和yunio可能被深度源收录,在LinuxDeepin下直接搜索安装即可。

pidgin的WebQQ插件,和UbuntuTweak可通过PPA安装:
\begin{lstlisting}
ppa:lainme/lwqq
ppa:tualatrix/ppa
\end{lstlisting}

Adobe Reader被Canonical源收录,名称为acroread:
\begin{lstlisting}
sudo add-apt-repository ``deb http://archive.canonical.com/ precise partner''
sudo apt-get update
sudo apt-get install acroread
\end{lstlisting}

其他软件也类似:
\begin{lstlisting}
Mendeley
deb http://download.mendeley.com/apt/ stable main 
Chrome
deb http://dl.google.com/linux/chrome/deb/ stable main
pidgin-lwqq:
deb http://ppa.launchpad.net/lainme/pidgin-lwqq/ubuntu precise main
deb-src http://ppa.launchpad.net/lainme/pidgin-lwqq/ubuntu precise main
UbuntuTweak:
deb http://ppa.launchpad.net/tualatrix/ppa/ubuntu precise main
deb-src http://ppa.launchpad.net/tualatrix/ppa/ubuntu precise main
\end{lstlisting}

\subsubsection{系统工具}
Gnome桌面:gnome-shell和经典桌面gnome-panel。gnome-shell可以安装panel-dock插件。通过PPA可以安装cinnamon桌面,是LinuxMint开发的Gnome2风格的桌面。

Deb包管理器gdebi。

系统设置工具:ubuntu-tweak(ppa:tualatrix/ppa),gnome-tweak-tool。

nautilus辅助:nautilus-open-terminal

深度软件中心:deepin-softwarecenter(ppa:noobslab/deepin-sc)。

Adobe Air:官网下载安装即可。在Ubuntu12.04上安装会出错,可以执行:
\begin{verbatim}
sudo ln -s /usr/lib/i386-linux-gnu/\
libgnome-keyring.so.0.2.0 \
/usr/lib/libgnome-keyring.so.0
\end{verbatim}

\subsubsection{网络通信}
Pidgin,chromium-browser, aMSN, sshpass, lcrt, 网盘等。

\subsubsection{音视频播放}
视频播放器:Gnome-Mplayer和VLC。

视频网站客户端:ppstream。

音乐播放器:可选择banshee,audacious,rhythmbox。

歌词搜索程序:osdlyrics(ppa:osd-lyrics/ppa)为独立程序。执行osdlyrics命令,以将其打开。Ctrl+Shift+L快捷键使其解除锁定。

\subsubsection{办公软件}

VIM:vim-scripts,vim-latexsuite和vim-doc包含了Vim的一些有用插件。vim-addon-manager为Vim插件管理程序,安装该工具之后可以用vim-addons命令激活各种插件。

Latex:在texlive和texlive-xetex, texlive-latex-extra。

办公套件:永中2012青春版。

PDF文件阅读与编辑:okular, xournal, pdfmod。 

文献管理:mendeleydesktop。

制图工具:QtiPlot(仿制Origin), dia(模仿Visio)。论文制图也可使用办公套件中的电子表格程序,或GNUplot。 

词典: Goldendict

编程相关: ctags, cscope, doxygen, manpages-dev(Ubuntu可能已经预装), qtcreator, python-qt4, python-qt4-doc

\subsubsection{虚拟机}
KVM虚拟系统:qemu-kvm libvirt-bin virt-manager bridge-utils。
KVM制作的虚拟机保存路径一般是/var/lib/libvirt/images。
打开virt-manager可能需要root权限。


\subsection{系统设置工具一览}

系统设置:gnome-control-center

高级设置:gnome-tweak-tool

系统配置机制:gnome3用Gsettings机制取代了gconf机制。用gsettings取代了gconftool-2工具。dconf-editor是gsettings的众多后端工具之一。

\subsection{设置unity白名单}
如果是Unity桌面,设置unity白名单:
\begin{shellcmd}
gsettings set com.canonical.Unity.Panel systray-whitelist "['all']"
\end{shellcmd}

\subsection{字体配置}
文泉驿可能被默认安装了。安装少量Windows字体即可。如simsun,simfang,simhei,simkai
使用\verb+fc-list :lang=zh-cn+查看当前系统字体。

\subsection{主菜单配置}
使用alacarte创建的应用程序的快捷入口,创建在~/.local/share/applications目录下,并默认以 alacarte-made[-X].desktop 的格式命名,其中X是数字,用户可以随后重命名这个文件,菜单上显示的内容不会改变。而在alacarte工具中删除的快捷入口,也不会真的删除对应的 desktop 文件,而只是将对应文件中的Hide字段的值改为true。
以root权限安装的程序,其快捷入口大多创建在/usr/share/applications目录下,而以用户权限安装的程序,则只能将快捷入口创建在~/.local/share/applications目录下。

\subsection{默认程序配置}
有几个文件用于存储指定类型文件的关联程序,分别是 /etc/gnome/defaults.list, /usr/share/applications/defaults.list, /usr/share/applications/mimeinfo.cache, ~/.local/share/applications/mimeapps.list, ~/.local/share/applications/mimeinfo.cache。前面三个文件保存全局设置,后面两个保存用户设置。如果要修改某个类型文件的关联程序,可以通过直接修改这几个文件的方式实现。

\subsection{开机自启动程序}
gnome-session-properties工具,可以在主菜单->启动应用程序中找到。

\subsection{桌面启动器配置}
gnome-desktop-item-edit,可以创建或编辑.desktop类型的文件。创建名为my.desktop的启动器,执行:
\begin{verbatim}
gnome-desktop-item-edit --create-new my.desktop
\end{verbatim}










\section{Fedora16安装之后}

\subsection{配置网络}
参\ref{netconfig}.

\subsection{配置时间和地区}
应该设置地区为上海,这样在安装很多应用程序到时候会下载中文语言

\subsection{配置Yum源}
参\ref{yum}.

\subsection{配置字体}
安装文泉驿字体
\begin{shellcmd}
sudo yum -y install wqy-bitmap-fonts wqy-zenhei-fonts wqy-unibit-fonts
\end{shellcmd}
安装Windows字体
\begin{itemize}
\item 将windows字体拷贝到/usr/share/fonts/某目录/下
\item chmod 755 *
\item mkfontscale;mkfontdir;fc-cache -fv
\end{itemize}

\subsection{安装系统工具}
\subsubsection{安装gnome-tweak-tool}
\begin{shellcmd}
sudo yum -y install gnome-tweak-tool
\end{shellcmd}
\subsubsection{安装gnome shell extension}
访问https://extensions.gnome.org/
\subsubsection{其他工具}
\begin{itemize}
\item xkill.需安装xkill或xorg-x11-apps
\item nautilus-open-terminal.安装完成后ctr+alt+backspace重启X
\item faenza-icon-theme.
sudo yum -y install faenza-icon-theme
\end{itemize}

\subsection{安装常用软件}
\subsubsection{安装google-chrome}
google并没有直接提供yum源,而是以sh文件的方式提供。那么就下载这个文件
\begin{shellcmd}
wget https://dl-ssl.google.com/linux/google-repo-setup.sh

sudo sh google-repo-setup.sh 

sudo yum -y install google-chrome-stable

rm google-repo-setup.sh 
\end{shellcmd}
\subsubsection{安装latex和xelatex}
\begin{shellcmd}
sudo yum -y install texlive texlive-xetex

fmtutil --enablefmt xelatex

sudo yum -y install texmaker
\end{shellcmd}
\subsubsection{安装多媒体解码器}
fedora默认没有安装视频解码器,所以不能听歌看视频,打开歌曲时会提示缺少MPEG-1 Layer3。
首先确保系统已经安装rpmfusion源,在终端中输入命令:
\begin{shellcmd}
sudo yum -y install ffmpeg ffmpeg-libs gstreamer-ffmpeg \
libmatrosca xvidcore libdvdread libdvdnav lsdvd
sudo yum -y install gstreamer-plugins-good \ 
gstreamer-plugins-bad gstreamer-plugins-ugly
\end{shellcmd}

\subsubsection{安装Office}
安装LibreOffice:
\begin{shellcmd}
sudo yum -y groupinstall "Office/Productivity"
sudo yum -y install libreoffice-langpack-zh-Hans
\end{shellcmd}
永中Office可以从官网下载

\subsubsection{Vim及其插件}
安装gvim
\begin{shellcmd}
sudo yum -y install gvim
\end{shellcmd}
安装vim-latex
\begin{shellcmd}
sudo yum -y install vim-latex
\end{shellcmd}
所谓vim-addon-manager有两个意思,一个指debian下的软件,一个是vim插件,这里指后者.从官网下载该插件后,解压,然后配置.vimrc文件指定vim-addon-manager路径和想安装的插件的名称.
例如:
\begin{shellcmd}
set runtimepath+=/PATH/TO/VIM-ADDON-MANAGER
call vam#ActivateAddons([``vim-haxe'',``snipmate''])
call vam#ActivateAddons([``OmniCppComplete''])
call vam#ActivateAddons([``The_NERD_Commenter''])
\end{shellcmd}
下一次打开vim的时候会自动提示安装相应插件.如果插件名称有微小的错误(typo),可能会得到正确提示.

\subsubsection{KVM虚拟机}
安装KVM虚拟机
\begin{shellcmd}
sudo yum -y install kvm qemu libvirt virt-manager
\end{shellcmd}
可以利用virt-manager安装Windows XP系统,然后安装360安全卫士,搜狗拼音,360浏览器,360压缩等.注意yunio不支持ie系统的浏览器,所以可以再安装一个chrome浏览器.

\subsubsection{其他}
mendeley,yozo office,fcitx输入法,unrar解压软件,qt-creator(在ubuntu下叫做qtcreator),AdobeReader\_chs,antiword

\subsection{卸载不需要的软件}
ibus等

\subsection{修改默认应用程序}
在Fedora下,有两个配置文件:
/usr/share/applications/defauts.list \\
/usr/local/share/applications/defauts.list
其关系不明

\subsection{配置硬盘自动挂载}
参\ref{diskpartition}.

\subsection{关闭SELinux}
修改/etc/selinux/config




