\section{网络流量测量}
\subsection{系统网速测量}
最原始的办法,是连续两次使用date;ifconfig命令,计算一定时间间隔内的数据量.
也可以通过查看/proc/net/dev获取数据量.

在Ubuntu下,可以使用vnstat工具
\begin{shellcmd}
#-ru 0 使其以byte为单位,1使其以bit为单位.
vnstat -l -ru 0  
vnstat -tr
\end{shellcmd}
在Gnome3下,可以使用一个叫做netspeed的gnome shell插件.

在Fedora下,可以使用iftop
\begin{shellcmd}
#-B表示以byte而非bit为单位,-P显示端口号
sudo iftop -B -P 
\end{shellcmd}
也可以使用sar工具.在Fedora下,sar工具位于sysstat软件包中.
\begin{shellcmd}
#最后的数字表示刷新时间间隔,单位为秒
sar -n DEV 3 
\end{shellcmd}
此外,还有ntop,iptraf等工具
\subsection{进程端口号查询}
如果需要知道某应用程序的网络速率,需知道其端口号.可使用netstat命令.
例如,欲知道程序yunio的端口号,使用以下命令:
\begin{shellcmd}
#a表示all,p表示program
netstat -ap |grep yunio 
\end{shellcmd}




