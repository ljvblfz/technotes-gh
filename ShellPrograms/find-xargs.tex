\section{find和xargs}

\subsection{xargs命令}
执行一条命令,其参数从标准输入获取。

其用途之一是用于将参数分成多行输入,类似于行末添加了反斜杠。例如输入xargs file,分多行输入file命令的参数,用CtrlD终止输入。

用途之二是构造管道,第一个命令的输入并不连接到第二个命令的输入,而是其命令行参数。如

\begin{verbatim}
ls | xargs file
find /tmp -name core -type f -print0 | xargs -0 /bin/rm -f
\end{verbatim}

\subsection{find命令}
\begin{verbatim}
find [-H] [-L] [-P] [-D debugopts] [-Olevel] [path...] [expression]
\end{verbatim}
H,L,P选项控制是否进入符号链接。D,O选项也不太常用。
expression包含选项、测试、行动三部分,因此find命令的使用方式通常为如下形式:

\verb+find 起始目录 寻找条件 操作+

寻找条件可以用and,or,not连接起来,分别写作:

\verb+ -a -o 和 !+

如

\verb+ find ! -name 'tmp'+

\verb+-prune选项相当与-maxdetph 0+,表示只对path参数包含的路径进行操作,不进行递归。

需要说明的是:当使用很多的逻辑选项时,可以用括号把这些选项括起来。为了避免Shell本身对括号引起误解,在话号前需要加转义字符来去除括号的意义。例:

\verb+find \(–name ’tmp’ -a type c -user ’inin’ \)+


\subsection{只显示文件夹}
只显示当前目录下的文件夹:

\verb+ls -l | grep ^d+
\verb+find * -type d -prune+
\verb+find . -maxdepth 1 -mindepth 1 -type d+

只显示当前目录下的非文件夹:

\verb+ls -l | grep -v ^d+

