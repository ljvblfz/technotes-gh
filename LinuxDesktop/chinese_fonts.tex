\section{中文字体的安装使用}
\subsection{查看已安装的中文字体}
\begin{shellcmd}
fc-list :lang=zh|sort
\end{shellcmd}

\subsection{安装文泉驿字体}
在Ubuntu软件仓库中,文泉驿点阵宋体和矢量正黑体分别包含于xfonts-wqy,ttf-wqy-zenhei软件包,不过很可能已经预装。
在Fedora中,点阵宋体和矢量正黑体分别位于wqy-bitmap-fonts,wqy-zenhei-fonts

\subsection{安装文鼎字体}
在Ubuntu软件仓库中,we文鼎楷体和明体分别位于ttf-arphic-ukai,ttf-arphic-uming软件包,不过很可能已经预装。
新发布的有文鼎PL报宋二GBK

\subsection{安装Windows字体}

将simsun等字体复制到目录/usr/share/fonts/TTF

在字体所在目录执行命令:
\begin{shellcmd}
mkfontdir  生成font.dir
ttmkfdir   为ttc字体生成font.dir
mkfontscale 为simsun.ttc生成font.scale
\end{shellcmd}
生成encoding文件:
\begin{shellcmd}
sudo cp /usr/share/fonts/encodings/encodings.dir ./
\end{shellcmd}
以上命令为X.org 字体系统服务。以下命令为xft字体系统服务:
\begin{shellcmd}
sudo fc-cache -fv   刷新xft字体库
\end{shellcmd}
重启X即可。

\subsection{字体大小}
1em=16px

\subsection{字体与文件名对照}
\begin{verbatim}
方正舒体              FZSTK.TTF

方正姚体              FZYTK.TTF

微软雅黑              MSYH.TTF

微软雅黑 Bold         MSYHBD.TTF

仿宋体               simfang.ttf

黑体                 simhei.ttf

楷体                 simkai.ttf

隶书                 SIMLI.TTF

宋体 & 新宋体        simsun.ttc

幼园                 SIMYOU.TTF

华文彩云            STCAIYUN.TTF

华文仿宋            STFANGSO.TTF

华文琥珀            STHUPO.TTF

华文楷体            STKAITI.TTF

华文隶书            STLITI.TTF

华文宋体            STSONG.TTF

华文细黑            STXIHEI.TTF

华文行楷            STXINGKA.TTF

华文新魏            STXINWEI.TTF

华文中宋            STZHONGS.TTF

宋体-方正超大字符集 SURSONG.TTF
\end{verbatim}
