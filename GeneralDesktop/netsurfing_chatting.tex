\section{网络通信工具}

\subsection{邮件提醒}

\begin{description}
    \item[Pidgin插件]安装pidgin-guifications插件,并设置Pidgin接收MSN邮件。
    \item[checkgmail]checkgmail可以检查Gmail邮箱。将所关注的邮箱设置成自动转发到gmail,设置gmail的动作为打开邮件客户端如evolution。
    \item[其他方案]mail-notification;gnubiff。
\end{description}

\subsection{即时通信}

常见的聊天工具如QQ,MSN和飞信都有网页版,可以用浏览器制作Web App。除此之外,还有一些第三方客户端。

飞信: Web飞信,3G飞信,openfetion,cliofetion,pidgin-openfetion。

QQ:Web Q+, libqq-pidgin(ppa:lainme/libqq,基于QQ国际版,已经失效),Wine-qq。基于Web QQ协议,分别有人使用Python,GTK,Qt,Java、Pidgin开发了QQ客户端,包括:python-webqq(ppa:linux-deepin-team/linux-deepin), gtkqq(ppa:bill-zt/gtkqq),QtQQ, iQQ,pidgin-lwqq(ppa:lainme/pidgin-lwqq)。

MSN:Web Windows Live,pidgin,aMSN,kmess, kopete, emesene,empathy等。

\subsection{聊天消息提醒}
Web Q+:桌面通知功能,可以实现跨工作区提醒,直接显示聊天内容于桌面。标题栏提醒功能,不能跨工作区提醒,不暴露消息内容。消息走马灯功能,尚不知如何使用。

Pidgin:libnotify插件,会弹出消息,不够隐私;guification(软件包pidgin-guifications)插件,和notify(中文:消息提示)插件,可以做到跨工作区提醒,不直接显示消息。后来发现notify插件没什么作用(Linux Deepin 12.06)。好友千里眼,添加对单独好友的新消息监视,一次性使用,也可以勾选“重复”,每条消息都会提示一次。最好的办法是右键面板图标,勾选“有新消息时闪烁”。

其他聊天工具具体情况有待补充。

\subsection{Chromium浏览器}
软件包叫做chromium-browser,默认不自带flash插件。需要安装flashplugin-installer。然后执行:
\begin{shellcmd}
    sudo cp /usr/lib/flashplugin-installer/\
    libflashplayer.so \ 
    /usr/lib/chromium-browser/
\end{shellcmd}

\subsection{远程登录}
ssh远程登录:sshpass,lcrt。lcrt是为数不多的能保存密码的ssh登录工具。

