\section{复杂的C语言声明语法}

\subsection{声明的语法构成}
以下内容为声明语句的语法构成,选自《C专家编程》
\begin{lstlisting}
声明语句:
至少一个类型说明符(type-specifier) 存储类型(storage-class) 类型限定符(type-qualifier) 一个或多个声明器(declarator) 一个分号

声明器:
指针 直接声明器 初始化内容

\end{lstlisting}

类型说明符:
\begin{enumerate}
    \item char, short, int, long, double, float, signed, unsigned, void 
    \item 结构说明符(struct-specifier)
    \item 枚举说明符(enum-specifier)
    \item 联合说明符(union-specifier)
\end{enumerate}

存储类型:extern, static, register, auto, typedef

类型限定符:const, volatile

指针(星号,后面可有volatile或/和const符号,顺序不限):
\begin{lstlisting}
    * [volatile][const]
\end{lstlisting}

直接声明器:
\begin{enumerate}
    \item 标识符
    \item 标识符[下标]
    \item 标识符(参数)
    \item (声明)
\end{enumerate}

初始化内容:\verb|=初始值|

\subsection{声明语句的解析}
声明语句分解的优先级规则为:
\begin{itemize}
    \item 从最左标识符开始读取
    \item 优先级从高到低依次是: 
    \item const和 
\end{itemize}

















