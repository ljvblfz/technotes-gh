\section{SSH免密登录}
默认情况下ssh命令的每次执行都需要输入密码以进行验证。scp命令基于ssh,也需要密码。有两种方法能够避免在使用ssh和scp时输入密码。

\subsection{sshpass}
sshpass工具可用于保存ssh密码。在ssh和scp命令前添加\verb+sshpass -p <password>+即可。

\subsection{服务器端保存客户机公钥}
首先,客户端需要生成自身的公钥和私钥,使用ssh-keygen命令,公私钥文件在生成后保存到客户机~/.ssh目录下。
\begin{verbatim}
ssh-keygen -t rsa
\end{verbatim}

其次,将公钥文件追加到服务器主机\verb+~/.ssh/authorized_keys+文件结尾。该文件模式需为600,.ssh目录需为700。
\begin{verbatim}
client> scp <pubkey_file> root@server_host:~
server> cat <pubkey_file> >> /root/.ssh/authorized_keys
server> chmod 700 /root
server> chmod 700 /root/.ssh
server> chmod 600 /root/.ssh/authorized_keys
\end{verbatim}
经实验,root家目录访问权限也可以为750, .ssh权限也可以为755, authorized\_keys权限也可为644。
Ubuntu上传客户端公钥的方法也可以是:
\begin{verbatim}
ssh-copy-id -i ~/.ssh/id_rsa.pub root@serverhost
\end{verbatim}


然后,按正常方式进行ssh登录。此时输入密码。以后则无需输入密码了。

\section{sshd服务启动}
在Ubuntu上,临时启动、停止和查看一个服务可以执行:
\begin{verbatim}
service ssh start/stop/status
/etc/init.d/ssh start/stop/status
\end{verbatim}

如果希望开机时不自动启动sshd,只有需要用时才开启sshd,目前找到的办法是:破坏\verb+/etc/init/ssh.conf+文件,如将其重命名,或将exec语言破坏。
这样,sshd将脱离service命令的管理。需要临时开启时,执行
\begin{verbatim}
/usr/sbin/sshd 或者
/usr/sbin/sshd -D
\end{verbatim}

如果遇到一个错误:
\begin{verbatim}
Missing privilege separation directory: /var/run/sshd
\end{verbatim}

只需执行:
\begin{verbatim}
mkdir -p /var/run/sshd
\end{verbatim}



