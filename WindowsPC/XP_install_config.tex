\section{Windows XP的安装配置}

\subsection{安装版XP}
使用安装版CD安装Windows XP,大致需要四五十分钟。其中,二十分钟后会提示输入地域、账户、序列号等信息。余下不需要刻意关注。

在实验室电脑安装KVM版虚拟XP时,没有额外安装任何驱动程序。显示外观在全屏时比较模糊。设置以1280*800分辨率使得全屏时恰好覆盖整个屏幕。

如果是在PC上安装真实XP,第一件是安装网卡驱动和其他驱动,推荐使用自带网卡驱动的驱动精灵。此外,我至少需安装的软件包括360浏览器和安全卫士,输入法, Chrome浏览器。然后运行激活工具。

\subsection{克隆版XP}
要点有二:MBR和设置活动分区。

克隆版在复制C盘数据的过程中不更新MBR。因此,对于MBR不需要改动的情形,克隆版可以顺利安装。否则,就需要改动MBR。可以使用WinPE安装XP,PE下带有分区管理工具,在分区后可以先设置C盘为活动分区。然后使用本工具或另一款分区修复工具,更新MBR。

\subsection{VirtualBox安装XP虚拟机}
使用虚拟机安装克隆版XP经常发生蓝屏错误,提示processor.sys出现了问题。可以通过PE系统修改\verb+C:/windows/system32/drivers/processor.sys+的名称,使系统找不到这个文件。

对于某些机构IP和MAC绑定上网的情形,VirtualBox联网方式设置成NAT方式。在虚拟机里无需更改MAC地址,只需要设置正确的DNS服务器地址就好。

\subsection{IP与MAC地址查看}
在命令提示符下输入
\begin{verbatim}
ipconfig /all
\end{verbatim}

\subsection{MAC地址设置}
如果需要更改MAC地址,而且不借助于第三方工具,方法如下:
通过设备管理器找到相关的网卡,从``属性''对话框‘Network Adress’标签找到一个叫’值‘的编辑框,在此处设置MAC地址。

