\section{用户与组}

\subsection{用户增加与删除}
useradd, adduser, userdel, deluser, usermod, delgroup

使用ueradd时,如果后面不添加任何参数选项,例如:\verb+#sudo useradd test+创建出来的用户将是默认“三无”用户:一无Home Directory,二无密码,三无系统Shell。
使用adduser时,创建用户的过程更像是一种人机对话,系统会提示你输入各种信息,然后会根据这些信息帮你创建新用户。
\begin{verbatim}
useradd -m -s /bin/bash user 
-m表示创建家目录,默认不创建。
注意-p选项以密文方式设置密码,似乎没什么用。
userdel -r user
删除用户及其家目录
\end{verbatim}

\subsection{设置用户为sudoer}
执行visudo命令,在打开的文件中添加一行
\begin{verbatim}
userA ALL=(ALL)  ALL
userB ALL=(ALL)  NOPASSWD: ALL
\end{verbatim}
visudo相当于修改/etc/sudoers文件。userB执行sudo的时候不需要输入密码。

可以在bashrc或profile中添加如下内容,修改PATH
\begin{verbatim}
PATH=/usr/sbin:/usr/local/sbin:/sbin:$PATH
export PATH
\end{verbatim}

\subsection{查看用户所在组}
方法一:id命令
方法二:/etc/group,格式:
\begin{verbatim}
group_name:password('x'):GID:user_list(separated by commas)
\end{verbatim}

/etc/passwd包含了家目录,shell等用户信息

\subsection{用户的shell}
查看系统安装的shell
\begin{verbatim}
cat /etc/shells
\end{verbatim}

查看当前shell
\begin{verbatim}
echo $SHELL
\end{verbatim}

更改某用户的shell
\begin{verbatim}
usermod -s /bin/zsh someuser
\end{verbatim}


\subsection{修改用户名和主机名}
用户名在/etc/shadow中修改,主机名需要在/etc目录下hostname和hosts两个文件下修改

\subsection{用户密码}
只能使用passwd命令。在创建用户后创建密码。
