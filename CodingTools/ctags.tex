\section{Ctags关键用法}

创建元数据:
\begin{verbatim}
  ctags -R
\end{verbatim}
  对于C++,结合omnicppcomplete插件,有:
\begin{verbatim}
   ctags -R --c++-kinds=+p --fields=+iaS --extra=+q
\end{verbatim}

为vim配置tags搜索路径:
在.vimrc中添加设置。如使用绝对路径:
\begin{verbatim}
 set tags=/home/xxx/myproject/tags
\end{verbatim}

如果设置成自动搜索上级目录的tags:
\begin{verbatim}
set tags=./tags;
\end{verbatim}
注意第一行的分号表示递归向上搜索,点斜杠表示当前文件所在目录而非当前目录。

\begin{verbatim}
:set tags=./tags,./../tags,./*/tags
\end{verbatim}
使用当前目录下的tags文件, 上一级目录下
的tags文件, 以及当前目录下所有层级的子目录下的tags文件.


\begin{verbatim}
:set tags=~/proj/**/tags
\end{verbatim}
一种深度搜索目录的形式

查找:
\begin{verbatim}
vim -t tag名
:ta tag名
:tselect tag名 同名tag选择

\end{verbatim}

跳转:
\begin{verbatim}
Ctrl+]  跳转至函数定义
ctrl+O 返回
ctrl+I 前进
Ctrl+T 返回(与ctrl+])对应
:tp  同名tag,跳转到前一个
:tn  同名tag,跳转到下一个
:tfirst, :tlast 
\end{verbatim}


裂屏显示:
\begin{verbatim}
Ctrl-W ] 裂屏跳转至函数定义
[vertical] stag name 裂屏查找并跳转
\end{verbatim}


