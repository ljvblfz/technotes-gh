\section{中文输入与显示问题}

\subsection{修改locale为GB18030}

\begin{verbatim}

/usr/share/i18n目录下为系统已经下载的字符集和locale定义文件。

在 /var/lib/locales/supported.d/local 添加两行:
zh_CN.GBK GBK和zh_CN.GB18030 GB18030 

执行sudo locale-gen 生成相应locale
在/etc/default/locale文件配置LANG,LC_*, LC_ALL等参数。其优先级递增。
也有说修改/etc/environment。但修改的结果都是只改变了root用户的locale。

普通用户的locale可以通过export LANG修改环境变量的方法进行修改。
但将locale编码改为GB码后会出现各种乱码,网上建议还是保留为UTF-8。

\end{verbatim}




\subsection{Evolution客户端没有附件}
Evolution发的邮件附件Foxmail不认,在 Evolution 的 [编辑] -> [首选项] -> [编写器首选项] 中
将“使用 Outlook / Gmail 的方式编码文件名” 前的复选框打上勾就可以了

\subsection{查看文件编码}
\begin{shellcmd}
file *
enca *
本地化编码 enconv *
\end{shellcmd}

\subsection{转换文件名由GBK为UTF8}
\begin{shellcmd}
convmv -r -f cp936 -t utf8 --notest --nosmart *
\end{shellcmd}

批量转换src目录下的所有文件内容由GBK到UTF8
\begin{shellcmd}
find src -type d -exec mkdir -p utf8/{} \;
find src -type f -exec iconv -f GBK -t UTF-8 {} -o utf8/{} \;
mv utf8/* src
rm -fr utf8
\end{shellcmd}

\subsection{文字大小不一致}
\begin{shellcmd}
sudo apt-get remove ttf-kochi-gothic ttf-kochi-mincho ttf-unfonts ttf-unfonts-core
\end{shellcmd}

\subsection{gedit中文乱码}

 命令行指定编码,例如gedit --encoding=gbk 霜冷长河.txt 。也可以配置Gsettings参数,如:
   
   \begin{shellcmd}
    gsettings set org.gnome.gedit.preferences.encodings auto-detected 
	"['UTF-8','GB18030','GB2312','GBK','BIG5','CURRENT','UTF-16']"
    gsettings set org.gnome.gedit.preferences.encodings shown-in-menu 
	"['GB18030','UTF-8','GB2312','GBK','BIG5','CURRENT','UTF-16']"
   \end{shellcmd}  

	相关图形工具叫做dconf-editor,在dconf-tools包中

  
  


\subsection{vim字符编码}
涉及3个参数:enc,fenc,fencs.
fenc为当前文件的属性,可以对当前文件set fenc,在保存,从而设置文件的编码方式。在vimrc中设置fenc,就是设置以后写入的所有文件的编码方式。
fencs为读入文件时的编码猜测列表。enc为vim工作时的编码方式,但最终写入文件时采取的编码仍取决于fenc。
 set fencs=utf-8,gbk
 set fenc=gb18030

\subsection{PDF 文件乱码}
PDF中文乱码有两种
1.汉字显示为方块:如各种从中国知网下载的论文。网上的方法多针对此种乱码

2.汉字显示为各种欧洲字母:如我电脑中的"西游记.pdf", 基于popplar的阅读器都不能搞定,foxit4linux和Adobe Reader也都失败。永中office的pdf阅读器可以搞定。foxit for linux在2009年烂尾,目前仍为1.1版。
\begin{shellcmd}
sudo apt-get install xpdf-chinese-simplified xpdf-chinese-traditional poppler-data
\end{shellcmd}

\begin{shellcmd}
sudo gedit /etc/fonts/conf.d/49-sansserif.conf 
将倒数第四行 <string>sans-serif</string>
改为 <string>sans</string>
保存即可,重启firefox
\end{shellcmd}

\subsection{unzip中文文件名乱码}
\begin{shellcmd}
sudo apt-get install p7zip-full
export LANG=zh_CN.GBK  #临时在控制台修改环境为zh_CN.GBK,然后解压缩即可
7za e docs.zip
\end{shellcmd}

\subsection{输入法问题}
ibus跟随
\begin{shellcmd}
安装ibus-gtk即可,最好另外安装:ibus-qt4
\end{shellcmd}
输入法切换
\begin{shellcmd}
im-switch -c
\end{shellcmd}









