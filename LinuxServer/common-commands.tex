\section{常用命令详解}

\subsection{ls命令用法}
解释几个比较实用的选项。当不使用任何选项时,输出按照名字排序。S,t选项能更改排序的依据。r指示反序。

\subsubsection{l选项}
会产生类似如下输出。
\begin{verbatim}
srw-rw-rw-   1 root root             0 11月  3 15:45 log=
crw-------   1 root root       10, 237 11月  3 15:45 loop-control
-rwsr-xr-x   1 root root         45420  7月 27 01:07 /usr/bin/passwd*
\end{verbatim}
其中第1个字符表示文件类型。s表示socket,p表示FIFO(命名管道),b表示块设备,c表示字符设备,l表示符号链接,-表示普通文件,?表示未知类型。

在文件权限中,s表示设置用户(组)id位为1,且文件可执行。t表示粘住位为1,且文件可执行。相应的S和T表示文件不可执行。

第2个位段表示硬链接数。第3、4位分别表示用户和组,-n选项让二者用数字而非名字表示。

第5位单位为字节。对于设备文件(b和c),显示为主从设备号。

第6位表示时间戳。文件的三个时间属性为修改时间(mtime),访问时间(atime)和i节点状态改变时间(ctime)。-l选项的默认时间戳为mtime。使用u选项使得时间戳使用atime,使用-c选项表示时间戳为ctime。

最后一位为文件名。

\subsubsection{R选项}
递归输出。
相当于\verb|`--recursive'|
\subsubsection{F选项}
文件名后面加上一个符号,表示文件类型。如/表示目录,@表示符号链接,|表示FIFO,=表示socket。普通文件后面什么也没有。
相当于\verb|`--classify'|
\subsubsection{d选项}
-d选项表示显示的为目录本身的信息,而非其子目录或包含文件的信息。
相当于\verb|`--directory'|
\subsubsection{t选项}
t选项表示按照时间戳排序。
相当于\verb|`--sort=time'|
\subsubsection{S选项}
S选项表示按照size排序。
相当于\verb|`--sort=size'|
\subsubsection{r选项}
将排序结果反转。
相当于\verb|`--reverse'|
\subsubsection{c选项}
使用ctime表示时间。如果同-t选项一同使用,则按照ctime排序。
相当于:
\begin{verbatim}
`--time=ctime'
`--time=status'
\end{verbatim}

\subsubsection{u选项}
使用atime表示时间。如果同-t选项一同使用,则按照atime排序。
相当于:
\begin{verbatim}
`--time=atime'
`--time=access'
`--time=use'
\end{verbatim}

\subsubsection{i选项}
显示inode号。

\subsection{grep}

略


\subsection{xargs命令}
执行一条命令,其参数从标准输入获取。

其用途之一是用于将参数分成多行输入,类似于行末添加了反斜杠。例如输入xargs file,分多行输入file命令的参数,用CtrlD终止输入。

用途之二是构造管道,第一个命令的输入并不连接到第二个命令的输入,而是其命令行参数。如

\begin{verbatim}
ls | xargs file
find /tmp -name core -type f -print0 | xargs -0 /bin/rm -f
\end{verbatim}

\subsection{find命令}
\begin{verbatim}
find [-H] [-L] [-P] [-D debugopts] [-Olevel] [path...] [expression]
\end{verbatim}
H,L,P选项控制是否进入符号链接。D,O选项也不太常用。
expression包含选项、测试、行动三部分,因此find命令的使用方式通常为如下形式:

\verb+find 起始目录 寻找条件 操作+

寻找条件可以用and,or,not连接起来,分别写作:

\verb+ -a -o 和 !+

如

\verb+ find ! -name 'tmp'+

\verb+-prune选项相当与-maxdetph 0+,表示只对path参数包含的路径进行操作,不进行递归。

需要说明的是:当使用很多的逻辑选项时,可以用括号把这些选项括起来。为了避免Shell本身对括号引起误解,在话号前需要加转义字符来去除括号的意义。例:

\verb+find \(–name ’tmp’ -a type c -user ’inin’ \)+


只显示当前目录下的文件夹:

\verb+ls -l | grep ^d+
\verb+find * -type d -prune+
\verb+find . -maxdepth 1 -mindepth 1 -type d+

只显示当前目录下的非文件夹:

\verb+ls -l | grep -v ^d+

\subsection{sort用法笔记}
比较重要的选项包括:

\begin{itemize}
    \item o 指定输出文件
    \item r 逆序
    \item k=pos 指定关键词在第几列,默认为第一列
    \item n 按照数值大小排序;默认为按字典序
    \item u 删除重复行
    \item t 指定field分隔符;默认为空格
    \item f 忽略大小写 
\end{itemize}


举例:
\begin{verbatim}
sort -n -t ‘ ‘ -k 3r -k 2 facebook.txt
sort -t ‘ ‘ -k 1 facebook.txt
\end{verbatim}

sort命令选项众多,十分强大。
\subsection{uniq命令用法笔记}

uniq命令用来检查或忽略文件中重复的行。重要选项有:

\begin{itemize}
    \item -u 只显示不重复的行
    \item -d 只显示重复的行
    \item -c 显示重复次数
\end{itemize}



