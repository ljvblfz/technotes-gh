\section{字符编码}

\subsection{ASCII与ISO/IEC 646}
ASCII(American Standard Code for Information Interchange,美国信息交换标准代码)是基于拉丁字母的一套电脑编码系统。它主要用于显示现代英语,而其扩展版本EASCII则可以勉强显示其他西欧语言。它是现今最通用的单字节编码系统(但是有被Unicode追上的迹象),并等同于国际标准ISO/IEC 646。ISO/IEC 646是国际标准化组织(ISO)和国际电工委员会(IEC)于1972年制订的标准。它是一个 7-位元字符的字集,来自数个国家标准,最主要来自美国的 ASCII(美国信息互换标准代码)。

ASCII第一次以规范标准的型态发表是在1967年,最后一次更新则是在1986年,至今为止共定义了128个字符;其中33个字符无法显示(这是以现今操作系统为依归,但在DOS模式下可显示出一些诸如笑脸、扑克牌花式等8-bit符号),且这33个字符多数都已是陈废的控制字符。控制字符的用途主要是用来操控已经处理过的文字。在33个字符之外的是95个可显示的字符,包含用键盘敲下空白键所产生的空白字符也算1个可显示字符(显示为空白)。

\subsection{ISO 8859}
ISO 8859,全称ISO/IEC 8859,是国际标准化组织(ISO)及国际电工委员会(IEC)联合制定的一系列8位字符集的标准,现时定义了15个字符集。这15个字符集是互斥的。如Latin-1和Latin-2而言相同数字代表不同字符。除了使用拉丁字母的语言外,使用西里尔字母的东欧语言、希腊语、泰语、现代阿拉伯语、希伯来语等,都可以使用这个形式来储存及表示。

ISO 8859-1,正式编号为ISO/IEC 8859-1:1998,又称Latin-1或“西欧语言”,是国际标准化组织内ISO/IEC 8859的第一个8位字符集。它以ASCII为基础,在空置的0xA0-0xFF的范围内,加入96个字母及符号,藉以供使用附加符号的拉丁字母语言使用。

ISO 8859-2,正式编号为ISO/IEC 8859-2:1999,又称Latin-2或“中欧语言”,是国际标准化组织内ISO/IEC 8859的其中一个8位字符集。

ISO 8859-15字符集,也称为 Latin-9,或者被匿称为 Latin-0,它将 Latin-1中较少用到的符号删除,换成当初遗漏的法文和芬兰字母;还有,把英镑和日元之间的金钱符号,换成了欧盟货币符号。

\subsection{通用字符集}
通用字符集(Universal Character Set,UCS)是由ISO制定的ISO 10646(或称ISO/IEC 10646)标准所定义的标准字符集。通用字符集是所有包括了其他字符集。它保证了与其他字符集的双向兼容,即,如果你将任何文本字符串翻译到UCS格式,然后再翻译回原编码,你不会丢失任何信息。UCS包含了已知语言的所有字符。除了拉丁语、希腊语、斯拉夫语、希伯来语、阿拉伯语、亚美尼亚语、格鲁吉亚语,还包括中文、日文、韩文这样的方块文字,UCS还包括大量的图形、印刷、数学、科学符号。ISO/IEC 10646定义了一个31位的字符集。ISO/IEC 10646-1标准第一次发表于1993年,现在的公开版本是ISO/IEC 10646-1:2000。ISO/IEC 10646-2在2001年发表。

UCS不仅给每个字符分配一个代码,而且赋予了一个正式的名字。表示一个UCS或Unicode值的十六进制数通常在前面加上“U+”,例如“U+0041”代表字符“A”。

\subsection{Unicode}
位于美国加州的Unicode组织允许任何愿意支付会员费用的公司或是个人加入,其成员包含了主要的电脑软硬件厂商,例如奥多比系统、苹果公司、惠普、IBM、微软、施乐等。20世纪80年代末,组成Unicode组织的商业机构,和国际合作的国际标准化组织(International Organization for Standardization,简称ISO)因为电脑普及和资讯国际化的前提下,分别各自成立了Unicode组织和ISO-10646工作小组。他们不久便发现对方机构的存在,大家为着相同的目的而工作,于是两个组织便共同合作开发适用于各国语言的通用码,而且“相当有默契地”各自发表Unicode和ISO-10646字集。虽然实际上两者的字集编码相同,但实质上两者确实为两个不同的标准。

目前,几乎所有电脑系统都支持基本拉丁字母,并各自支持不同的其他编码方式。Unicode为了和它们相互兼容,其首256字符保留给ISO 8859-1所定义的字符,使既有的西欧语系文字的转换不需特别考量;并且把大量相同的字符重复编到不同的字符码中去,使得旧有纷杂的编码方式得以和Unicode编码间互相直接转换,而不会遗失任何资讯。

目前实际应用的统一码版本对应于UCS-2,使用16位的编码空间。也就是每个字符占用2个字节。上述16位统一码字符构成基本多文种平面(Basic Multilingual Plane,简称BMP),包含了常见的CJK字。最新(但未实际广泛使用)的统一码版本定义了16个辅助平面,两者合起来至少需要占据21位的编码空间,比3字节略少。但事实上辅助平面字符仍然占用4字节编码空间,与UCS-4保持一致。目前辅助平面的工作主要集中在第二和第三平面的中日韩统一表意文字中,因此包括GBK、GB18030、Big5等简体中文、繁体中文、日文、韩文以及越南喃字的各种编码与Unicode的协调性被重点关注。

Unicode的码空间从U+0000到U+10FFFF,共有1,112,064个码位(code point)可用来映射字符. Unicode的码空间可以划分为17个平面(plane),每个平面包含216(65,536)个码位。每个平面的码位可表示为从U+xx0000到U+xxFFFF, 其中xx表示十六进制值从0016 到1016,共计17个平面。第一个平面成为基本多文种平面(Basic Multilingual Plane, BMP),或称第零平面(Plane 0)。其他平面称为辅助平面(Supplementary Planes)。基本多语言平面内,从U+D800到U+DFFF之间的码位区段是永久保留不映射到字符,因此UTF-16利用保留下来的0xD800-0xDFFF区段的码位来对辅助平面的字符的码位进行编码。

Unicode依随着UCS的标准而发展。目前最新的版本为第六版,已收入了超过十万个字符(第十万个字符在2005年获采纳)。Unicode备受认可,并广泛地应用于电脑软件的国际化与本地化过程。有很多新科技,如XML、Java,以及现代的操作系统,都采用Unicode编码。

\subsection{UTF}
Unicode的实现方式不同于编码方式。一个字符的Unicode编码是确定的。但是在实际传输过程中,由于不同系统平台的设计不一定一致,以及出于节省空间的目的,对Unicode编码的实现方式有所不同。Unicode的实现方式称为Unicode转换格式(Unicode Transformation Format,简称为UTF)。

例如,如果一个仅包含基本7位ASCII字符的Unicode文件,如果每个字符都使用2字节的原Unicode编码传输,其第一字节的8位始终为0。这就造成了比较大的浪费。对于这种情况,可以使用UTF-8编码,这是一种变长编码,它将基本7位ASCII字符仍用7位编码表示,占用一个字节(首位补0)。而遇到与其他Unicode字符混合的情况,将按一定算法转换,每个字符使用1-3个字节编码,并利用首位为0或1进行识别。这样对以7位ASCII字符为主的西文文档就大大节省了编码长度。类似的,对未来会出现的需要4个字节的辅助平面字符和其他UCS-4扩充字符,2字节编码的UTF-16也需要通过一定的算法进行转换。再如,如果直接使用与Unicode编码一致(仅限于BMP字符)的UTF-16编码,由于每个字符占用了两个字节,在麦金塔电脑 (Mac)机和个人电脑上,对字节顺序的理解是不一致的。

目前在PC机上的Windows系统和Linux系统对于UTF-16编码默认使用UTF-16 LE。此外Unicode的实现方式还包括UTF-7、Punycode、CESU-8、SCSU、UTF-32、GB18030等,这些实现方式有些仅在一定的国家和地区使用,有些则属于未来的规划方式。目前通用的实现方式是UTF-16小端序(LE)、UTF-16大端序(BE)和UTF-8。

\subsection{UTF-8}
UTF-8(8-bit Unicode Transformation Format)是一种针对Unicode的可变长度字符编码(定长码),也是一种前缀码。它可以用来表示Unicode标准中的任何字符,且其编码中的第一个字节仍与ASCII相容,这使得原来处理ASCII字符的软件无须或只须做少部份修改,即可继续使用。因此,它逐渐成为电子邮件、网页及其他储存或传送文字的应用中,优先采用的编码。互联网工程工作小组(IETF)要求所有互联网协议都必须支持UTF-8编码。互联网邮件联盟(IMC)建议所有电子邮件软件都支持UTF-8编码。

UTF-8使用一至四个字节为每个字符编码:128个US-ASCII字符只需一个字节编码(Unicode范围由U+0000至U+007F)。带有附加符号的拉丁文、希腊文、西里尔字母、亚美尼亚语、希伯来文、阿拉伯文、叙利亚文及它拿字母则需要二个字节编码(Unicode范围由U+0080至U+07FF)。其他基本多文种平面(BMP)中的字符(这包含了大部分常用字)使用三个字节编码。其他极少使用的Unicode 辅助平面的字符使用四字节编码。

每个使用UTF-8储存的字符,除了第一个字节外,其余字节的头两个位元都是以"10"开始,使文字处理器能够较快地找出每个字符的开始位置。在ASCII码的范围,用一个字节表示。大于ASCII码的,就会由上面的第一字节的前几位表示该unicode字符的长度,比如110xxxxxx前三位的二进制表示告诉我们这是个2BYTE的UNICODE字符;1110xxxx是个三位的UNICODE字符,依此类推;第一个字节的开头"1"的数目就是整个串中字节的数目。ASCII字母继续使用1字节储存,重音文字、希腊字母或西里尔字母等使用2字节来储存,而常用的汉字就要使用3字节。辅助平面字符则使用4字节。


\subsection{UTF-16}
UTF-16是Unicode字符集的一种转换方式,即把Unicode的码位转换为16比特长的码元序列,以用于数据存储或传递。UTF-16正式定义于ISO/IEC 10646-1的附录C。

BMP内,从U+D800到U+DFFF之间的码位区段是永久保留不映射到字符,因此UTF-16利用保留下来的0xD800-0xDFFF区段的码位来对辅助平面的字符的码位进行编码。

从U+0000至U+D7FF以及从U+E000至U+FFFF的码位,UTF-16与UCS-2编码这个范围内的码位为单个16比特长的码元,数值等价于对应的码位. BMP中的这些码位是仅有的码位可以在UCS-2被表示.从U+10000到U+10FFFF的码位为辅助平面(Supplementary Planes)中的码位,在UTF-16中被编码为一对16比特长的码元(即32bit,4Bytes),称作代理对(surrogate pair)。

UTF-16的大尾序和小尾序储存形式都在用。一般来说,以Macintosh制作或储存的文字使用大尾序格式,以Microsoft或Linux制作或储存的文字使用小尾序格式。

\subsection{大小端和BOM}

字节序,又称端序,尾序(英语:Endianness)。在计算机科学领域中,字节序是指存放多字节数据的字节(byte)的顺序,典型的情况是整数在内存中的存放方式和网络传输的传输顺序。Endianness有时候也可以用指位序(bit)。一般而言,字节序指示了一个UCS-2字符的哪个字节存储在低地址。如果LSByte在MSByte的前面,即LSB为低地址,则该字节序是小端序;反之则是大端序。

网络传输一般采用大端序,也被称之为网络字节序,或网络序。IP协议中定义大端序为网络字节序。

字节顺序标记(英语:byte-order mark,BOM)是位于码点U+FEFF的统一码字符的名称。当以UTF-16或UTF-32来将UCS/统一码字符所组成的字串编码时,这个字符被用来标示其字节序。它常被用来当做标示文件是以UTF-8、UTF-16或UTF-32编码的记号。

统一码中,值为U+FFFE的码位被保证将不会被指定成一个统一码字符。这意味着0xFF、0xFE将只能被解释成小尾序中的U+FEFF。

字节顺序标记U+FEFF字符在UTF-8中被表示为序列EF BB BF。许多Windows程序(包括Windows记事本)在UTF-8编码的档案的开首加入一段字节串EF BB BF。这是字节顺序记号U+FEFF的UTF-8编码结果。对于没有预期要处理UTF-8的文字编辑器和浏览器会显示成ISO-8859-1字符串""。

\subsection{汉字内码}

在计算机科学及相关领域当中,内码指的是“将资讯编码后,透过某种方式储存在特定记忆装置时,装置内部的编码形式”。在以往的英文系统中,内码为ASCII。 在繁体中文系统中,目前常用的内码为大五码。在简体中文系统中,内码则为国标码。为了软件开发方便,如国际化与在地化,现在许多系统会使用统一码做为内码,常见的Windows、麦金塔、Linux皆如此。许多语言也采用统一码为内码,如Java、Python 3。

内码是指整机汉字系统中使用的二进制字符编码,是沟通输入、输出与系统平台之间的交换码,通过内码可以达到通用和高效率传输文本的目的。比如MS Word中所存储和调用的就是内码而非图形文字。英文ASCII 字符采用一个字节的内码表示,中文字符如国标字符集中,GB2312、GB12345、GB13000皆用双字节内码,GB18030(27,533汉字)双字节内码汉字为20,902个,其余6,631个汉字用四字节内码。

汉字内码主要有:GB码(1980年国家公布的简体汉字编码方案,在大陆、新加坡得到广泛的使用,也称国标码),GBK码(简体版的Win95和Win98都是使用GBK作系统内码),BIG5码(针对繁体汉字的汉字编码,目前在台湾、香港的电脑系统中得到普遍应用),HZ码(在Internet上广泛使用的一种汉字编码),ISO-2022CJK码,Unicode码。

\subsection{GB2312与区位码}
GB 2312 或 GB 2312-80 是中国国家标准简体中文字符集,全称《信息交换用汉字编码字符集·基本集》,又称GB0,由中国国家标准总局发布,1981年5月1日实施。GB2312编码通行于中国大陆;新加坡等地也采用此编码。中国大陆几乎所有的中文系统和国际化的软件都支持GB 2312。GB 2312标准共收录6763个汉字,覆盖中国大陆99.75\%的使用频率同时收录了包括拉丁字母、俄语西里尔字母在内的682个字符。对于人名、古汉语等方面出现的罕用字,GB 2312不能处理,这导致了后来GBK及GB 18030汉字字符集的出现。

GB 2312中对所收汉字进行了“分区”处理,每区含有94个汉字/符号。这种表示方式也称为区位码。01-09区为特殊符号。16-55区为一级汉字,按拼音排序。56-87区为二级汉字,按部首排序。10-15区及88-94区则未有编码。举例来说,“啊”字是GB2312之中的第一个汉字,它的区位码就是1601。国标码是一个四位十六进制数,区位码是一个四位的十进制数,每个国标码或区位码都对应着一个唯一的汉字或符号,但因为十六进制数我们很少用到,所以大家常用的是区位码,它的前两位叫做区码,后两位叫做位码。

EUC-CN是GB 2312最常用的表示方法,兼容ASCII。浏览器编码表上的“GB2312”,通常都是指“EUC-CN”表示法。GB 2312非ASCII字符使用两个字节来表示。“第一位字节”使用0xA1-0xF7,“第二位字节”使用0xA1-0xFE.举例来说,“啊”字的区位码是1601。在EUC-CN之中,它把0xA0+16=0xB0,0xA0+1=0xA1,得出0xB0A1。

HZ is another encoding of GB2312 that is used mostly for Usenet postings.

Compared to UTF-8, GB2312 (whether native or encoded in EUC-CN) is more storage efficient, this because no bits are reserved to indicate three or four byte sequences, and no bit is reserved for detecting tailing bytes.

\subsection{微软GBK}
GBK即汉字内码扩展规范,K为汉语拼音 Kuo Zhan(扩展)中“扩”字的声母。
1993年,Unicode 1.1版本推出,收录中国大陆、台湾、日本及韩国通用字符集的汉字,总共有20,902个。中国大陆订定了等同于Unicode 1.1版本的“GB13000.1-93”“信息技术通用多八位编码字符集(UCS)第一部分:体系结构与基本多文种平面”。

微软利用GB 2312-80未使用的编码空间,收录GB 13000.1-93全部字符制定了GBK编码。根据微软资料,GBK是对GB2312-80的扩展,也就是CP936字码表 (Code Page 936)的扩展(之前CP936和GB 2312-80一模一样),最早实现于Windows 95简体中文版。虽然GBK收录GB 13000.1-93的全部字符,但编码方式并不相同。GBK自身并非国家标准,只是曾由国家技术监督局标准化司、电子工业部科技与质量监督司公布为“技术规范指导性文件”。原始GB13000一直未被业界采用,后续国家标准GB18030技术上兼容GBK而非GB13000。

字符有一字节和双字节编码,00–7F范围内是一位,和ASCII保持一致,此范围内严格上说有96个文字和32个控制符号。之后的双字节中,前一字节是双字节的第一位。总体上说第一字节的范围是81–FE(也就是不含80和FF),第二字节的一部分领域在40–7E,其他领域在80–FE。

GBK向下完全兼容GB2312-80编码。 支持GB2312-80编码不支持的部分中文姓,中文繁体,日文假名,还包括希腊字母以及俄语字母等字母。不过这种编码不支持韩国字,也是其在实际使用中与unicode编码相比欠缺的部分。微软的CP936通常被视为等同GBK,连 IANA 也以“CP936”为“GBK”之别名。事实上比较起来, GBK 定义之字符较 CP936 多出95字。

\subsection{GB18030}
GB 18030,最新版本为GB 18030-2005,其全称为中华人民共和国国家标准GB 18030-2005《信息技术 中文编码字符集》,与GB 2312-1980完全兼容,与GBK基本兼容,支持GB 13000及Unicode的全部统一汉字,共收录汉字70244个。

此标准中,单字节的部分收录了GB/T 11383-1989的0x00到0x7F全部128个字符。双字节部分采用两个八位二进制位串表示一个字符,其首字节码位从0x81至0xFE,尾字节码位分别是0x40至0x7E和0x80至0xFE。四字节部分采用GB/T 11383-1989未采用的0x30至0x39作为对双字节编码的扩充的后缀。这样扩充的四字节编码,其范围为0x81308130到0xFE39FE39。四字节字符的第一个字节的编码为0x81至0xFE;第二个字节的编码范围为0x30至0x39;第三个字节编码范围为0x81至0xFE;第四个字节编码范围为0x30至0x39。

GB18030与Unicode的关系:GB 18030是一种对字符集的多字节编码格式,相当于UTF-8(对Unicode码点(code point)的编码传输格式),而且都是向后兼容ASCII,并且能表示所有的Unicode码点。GB 18030的四字节编码共有1,587,600 (126×10×126×10), 足以覆盖Unicode的1,111,998 (17×65536 − 2048 surrogates − 66 noncharacters)码点。此外,GB18030还向后兼容了GB 2312与GBK编码。与Unicode码点的映射关系(mapping)一部分要查表,其它可以通过算法求出,这与UTF-8相比不够方便。


\subsection{ISO 2022}

全称ISO/IEC 2022,由国际标准化组织(ISO)及国际电工委员会(IEC)联合制定,是一个使用7位编码表示汉语文字、日语文字或朝鲜文字的方法。
ISO 2022等同于欧洲标准组织(ECMA)的ECMA-35、中国国标GB 2312、日本工业规格JIS X 0202(旧称JIS C 6228)及韩国工业规格KS X 1004(旧称KS C 5620)。拉丁字母、希腊字母、西里尔字母、希伯来字母等的语文,由于只使用数十个字母,传统上均使用8位编码的ISO/IEC 8859标准来表示。但由于汉语、日语及朝鲜语字数众多,无法用单一个8位字符来表达,故需要多于一个字节来代表一个字。于是,ISO 2022就设计出来让汉语、日语及朝鲜语可以使用数个7位编码的字符来示。ISO 2022使用“逃逸字串”(Escape sequence)。逃逸字串由1个“ESC”字符(0x1B),再由两至三个字串组成。此标记代表它后面的字符,属于下表字符集的文字。





























\section{数字证书}
1.带有私钥的证书
由Public Key Cryptography Standards \#12,PKCS\#12标准定义,包含了公钥和私钥的二进制格式的证书形式,以pfx作为证书文件后缀名。

2.二进制编码的证书
证书中没有私钥,DER 编码二进制格式的证书文件,以cer作为证书文件后缀名。

3.Base64编码的证书
证书中没有私钥,BASE64 编码格式的证书文件,也是以cer作为证书文件后缀名。
由定义可以看出,只有pfx格式的数字证书是包含有私钥的,cer格式的数字证书里面只有公钥没有私钥。
在pfx证书的导入过程中有一项是“标志此密钥是可导出的。这将您在稍候备份或传输密钥”。一般是不选中的,如果选中,别人就有机会备份你的密钥了。如果是不选中,其实密钥也导入了,只是不能再次被导出。这就保证了密钥的安全。
如果导入过程中没有选中这一项,做证书备份时“导出私钥”这一项是灰色的,不能选。只能导出cer格式的公钥。如果导入时选中该项,则在导出时“导出私钥”这一项就是可选的。
如果要导出私钥(pfx),是需要输入密码的,这个密码就是对私钥再次加密,这样就保证了私钥的安全,别人即使拿到了你的证书备份(pfx),不知道加密私钥的密码,也是无法导入证书的。相反,如果只是导入导出cer格式的证书,是不会提示你输入密码的。因为公钥一般来说是对外公开的,不用加密
\section{大小端}

如果LSByte在MSByte的前面,即LSB为低地址,则该字节序是小端序;反之则是大端序。A big-endian machine stores the most significant byte first, and a little-endian machine stores the least significant byte first. 

网络传输一般采用大端序,也被称之为网络字节序,或网络序。IP协议中定义大端序为网络字节序。

Linux和Windows使用小尾端。Macintosh使用大端序。

\section{字体基础知识}

\subsection{字库标准}
PostScript(PS)是主要用于电子产业和桌面出版领域的一种页面描述语言和编程语言。
桌面出版系统使用的字库有两种标准: PostScript字库和truetype字库。这两种字体标准都是采用曲线方式描述字体轮廓,因此都可以输出很高质量的字形。TrueType最初是由苹果公司引入,用以回应Adobe专有的PostScript格式。早期的TrueType字体多为PostScript字体的拙劣翻版,但时至今日,各大字体公司都以TrueType格式来发布他们的产品(其中有些是有PostScript原型的,有些则原生就是TrueType,再无其他格式)。早在80年代末,苹果公司为了对抗Adobe公司的Type 1 PostScript字体,设计开发了TrueType,之后微软加入了开发,后来视窗系统的字体格式基本上都统一成TrueType,而在苹果的麦金塔系统中却成了PostScript和TrueType对立的局面。TrueType后来也被Linux等系统使用,成为标准字体。TrueType的主要强项在于它能给开发者提供关于字体显示、不同字体大小的像素级显示等的高级控制。在新开发的OpenType类型字体中,可以选择PostScript还是TrueType作为记述方式。

TrueType字体,中文名称全真字体。TrueType是由Apple公司和Microsoft公司联合提出的一种新型数学字形描述技术。它用数学函数描述字体轮廓外形,含有字形构造、颜色填充、数字描述函数、流程条件控制、栅格处理控制、附加提示控制等指令。其采用几何学中二次B样条曲线及直线来描述字体外形轮廓,特点是:TrueType既可作打印字体,又可用作屏幕显示;由于它用指令对字形进行描述,因此它与分辨率无关,输出时总是按打印机分辨率输出。无论放大或缩小,字符总是光滑的,不会有锯齿出现。但相对PostScript字体来说,其质量要差一些。特别是在文字太小时,就表现得不是很清楚。truetype字便宜,字款丰富。但一般情况厂truetype字不能直接由rip输出。需要经过特殊处理,比如转成曲线或输出时下载,使用起来较麻烦。速度也要慢一些,尤其是处理大量文字时很不方便,因此不适合用来作为页面的正文文字使用。

PostScript汉字库分为显示字库和打印字库,显示字库安装在制作计算机上,用来制作版面时显示用,通常由低分辨率的点阵字构成。打印字库要挂接在rip上,在解释页面时由rip把需要的字库调入页面并解释成记录的点阵。 PostScript汉字使用方便,输出速度快,是输出中心必备的。

TrueType vs. PostScript:TrueType有着更好的hinting属性,如果定义了hinting的话。这是一个小陷阱:必须要有人愿意为这些特性编写代码。Hinting是一种技术,用于数字化字体的平滑显示。它负责在小字号下处理矢量数据并将其转换为像素,以精确的显示字体;可以把hinting视作一种抗锯齿技术。这是另一个陷阱:hinting仅仅在屏幕小字号显示时才有用武之地。

TrueType并非不可以印刷,之所以不适合用于高端印刷,是因为打印机必须将TrueType矢量数据转换成PostScript语言。但不幸的是,PostScript不象TrueType那样支持如此多的曲线段 。
TrueType 使用二次曲线(quadratic curves)。和你在 Illustrator 和 Photoshop 中绘图时所用的曲线不同,二次曲线需要4个点来定义一条曲线。PostScript 使用三次曲线(cubic curves)。这是我们所熟悉的曲线,我们第一次在Illustrator 或 Photoshop 中理解和学习钢笔工具还是一个痛苦的回忆,但现在我们已经是轻车熟路了。三次曲线只需要3个点就可以定义一条曲线。对吧?因为三次方的意思就是自乘3次(其实应该是两次..译注),一个立方体就是3D。当TrueType 转换为 PostScript的时候,不是所有的二次曲线都能够转换为平滑的三次曲线。

OpenType,是一种可缩放字型(scalable font)电脑字体类型,采用PostScript格式,是美国微软公司与Adobe公司联合开发,用来替代TrueType字型的新字型。这类字体的文件扩展名为.otf,类型代码是OTTO,现行标准为OpenType 1.4。OpenType最初发表于1996年,并在2000年之后出现大量字体。它源于微软公司的TrueType Open字型,TrueType Open字型又源于TrueType字型。OpenType font包括了Adobe CID-Keyed font技术。Adobe公司已经在2002年末将其字体库全部改用OpenType格式。到2005年大概有一万多种OpenType字体,Adobe产品占了三分之一。

结论?使用Open Type格式,因为它可以同时嵌入TrueType 和 PostScript信息,同时它又是跨平台兼容的。全世界最棒!加油Adobe和Microsoft!听完我上面的解释,你应该能理解为什么要使用Open Type格式而不是其他格式了吧?如果你还想知道OpenType格式为什么如此优秀,看看这段引自Thomas Phinney的文章TrueType vs. PostScript Type 1中的描述:OpenType 将 PostScript 或 TrueType 轮廓都放入一个 TrueType 风格的“包装袋”中。应用程序和大多数的操作系统都在这个字体的子系统之外操作,不再关心这个“包装袋”中装的是什么类型的字体。

\subsection{ClearType技术}
ClearType,是由美国微软公司在其视窗操作系统中提供的屏幕亚像素微调字体平滑工具,让Windows字体更加漂亮。ClearType主要是针对LCD液晶显示器设计,可提高文字的清晰度。基本原理是,将显示器的R, G, B各个次像素也发光,让其色调进行微妙调整,可以达到实际分辨率以上(横方向分辨率的三倍)的纤细文字的显示效果。Windows上的像素和显示器上的像素对应的液晶显示器上效果最为明显,使用阶调控制一般CRT显示器上也可以得到一些效果。在Windows XP平台上,这项技术默认是关闭,到了Internet Explorer 7才默认为开启。而与ClearType几乎同样的技术在苹果电脑的Mac OS操作系统中,早在1998年发布的Mac OS 8.5就已经使用了。另外,依靠ClearType技术提高字体的可读性,相当程度上依赖于使用的字体,所以和原有的标准抗锯齿技术不能进行单纯比较。如果显示器不具有适用于ClearType的像素组合特性,以ClearType显示文字的实际效果会比使用前还要差。大多Windows默认的中文字体在显示小文字时使用点阵来显示,不使用ClearType。微软在Windows Vista里,新发布了两个ClearType中文字体:微软雅黑和微软正黑体。

为什么英文矢量字体就可以直接使用 ClearType 来进行平滑显示呢?这是因为大多数优秀的英文字体并不是采用内嵌点阵的方式来进行优化的,它们采用的是一种叫做 Hinting (字形微调)的技术来对小字号的显示进行优化。
  我们知道,矢量字体是可以无限平滑缩放的,在使用的时候,要通过操作系统的字体引擎自动的解析渲染为实际的像素,才能够在屏幕上显示出来。但是在字号很小的时候,由于能使用的像素非常有限,这种自动解析会出现很多问题,例如笔画粗细不匀,文字之间高低不齐,甚至笔画模糊无法识别等。因此必须由字体设计师人工干预,在矢量字库中嵌入一些附加的提示信息,来告诉字体渲染引擎在某个特定的字号下面,应该如何对这个字符的细节进行修正,才能准确的显示。这种在矢量字体中嵌入的提示信息,就叫做 Hinting 。
  对于中文字体来说,这种提示就更为重要,因为中文的笔画繁多,自动解析的错误也就更多更严重。在字号更小的情况下,根本无法显示全部的笔画,这时候还需要设计师在不影响整体的情况下,对笔画进行取舍,去掉一些不影响识别的笔画,否则这个文字就会因糊成一团无法识别。 Hinting 调整的范围需要涵盖各级小字号,一般最少要包括 9px - 18px 这个常用的字号区间。这种 Hinting ,即使是对于非常有经验的设计师,也是非常高难度而且费时费力的工作。

英文只有 26 个字母,但是对于中文的汉字情况就复杂的多了,仅仅是最常用的汉字就有 6000 个,然后为了在简繁体混排时候能完美的显示,就必须同时包含繁体和简体两套字符,再加上众多的不常用但是会在古籍文献中非常重要的生僻字,一套比较完整的大字符集字库所包含的字符数目将接近 3 万个。仅仅是这矢量造字的工作就是非常浩大的。
  这还不算,作为一套功能完整的正文字体,还需要考虑到斜体和粗体的显示。所有的斜体状态,也同样必须由设计师对不同的字号指定不同的 Hinting ,否则就会有显示问题。为了更完美的显示粗体,微软决定将标准体和粗体分开,作为两套单独的字体来设计,安装时也是两套字体,但在系统中使用时是显示为一套字体的不同状态。这套单独的黑体也同样需要单独造字,然后指定一系列的 Hinting 和斜体 Hinting 。因此要开发一套优秀的中文大型字库,耗费的人力物力是惊人的。这也正是这套字体会如此昂贵的原因之一。

Hinting信息是评价一款优秀矢量字体的一个重要指标,良好的Hinting能在小字号下面提供和内嵌点阵字一样优秀的显示质量,同时又降低内存的消耗。虽然我们现在已经拥有不少不错的矢量中文字体,但适合屏幕显示的正文字体很少,而包含完善 Hinting 信息的,一个也没有。所以,如果要在中文 Vista 平台下彻底完美的实现文本的平滑显示,微软就必须全新开发一套具备完善Hinting信息的ClearType中文字体。

\subsection{字体文件格式}

TTF(TrueTypeFont)是Apple公司和Microsoft公司共同推出的字体文件格式,随着windows的流行,已经变成最常用的一种字体文件表示方式。ttc是microsoft开发的新一代字体格式标准,可以使多种truetype字体共享同一笔划信息,有效地节省了字体文件所占空间,增加了共享性。但是有些软件缺乏对这种格式字体的识别,使得ttc字体的编辑产生困难。两者的不同处是 TTC 档会含超过一种字型,例如繁体 Windows 的 Ming.ttc 就包含细明体及新细明体两种字型 (两款字型不同处只是英文固定间距),而 TTF 就只会含一种字型。TTC是几个TTF合成的字库,安装后字体列表中会看到两个以上的字体。两个字体中大部分字都一样时,可以将两种字体做成一个TTC文件,现在常见的TTC中的不同字体,汉字一般没有差别,只是英文符号的宽度不一样,以便适应不同的版面要求。


























\subsection{常见商业字体}
维基百科有条目:\href{http://en.wikipedia.org/wiki/List_of_CJK_fonts}{List of CJK fonts}

\begin{verbatim}
中易宋体&新宋体,即通常被熟知为宋体、新宋体的字体,是由北京中易中标电子信息技术有限公司制作并持有版权的两个TrueType字体。中易宋体&新宋体是随着简体中文版Windows和Microsoft Office一起分发的字体(文件名 Simsun.ttc)。Simsun一直是简体中文版Windows XP系统及之前版本的默认字体。但由于白体的特性,在Windows Vista中已经改用支持OpenType的微软雅黑。因为对于电脑显示器来说,应该选择黑体即无衬线体作为显示器字体,才有助于显示和阅读。此字体西文的半角字符部分由于采用等宽字体设计,被指衬线和笔画的比例太夸张而不便阅读。此宋体在8pt时已经无法正常地看清(即使打开了ClearType)。此外,宋体只有区区几个最佳字体大小,在某些大小时(如14pt及以上)会发现字的笔划有残缺、断裂、粗细不均的问题(若打开了ClearType的话,“横”仍然会看到有断裂的地方),这主要是字体没有加入Hinting信息的缘故。

中易黑体,Windows XP系统及之前版本中的默认黑体。通常被熟知为``黑体''的电脑字体其实是中易黑体,是由北京中易中标电子信息技术有限公司制作并持有版权的一种TrueType字体。中易黑体是随着简体中文版Windows和Microsoft Office一起分发的字体,文件名Simhei.ttc,目前版本为5.01。此字形同中易宋体的字形设计被指太琐碎,西文部分采用等宽字体,使得美感颇为降低。

微软雅黑是美国微软公司委托中国方正集团设计的一款全面支持ClearType技术的字体。蒙纳公司(Monotype Corporation)负责了字体的修飾(Hinting)工作。它属于OpenType类型,文件名是MSYH.TTF,在字体设计上属于无衬线字体和黑体。在使用ClearType功能的液晶显示器中,微软雅黑比以前Windows XP默认的中易宋体更加的清晰易读。

微软正黑体,微软公司的一款全面支援ClearType技术的TrueType 无衬线字体,用于繁体中文系统。随Windows Vista及Office 2007(繁体中文)一起发布,是Windows Vista缺省字体,在使用的ClearType功能的液晶显示器中,其比以前Windows XP缺省的新细明体更加清晰易读。中文世界里一套合适的 ClearType 屏幕正文显示字体字体必须能解决在 ClearType 平滑显示状态下小字号正常阅读的问题。现有的所有中文字库都无法在 ClearType 平滑显示状态下完美的文本显示。之前的宋体之所以能够在小字号点阵状态下很好的显示,是由于宋体在矢量字库中嵌入了 12 、 14 、 16 、 18 等几个点阵字库,才得以比较优秀的显示。但在 ClearType 状态下,继续采用这样内嵌点阵的方式来显示汉字,就会和平滑显示的英文粗细不一致,同时风格上非常的不协调。由于当初的宋体不是为平滑显示而设计的,强制平滑显示的效果就显得纤细发虚,看起来很模糊。

对于微软雅黑和微软正黑,不好简单的用简体或者繁体来区分他们,因为这两套字体都同时包含了比较完整的简繁体汉字,以确保在简体和繁体混排的页面上都能够完美的显示。但由于两岸的文教部门在各自的文字规范中对汉字的写法规定有很多细节上的不同,所以这两套字形在正式场合是不能混淆使用的。同样的,日文的Meiryo字体中也包含了大量的繁体汉字,不过由于汉字在日本也经过了上千年的演变,日文中的汉字写法和中国大陆和台湾也有着相当的区别。

华文细黑,是由中国常州华文印刷新技术有限公司(SinoType,找不到官网,不知是否倒闭)制作并持有版权的一种TrueType电脑字体。在无中文的环境下显示的名称为STHeiti Light或STXihei,它属于华文黑体系列字体之一。在设计上,它属于黑体或无衬线体。

华文宋体,常州华文印刷有限公司推出的字体。较平常的宋体相比,华文宋体在审美上有着新的开创与突破。目前,华文宋体已广泛应用到商标设计,广告设计,报纸与图书等行业中,效果较好。

威锋数位(DynaComware),原名华康科技(DynaLab),是台湾的电脑软件公司,为少数以电脑字型为主要开发产品的公司。自Windows 3.0推出时起,华康科技与微软公司合作将Windows中文化,并开始提供字型至今。繁体中文版Windows常用的细明体、新细明体及标楷体等皆由此公司制作提供。

RedHat发布Liberation系列,包含“Liberation Serif”“Liberation Sans”“Liberation Mono”三种字体,目标非常明确,就是取代微软的“Times New Roman”“Arial”“Courier New”。字形尺寸与微软的完全一样,所以用这三种字体代替微软的那三种不会导致文档排版的偏差与错乱。

   Nimbus系列,包含“Nimbus Roman No9 L”“Nimbus Sans L”“Nimbus Mono L”三种。目标直指Adobe的“Times”“Helvetica”“Courier”,主要供打印使用。与Liberation系列不同的是,其衬线看起来和微软的很接近,但尺寸不同。因为微软的那几款字体本来也就是针对Adobe的,字形和Adobe的几乎一样而尺寸却并没做到一样。

\end{verbatim}

















\section{字号}
px:pixel,像素,屏幕上显示的最小单位;1px没有绝对的大小,受分辨率影响。
pt:point,点,是印刷业一个标准的长度单位,1pt=1/72 inch = 0.35146mm;1 inch =2.54cm;
dpi:dots per inch,意思是指每一英吋(1inch=2.54cm)长度中,取样或可显示或输出点的数目。打印机所设定之分辨率的 DPI 值越高,印出的图像会越精细。打印机通常可以调校分辨率。例如撞针打印机,分辨率通常是60至90DPI。喷墨打印机则可达1200DPI,甚至9600DPI。激光打印机则有600至1200DPI。一般显示器为 72 dpi,印刷所需位图的 DPI 数则视印刷网线数而定。一般 150 线印刷品质需要300dpi的位图。
ppi:pixels per inch.


点数制又叫磅数制,是英文point的音译,缩写为P,既不是公制也不是英制,是印刷中专用的尺度。
中国大都使用英美点数制。
1点(1P)=0.35146mm

号数制是以互不成倍数的几种活字为标准,加倍或减半自成体系。
? 四号序列:一号、四号、小六号
? 五号序列:初号、二号、五号、七号
? 小五号序列:小初号、小二号、小五号、八号
? 六号序列:三号、六号

\begin{table}
	\centering
	\begin{tabular}{ccc}

号数&            点数&                 尺寸(mm)\\
(无名)&          72&                 25.305\\
大特号&          63&                 22.142\\
特号&            54&                 18.979\\
初号&            42&                 14.761\\
小初号&          36&                 12.653\\
大一号&          31.5&                 11.071\\
一(头)号&      28&                 9.841\\
二号&            21&                 7.381\\
小二号&          18&                 6.326\\
三号&            16&                 5.623\\
四号&            14&                 4.920\\
小四号&          12&                 4.218\\
五号&            10.5&                 3.690\\
小五号&          9&                 3.163\\
六号&            8&                 2.812\\
小六号&          6.875&                 2.416\\
七号&            5.25&                 1.845\\
八号&            4.5&                 1.581\\
	\end{tabular}
	\caption{号数与点数}
\end{table}

从上表中可以看出,常用的MS-WORD软件中字号的大小与印刷业中字号的大小是不一致的。如MS-WORD中的二号字是22磅,但在印刷业中应该是21磅。







