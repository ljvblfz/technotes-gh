
\section{Vim跳转}

\subsection{同时打开两个文件}
\begin{verbatim}
vim -o a.txt b.txt 竖屏
vim -O a.txt b.txt 横屏
vim -p a.txt b.txt 多标签页
:new 新增空白窗口
Ctrl-w = 平分宽高,注意键入=时,ctrl和w必须松开
Ctrl-w +/> 增减高/宽
Ctrl-w -/< 减小高/宽
Ctrl-w 20< 宽度减小20个单位
20Ctrl-w < 宽度减小20个单位
:vertical res[ize] +8 宽度增减8
\end{verbatim}

\subsection{命令行窗口}
Normal模式下键入\verb+q:+。

\subsection{shell跳转}
\begin{verbatim}
:sh 暂时返回shell;exit会返回vim
ctrl+Z 将vi转入后台,用fg恢复vi
\end{verbatim}

\subsection{路径跳转}
\begin{verbatim}
:E[xplore] 裂屏显示文件目录
:Sex 同Explore, 但是是上下裂屏
\end{verbatim}

\subsection{文件跳转}
\begin{verbatim}
:find filename用于文件跳转。
:[vert] sfind filename裂屏跳转。
:tabfind filename 新建标签页并跳转
gf(goto file)跳转到光标下的文件。
\end{verbatim}

有时需要添加path信息,如
\begin{verbatim}
:set path+=<path to the file>
:set path 显示path的当前值
:checkpath 显示所有无法找到的#include文件
:checkpath! 显示所有#include文件
\end{verbatim}
当前目录下的文件不需要添加path信息。

我们还可以在这个命令中使用通配符来进行匹配,如:
\begin{verbatim}
:set path=/usr/include,/usr/include/*
\end{verbatim}
这样,\verb+:find stdio.h+就可以查看一些库文件。

**匹配整个目录树,如:
\begin{verbatim}
:set path=/usr/include/**
:set path=/home/oualline/progs/**/include
\end{verbatim}
空字符串指当前目录,.指我们正在编辑的文件所在的目录,例如下面的命令是告诉Vim查找的目录包括/usr/include及其所的子目录,我们正编辑的文件所在的目录(.)以及当前目录(空串):
\begin{verbatim}
:set path=/usr/include/**,.,,
\end{verbatim}

\subsection{语句跳转}
\begin{verbatim}
[/]{ 查找上/下一个层次高于当前位置的{,用于语句块跳转
[/]# 查找上/下一个层次高于当前位置的#,用于在#ifdef-#endif结构中跳转
[/]( 查找上/下一个层次高于当前位置的(,用于在条件表达式中跳转
[/]/ 查找上/下一个层次高于当前位置的/,用于在注释中跳转
[/][ 查找上/下一个层次为1的{,用于全局函数跳转
[/]m 查找上/下一个层次为1或2的{,在面向对象语言中method对应层次为2的{
{/} 转到上/下一个空行
*/# 转到当前光标所指的单词下/上一次出现的地方
\end{verbatim}
[I 会列出所有包含该标识符的行,其中第一行可能包含变量定义,搜索范围包括了include文件。

[I命令查找任何的标识符.要只查找宏使用[D。

[d显示[D的第一个结果,[i显示[I的第一个结果。

[+tab 跳转至函数声明或变量定义,估计就是[i所指示的结果。gD将搜索范围局限在了当前文件,gd将搜索范围局限在了当前函数。

\subsection{标签页操作}

\begin{verbatim}
vim -p file1 file2 ..
:tabnew file1
:tabe[dit] file2
:tabnew
:tab split
:tabs 显示所有标签页的信息
\end{verbatim}
使用标签页打开文件

\begin{verbatim}
:tabc :q 关闭当前标签页
:tabo 关闭所有标签页
gt 转入下一个标签页
gT 转入上一个标签页
3gt 转入第3个标签页
:tabn 转入下一个标签页
:tabp 转入上一个标签页
:tabfirst, :tablast 顾名思义
:tabdo cmd 对所有标签页执行命令
\end{verbatim}





