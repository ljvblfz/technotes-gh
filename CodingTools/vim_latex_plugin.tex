\section{在Vim中编写tex文件}

\subsection{latexsuite安装与配置}
\begin{shellcmd}
	sudo apt-get install vim-latexsuite
	vim-addons install latex-suite
\end{shellcmd}
意欲在打开tex文件时自动开启latexsuite,需配置.vimrc
详细参见help:latex-suite
\begin{verbatim}
    " REQUIRED. This makes vim invoke Latex-Suite when you open a tex file.
    filetype plugin on
    
    " IMPORTANT: win32 users will need to have 'shellslash' set so that latex
    " can be called correctly.
    set shellslash
    
    " IMPORTANT: grep will sometimes skip displaying the file name if you
    " search in a singe file. This will confuse Latex-Suite. Set your grep
    " program to always generate a file-name.
    set grepprg=grep\ -nH\ $*
    
    " OPTIONAL: This enables automatic indentation as you type.
    filetype indent on
    
    " OPTIONAL: Starting with Vim 7, the filetype of empty .tex files defaults to
    " 'plaintex' instead of 'tex', which results in vim-latex not being loaded.
    " The following changes the default filetype back to 'tex':
    let g:tex_flavor='latex'

\end{verbatim}

\subsection{配置文件}
配置文件texrc的路径为:
~/.vim/ftplugin/latex-suite
\subsection{环境插入}
F5用于输入环境。

\subsection{分节}
文档上给出了快捷键:
\begin{verbatim}
SPA for part
SCH for chapter
SSE for section
SSS for subsection
SS2 for subsubsection
SPG for paragraph
SSP for subparagraph
\end{verbatim}


\subsection{借助sort命令多域排序}
如果以\&为分隔符(常见于latex表格),第二列升序,第一列降序排序,则有:
\verb+sort -k 2n -k 1r -t'&' FILENAME+

\subsection{表格标记添加}
如果表格是从word或网页上拷贝的,则需要添加\&和换行符号。在vim中依次执行一下命令:
\begin{verbatim}
:22,40s/\s\+$//  #去除行尾空格
:22,40s/\(\s\+\)/\&\1/g #空格前添加&
:22,40s/$/\\\\/g #行尾添加\\符号
\end{verbatim}


