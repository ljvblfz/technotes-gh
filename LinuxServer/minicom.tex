\section{minicom串口配置}

\begin{verbatim}
创建或编辑配置文件,保存为/etc/minirc.Filename或/etc/minicom/minirc.Filename,因系统而异。
minicom -s Filename
\end{verbatim}
配置文件指定设备名、波特率等。Dell服务器的串口设备名为/dev/ttyS0, 4个USB接口的设备名为/dev/ttyUSB0-/dev/ttyUSB3。对于Hili的串口连接,要求码率为115200, 软、硬件流控制都设置为No。



\begin{verbatim}
按照指定配置文件,登录串口设备
minicom Filename
\end{verbatim}


minirc.dfl指定了默认配置,minicom命令不加任何参数时安装该文件进行初始化。

登录后ctrl+a z显示各种快捷键选项,ctrl+a x退出并reset,ctrl+a q退出不reset。


