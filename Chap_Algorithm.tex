\section{二分查找代码示例}

程序可以有多种写法,主要有如下区别:
\begin{itemize}
    \item 不变式区间的开闭性。不变式:待搜数据t不存在,或属于不变式区间。区间可以为[l,u]或(l,u]。区间的开闭性影响迭代更新的方式。
    \item 每次循环的比较操作次数,可以为1次或两次。每次循环如果作两次判断,效率较低,但循环次数可能会减少,因为可以直接发现目标退出循环。可以将等于比较合并到大于或小于比较中来减少判断。
\end{itemize}


当区间长度至少为3时,下轮迭代区间必能能够缩小。当区间长度为2或1时,按照\verb|m=(u+l)/2|计算到的中点m即为左端点l,如果区间端点按照l=m方式来更新,会导致区间不能缩小。当区间长度为1时,m同时等于l和u,当区间右端点按照u=m来更新时,也会发生区间不能缩小的情形,如编程不慎则会造成死循环。

如果采用全闭区间-两次比较的方式编程,u和l都不会直接赋为m,则只需确保区间长度大于零即可,无死循环之虞。如果区间存在开端,或者只进行一次比较,则需要慎重。

可以对算法提出额外要求,即当存在多个满足条件的值时,应能返回下标最低者。

如果数组长度为固定值,如1000,可以通过循环展开以进一步优化代码。
虽然二分搜索可以在log时间内完成搜索,但搜索后如果要插入新元素仍需要线性时间。不过,基于数组的连续操作具有很好的cache友好性。

\subsection{半闭区间,单次比较}
\cite{pp}9.3。 x[-1],x[n]作哨兵,但未真正访问。u和l都会直接赋为m,因此循环条件是区间长度至少为3,即l+1 != u。t包含于(l,u]。左边为开,能够保证返回最低下标。
\label{codes:binsort}

\begin{lstlisting}[language=C]

int binarysearch3(DataType t)
{	
	int l = -1, u = n, m;
	while (l+1 != u) {
		m = (l + u) / 2;
		if (x[m] < t) l = m;
		else u = m;
	}
	if (u >= n || x[u] != t) return -1;
	return u;
}
\end{lstlisting}

\subsection{半闭区间双次比较}
将单次比较改为双次比较没有难度。只需要多加一次相等比较。

\begin{lstlisting}[language=C]

int binarysearch3(DataType t)
{	
	int l = -1, u = n, m;
	while (l+1 != u) {
		m = (l + u) / 2;
		if (x[m] < t) l = m;
		else if x[m] &=& t return m;
		else u = m;
	}
	if (u >= n || x[u] != t) return -1;
	return u;
}
\end{lstlisting}

\subsection{闭区间单比较}
\cite{self}一段Python代码:
\begin{lstlisting}[language=Python]

def binsearch(x, n, t):
    #range size >= 1
    assert n > 0

    if x[0] > t or x[n-1] < t: return -1;

    l = 0
    u = n-1
    while (u-l >= 1): 
        #x[l] <= t <= x[u], l == 0 or x[l-1] < t, range size >= 2
        m = l + (u - l) / 2;
        print "[%d - %d], %d"%(l,u,m)
        if x[m] < t: 
            l = m + 1
        else: 
            u = m
    
    #range size is 1 
    if x[l] == t: return l
    return -1

\end{lstlisting}

\cite{bop}3.11也给出了闭区间单次比较算法的实现,该实现不满足返回最小下标的条件。代码没有对输入参数进行检查,如果输入区间为零,程序功能可能与期望不符。书中认为求midIndex不可以相加除以2,以防溢出。其实如此大数本不太可能作为数组的长度。
因为l按照来l=m方式更新,所以区间长度为2时即需要退出循环。更新u的方式也过于保守,是按照开区间的方式进行更新。

其代码的Python等效版本写为:
\begin{lstlisting}[language=Python]

def binsearch(x,l,u,t):
    while (l < u - 1): #range lengh >= 3 
        m = l + (u - l) / 2;
        print "[%d - %d], %d"%(l,u,m)
        if x[m] <= t:
            l = m
        else: 
            u = m # better be u = m-1
    
    #range size is 2 or 1 
    if x[u] == t: return u
    if x[l] == t: return l
    return -1

\end{lstlisting}

原书程序为:

\begin{lstlisting}[language=C]
int bisearch(char** arr, int b, int e, char* v)
{
    int minIndex = b, maxIndex = e, midIndex;
    //循环结束有两种情况:
    //若minIndex为偶数则minIndex ==  maxIndex
    //否则就是minIndex ==  maxIndex -1
    while (minIndex < maxIndex -1)
    {
	minIndex = minIndex + (maxIndex - minIndex) / 2;
	if (strcmp(arr[minIndex], v) <= 0)
	{
	    minIndex = midIndex;
	}
	else
	{
	    //不需要minIndex - 1, 防止minIndex == maxIndex
	    maxIndex = midIndex;
	}
    }

    if (!strcmp(arr[maxIndex],v))//先判断序号最大的值
    {
	return maxIndex;
    }
    else if (!strcmp(arr[minIndex],v))
    {
	return minIndex;
    }
    else
    {
	return -1;
    }
}
\end{lstlisting}


\subsection{闭区间双次比较}
\cite{pp}5.1。t包含于[l,u]。u和l都不会赋值为m。循环条件是区间长度至少为1,即l<=u。退出循环则意味着无解。如果一开始元素个数就为零,自然无法进入循环,算作无解。
\begin{lstlisting}[language=C]
int binarysearch2(DataType t)
{	
	int l, u, m;
	l = 0;
	u = n-1;
	while (l <= u) //区间大于等于1 
	{
		m = (l + u) / 2;
		if (x[m] < t)
			l = m+1;
		else if (x[m] == t)
			return m;
		else /* x[m] > t */
			u = m-1;
	}

	return -1;
}


\end{lstlisting}

\cite{sedgewick}P99也给出了这种算法:

\begin{lstlisting}[language=Java]
private int rank(int key)
{ // Binary search.
    int lo = 0;
    int hi = a.length - 1;
    while (lo <= hi)
    { // Key is in a[lo..hi] or not present.
	int mid = lo + (hi - lo) / 2;
	if (key < a[mid]) hi = mid - 1;
	else if (key > a[mid]) lo = mid + 1;
	else
	return mid;
    }
    return -1;
}

\end{lstlisting}










\section{快速排序代码示例}

\label{codes:quicksort}


相关原理参考\label{summary:quicksort}。

\begin{lstlisting}[language=C]

/* Simplest version, Lomuto partitioning */
void qsort1(int l, int u)
{	int i, m;
	if (l >= u)
		return;
	m = l;
	for (i = l+1; i <= u; i++)
		if (x[i] < x[l])
			swap(++m, i);
	swap(l, m);
	qsort1(l, m-1);
	qsort1(m+1, u);
}
\end{lstlisting}

\begin{lstlisting}[language=C]
/* Sedgewick's version of Lomuto, with sentinel */
void qsort2(int l, int u)
{	int i, m;
	if (l >= u)
		return;
	m = i = u+1;
	do {
		do i--; while (x[i] < x[l]);
		swap(--m, i);
	} while (i > l);
	qsort2(l, m-1);
	qsort2(m+1, u);
}
\end{lstlisting}


\begin{lstlisting}[language=C]
/* Two-way partitioning */
void qsort3(int l, int u)
{	int i, j;
	DType t;
	if (l >= u)
		return;
	t = x[l];
	i = l;
	j = u+1;
	for (;;) {
		do i++; while (i <= u && x[i] < t);
		do j--; while (x[j] > t);
		if (i > j)
			break;
		swap(i, j);
	}
	swap(l, j);
	qsort3(l, j-1);
	qsort3(j+1, u);
}
\end{lstlisting}


\begin{lstlisting}[language=C]
/* qsort3 + randomization + isort small subarrays + swap inline */
int cutoff = 50;
void qsort4(int l, int u)
{	int i, j;
	DType t, temp;
	if (u - l < cutoff)
		return;
	swap(l, randint(l, u));
	t = x[l];
	i = l;
	j = u+1;
	for (;;) {
		do i++; while (i <= u && x[i] < t);
		do j--; while (x[j] > t);
		if (i > j)
			break;
		temp = x[i]; x[i] = x[j]; x[j] = temp;
	}
	swap(l, j);
	qsort4(l, j-1);
	qsort4(j+1, u);
}

\end{lstlisting}

\cite{sedgewick}提出了三路分割,用于应对有大量keys重复的情形,并证明三路分割的快速排序具有熵最优性。该算法使用了$(2lg2)NH$次比较,H为香农熵,$H=-\sum{p_{k}lnp_k}$。


\begin{lstlisting}[language=Java]

public class Quick3way
{
    private static void sort(Comparable[] a, int lo, int hi)
    { // See page 289 for public sort() that calls this method.
	if (hi <= lo) return;
	int lt = lo, i = lo+1, gt = hi;
	Comparable v = a[lo];
	while (i <= gt)
	{
	    int cmp = a[i].compareTo(v);
	    if (cmp < 0) exch(a, lt++, i++);
	    else if (cmp > 0) exch(a, i, gt--);
	    else
	    i++;
	} // Now a[lo..lt-1] < v = a[lt..gt] < a[gt+1..hi].
	sort(a, lo, lt - 1);
	sort(a, gt + 1, hi);
    }
}
\end{lstlisting}





\section{数据结构问题举例}
\subsection{简洁实现升序链表插入}
链表插入的繁琐之处在于,要判断头指针是否为空,以及是否需要更新头指针(头前插)。
通过哨兵(存放上限值,\cite{pp}13.2)使得链表至少有一个元素,无需检查head是否为空。
通过使用指向指针的指针(\cite{pp}13.6.4),避免区分头前和头后插入两种情况。
\begin{verbatim}
for(pp=&head;(*pp)->val<t;pp=&((*pp))->next);
if (*pp)->val &=& t:return; 
*pp=new node(val=t, next=*pp);
\end{verbatim}
\label{problem:listInsert}。

\subsection{简洁实现BST插入}
通过哨兵(存放待搜索值)和指向指针的指针(\cite{pp}13.6.7),使得BST至少有一个元素,且无需单独判断是否需要更新root指针的值。
\begin{verbatim}
sentinel->val=t;
for(pp=&root;(*pp)->val!=t;pp=(t<(*pp)->val)?(&((*pp)->left)):(&((*pp)->right)));
if *pp&=& sentinel:*pp=new node(val=t, left=right=sentinel);
\end{verbatim}
\label{problem:BSTInsert}。


\subsection{m个互异整数搜索结构}
\cite{pp}13介绍了有序数组,有序链表,BST,位向量,桶。
链表相对于数组,适用于待存放元素个数不明确的情形。虽然链表在插入数据时不需要移动数据,但其内存访问不连续且需要额外空间存放指针,对cache不友好,性能甚至不如数组。对于纯搜索过程,有序数组可以在log时间内完成,链表需要线性时间。桶实质是一种散列结构,用链表处理碰撞。


\subsection{实现快速返回最大值的队列}
\cite{bop}3.7。
方法零:按照传统方式,以数组或链表存储队列,两个指针指向头尾。MaxVal操作复杂度为$O(N)$,增、删复杂度为常数。个人认为可以保存最大值,增删是予以更新,Enqueue的复杂度为常数, dequeue复杂度线性, MaxVal复杂度为常数。方法一:使用堆。Enqueue和Dequeue为$O(logN)$, MaxVal为常数。类似于\cite{ita}中的优先级队列。方法二:\cite{bop}提出了一种$O(1)$时间MaxVal操作的栈结构,用链表连接不同状态时的最大值,并用两个这种栈串联实现了队列。


\subsection{有序文件多路归并}
\cite{pp}14.6.4.d。用堆(优先级队列)表示每个文件的下一个元素。从堆中选出最小元素,再从对应文件中补上。

\subsection{二叉树中最远节点}
\cite{bop}3.8。
经分析可知,最远两节点有两种情况:两个叶子节点;根节点和一个叶子节点,此时根结点左右子树其一必为空。总之,为某子树(或本树)左右子树最大高度之和(叶子结点高度为1)。可深度优先遍历,检查以各结点为根结点的子树的左右子树的高度之和,更新当前最优解。书上使用的是递归方法,要求自作非递归法。

\subsection{分层遍历二叉树}
\cite{bop}3.10。有两问。第一问是打印某一层结点。第二问是从上到下分层打印各结点,层内从左到右。扩展题是从下到上分层打印各结点,层内从左到右或从右到左。对于第一问,按照正常流程遍历树,判断层数合适则打印。其他问题,使用一个数组,将根结点压入,遍历数组,同时按照一定规则将各结点压入数组末尾。最终数组中元素排列符合要求。此题比较简单。


\subsection{实现快速返回最大值的队列}
\cite{bop}3.7。
方法零:按照传统方式,以数组或链表存储队列,两个指针指向头尾。MaxVal操作复杂度为$O(N)$,增、删复杂度为常数。个人认为可以保存最大值,增删是予以更新,Enqueue的复杂度为常数, dequeue复杂度线性, MaxVal复杂度为常数。方法一:使用堆。Enqueue和Dequeue为$O(logN)$, MaxVal为常数。类似于\cite{ita}中的优先级队列。方法二:\cite{bop}提出了一种$O(1)$时间MaxVal操作的栈结构,用链表连接不同状态时的最大值,并用两个这种栈串联实现了队列。

\subsection{二叉树中最远节点}
\cite{bop}3.8。
经分析可知,最远两节点有两种情况:两个叶子节点;根节点和一个叶子节点,此时根结点左右子树其一必为空。总之,为某子树(或本树)左右子树最大高度之和(叶子结点高度为1)。可深度优先遍历,检查以各结点为根结点的子树的左右子树的高度之和,更新当前最优解。书上使用的是递归方法,要求自作非递归法。

\subsection{分层遍历二叉树}
\cite{bop}3.10。有两问。第一问是打印某一层结点。第二问是从上到下分层打印各结点,层内从左到右。扩展题是从下到上分层打印各结点,层内从左到右或从右到左。对于第一问,按照正常流程遍历树,判断层数合适则打印。其他问题,使用一个数组,将根结点压入,遍历数组,同时按照一定规则将各结点压入数组末尾。最终数组中元素排列符合要求。此题比较简单。



\subsection{稀疏矩阵存储}
\cite{pp}10.2提出用数组表示各列,指向行链表。同时提出如果编程语言不支持指针,可以把所有行依次相连为单一大数组,用另一个数组表示各列首元素在大数组中的位置。\cite{pp}10.6.2提出,依此按照x、y坐标排列各元素,以支持二分搜索。对于其他空间节省技术,经常是通过仅存储差量来实现,r如\cite{pp}10.6.4,10.6.5。
\label{problem:sparseMatrix}


\section{数据分析问题举例}

\subsection{数组中最大的K个数}
参\cite{bop}2.5, \cite{ita}9.2-9.3(\cite{ita}称找到第k小的数为kth order statistics)。
方法零:全部排序,$O(NlogN)$, 适用于多次查询场景;或只对前k个数部分排序,$O(nk)$,可使用利用选择或交换的排序算法\cite{subsec:sortAthmClass}。
方法一:随机分割保留单侧。\cite{ita}9.2称之为RANDOMIZED-SELECT算法, \cite{pp}11.5.9和\cite{wikipedia}称之为选择算法。期望运行时间为线性,最差为平方时间。\cite{ita}9.3的SELECT算法通过精心挑选pivot来确保最差时间也为线性。涉及随机访问,不适合外存储操作。\cite{bop}2.5.3认为线性算法常数项太大,未必好。
方法二:值域二分查找。每次迭代需要线性时间,迭代次数为$log_{2}(V_{max}-V_{min}/\delta)$,$\delta$为元素间最小间隔(对于整数,$\delta$为1)。对于均匀分布的数据,总时间为NlogN。文件操作优化(\cite{bop}2.5.3):每次迭代,将当期搜索区间的样本复制到新的文件中(\cite{self}:磁盘可能频繁换道,但适合于两个磁盘或磁带,参\ref{subsec:dupNumberInFile}),当区间内的样本数能够全部载入内存时,则不需要再进行文件操作。\cite{self}认为,为减少磁盘换道,对文件的读取采用搅拌式。
方法三:前K个元素生成容量为K的大顶堆,从第K+1个元素开始遍历,更新堆的内容。时间为$O(NlogK)$适合外存储操作。如果K个数可以全部载入内存,则只遍历文件一遍,不涉及随机访问。如果K也很大,需要将k分割为多个可以载入内存的部分(\cite{bop}2.5.4)。
\label{subsec:orderStatistics}

方法四:如果值域有限且非常小,计数排序(\cite{bop}2.5.5)。
\subsection{找到数组中最大数和最小数}
\cite{bop}2.10。如分别独立找到最大值和最小值,需要2N次比较。其他方法可以达到1.5N次比较。
方法一:每邻近的一对数为一组,组内比较,较小者调整到奇数位置。进而分别在奇数位置和偶数位置寻找最小和最大。
方法二:每邻近的一对数为一组,遍历每个组,每组内用3次比较更新当前的最大值和最小值。
方法三:分治法。
其时间复杂度见:\ref{divide_complex}。


\subsection{求众数和绝对众数}
求众数(Mode):
\cite{pp}11.5.1。
方法1:基于选择算法,参考\ref{subsec:orderStatistics},取轴值x, 根据快排的过程,小于x的放在左边,大于x的放在右边。同时统计x的出现次数T。 如果x左边的个数(不算X)多于T,向左递归;同理,如果x右边的个数对于T,向右递归。复杂度为$O(nlogn)$。
方法2:基于分布的统计,利用数组(如果数据取值范围为有限区间内的整数)或散列。时间复杂度为线性,空间复杂度较高。
方法3:先排序。复杂度主要来自排序操作。

求绝对众数:
\cite{bop}2.3,发帖水王问题。
方法:不断去除相异的两个值。
扩展为有三个ID,各自超过ID出现次数的四分之一。

\subsection{数组相邻差值最大值}
\cite{bop}2.11扩展1\\
所述相邻指值的相邻,并非数组中位置的相邻。
根据抽屉原理,相邻差值的最大值应不小于delta=(Vmax-Vmin)/(N-1)。将取值域分解为长为delta的若干桶,每隔桶内不会有我们希望的值。记录每个桶的最大值和最小值,然后遍历所有的桶即可。时间和空间复杂度都是线性。



\subsection{找出数组中和为给定值S的两个数}
\cite{bop}2.12, \cite{ms100}14\\
方法一:排序,对数组中每个数A,用二分法查找S-A是否在数组中。$O(NlogN)$。方法二:建立Hash表,使得用$O(1)$时间即可求出某个值是否在数组中,对数组中每个数A,查找S-A是否在数组中。时间与空间复杂度均为线性。方法三:双下标相向游动.先排序,然后从数组的头和尾分别寻找一个数,使其和为S。查找时间为线性。书中提到,很多题目要求返回两个数组下标的,适用类似方法,即在一个循环体里用两个变量反向遍历。

\subsection{子向量最大和}
\cite{bop}2.14, \cite{pp}第8章整章讲述该问题。\label{problem:maxsubvector}\\
方法零:枚举所有子向量$O(N^2)$,计算其和, 总时间$O(N^3)$。若将求和与枚举操作的第二个循环合并,或者预先计算累加数组,总时间$O(N^2)$。方法二:分治法,分别在两个砍半数组中求解局部最优,再求解跨界解,给定两个序列头部,分别向前、向后搜索合适的子数组尾部。总时间$O(NlgN)$。方法三:动态规划,寥寥数行代码,线性时间复杂度。遍历数组,更新已遍历部分以当前位置为结尾的最优解,和已遍历部分最优解,后者的更新依赖于前者。方法四:\cite{self}为便于理解方法三而创,全局最优解必为以某i为结尾的子数组,先用线性的时间和空间求出以各i为结束的子数组的最大值序列maxEnd[i],maxEnd[k]依赖于maxEnd[k-1]。再用线性时间找到maxEnd[i]数组中的最大值即可。方法五:线性时间分治法\cite{pp}8.7.8。分别在两个砍半数组中求出局部最优和跨界最优,则总的跨界最优借即为两个砍半数组跨界最优解之和。
\label{subsec:maxsubsum}

\subsection{总和最接近0的子向量}
\cite{pp}8.7.10,是\ref{problem:maxsubvector}的延伸。使用累加数组,$cum[-1]=0,cum[i]=\sum_{k=0}^{k=i}{x[i]}$,转化为求cum[n]中最接近元素的问题。可以在$O(NlogN)$时间内通过排序来完成。似乎可以仿照原题对动态规划算法,将评价指标更改即可。\label{problem:zerosubvec}

\subsection{总和最接近常数t的子向量}
\cite{pp}8.7.10,是\ref{problem:maxsubvector}的延伸。书上未给出答案。如果仿照\ref{problem:zerosubvec}使用累加数组,转化为求cum[n]中差值最接近t的问题。可以在$O(NlogN)$时间完成排序,通过$O(NlogN)$时间在有序数组中完成查找。似乎可以仿照原题对动态规划算法,将评价指标更改即可。

\subsection{环形数组子向量最大和}
\cite{bop}2.14扩展1\\
将解空间分为两部分,一部分不过尾,参考\ref{subsec:maxsubsum}。另一部分过尾。过尾者,分别从头和尾按照两个方向求出最优子序列。书上如此似乎未能妥善处理两个最优子序列重叠的情况。我认为应该求出数组总和,减去子向量长度最小值。

\label{subsec:maxroundsubsum}

\subsection{二维数组子向量最大和}
\cite{bop}2.15\\
方法零:枚举,$O(N^2 \cdot M^2) \cdot \textrm{二维求和复杂度}$,通过部分和缓存方法,可以将二维求和复杂度做到$O(1)$。所谓部分和就是从某元素到原点的连线为对角线的矩形区域进行求和。方法二:只在Y维度进行枚举,确定上下边,按照一维的方法(参\ref{subsec:maxsubsum})确定左右边,求出子数组之和,$O(NM \cdot min(N,M))$。按照这两种方法,可以将问题扩展为三维、四维。参考\ref{subsec:maxroundsubsum} ,可以求解首位相连情形。


\subsection{最长递增子序列}
\begin{verbatim}
http://en.wikipedia.org/wiki/Longest_increasing_subsequence
\end{verbatim}
\cite{bop}2.16\\
同\ref{subsec:maxsubsum}节不同,此处序列不必连续。
此题如使用枚举方法,子序列共有$2^N$个,不合适。方法一:动态规划,记LIS[i]为以i为末尾的最长递增子序列(书上说是前i个元素中的最长子序列),至少为1。$LIS[i]=max\{1,LIS[k]\}, k<i, array[k]<array[i]$。最终返回$max{LIS[i]}$,时间复杂度为$O{N^2}$,需存储LIS数组。方法二:动态规划,遍历数组,存储$TAIL[j]$为当前(在遍历过程中更新)时刻所有长度为j的递增序列的尾值的最小值,MAXL为当前的最长递增序列,利用TAIL计算LIS,$j \gets MAXL to 1$, 如某j满足TAIL[j]小于array[i],则$LIS[i] \gets j+1$。同时,可能会更新TAIL[j+1],MAXL。时间复杂度$O{N^2}$。方法三:思路同方法二,但在选取j时,从LIS[i-1]开始,并使用二分查找。利用了TAIL为递增数组,及$LIS[i] \le LIS[i-1]+1$。本节及\ref{subsec:maxsubsum}启发我们,数组不仅可看作静态的数据集合,也可看作一个更大的动态增长的数据集的组成部分。

\subsection{数组均匀分割}
\cite{bop}2.18,要求长度为2N的数组分成两个长度为N的数组,且和最接近\\记数组为array[k], 和的一半为S。
方法零:枚举法的复杂度$\left( \begin{array}{c} N \\ 2N \end{array} \right) = \frac{(2N)!}{N!N!}$,找出最接近S的组合。书上未提及此法。
    方法一:动态规划,遍历k,更新集合类$V_i,i \le k, V_i$表示任意i个元素的和的可取值。最终研究$V_N$中的最优值。书中未提及如何据此找到对应分配方案。可能需要在$V_i$中添加附加信息。方法二:如果取值域很小,用boolean数组isOK[i][s]表示是否能存在i个变量其和为s。三维遍历k,i,s, $i \le N,i \le k, s \le S$, 如果isOk[i-1][s-array[k]]为真,则isOk[i][s]为真。



\subsection{坐标系中最近点对}
\cite{bop}2.11\\
方法零:枚举,$O(N^2)$;方法一:按X坐标分治,$O(N^2)$;方法二:按X坐标分治,求出不跨界的最近点对的距离为L;探索跨界点对距离时,范围缩小至距界线不超过L的范围,由此进行枚举,总复杂度仍为$O(N^2)$。跨界最近点对必处于Y方向距离小于L的带状区内(面积$2L^2$),且区内点数不超过8,左右半区点数各不超过4,则最近点对在Y方向间的点不会超过6个。书上说用$O(N)$时间完成对所有带状区内最近点对的查找,总时间$O(NlgN)$,具体方法不明。如果在X方向上2L区域内对点按Y值排序,$O(lgN)$,则总时间$O(Nlg^{2}N)$,比枚举法有进步。
扩展题2要求坐标系中最远点对,不解。

\subsection{目标区间是否包含于源区间并集}
\cite{bop}2.19\\
先对源区间进行排序$O(NlogN)$和合并$O(N)$,此后,目标区间包含于源区间并集当且仅当目标区间包含于某个源区间。如果目标区间反包含某源区间,或同某源区间交叠,均不成立。

\subsection{丢失的备份}
\cite{bop}1.5。每个结点都冗余备份为2份,因此活跃ID列表中每个ID会出现两次,但一些结点可能会失效,其ID不出现于列表。在已知有1个/2个结点出故障的情况下,找到其ID。如果只有1个ID失效,将所有活跃的ID相异或,将得到失效ID。如果两个ID失效,事先应计算好所有ID总和,减去活跃ID总和,可以得到一个方程。另一个方程可以基于所有ID的乘积、平方和、异或等。
\subsection{缺失的整数}
\cite{pp}2.1A,文件中包含不足40亿个32位整数(32位整数有42.9亿种取值),要求找到一个缺失的整数。如果内存无法容纳位图(32位整数需要512MB的位图空间),可以在值域二分搜索,每次迭代将当前值域的样本复制到另一个文件上,空域空间因小于值域空间,因此空域能保证缩半。如果题目条件是所有32bit整数中只缺1个,可以异或。因为所有整数异或为0。参\ref{subsec:dupNumberInFile}。

\subsection{至少出现两次的整数}
\cite{pp}2.6.2,文件中包含多于43亿个32位整数(32位整数有42.9亿种取值),要求找到一个重复的整数。如果内存无法容纳位图(32位整数需要512MB的位图空间),可以在值域二分搜索,即查找小于中间值的数值是否冗余,然后值域缩半。每次迭代将当前值域的样本复制到另一个工作磁带上,但空域空间大于值域空间,不能保证空域缩半,因此最差情况下复杂度为$NlogN$。优化:(\cite{pp}2.6.2 Jim Saxe),如果值域容量为m,则只复制m+1个样本,保证空域缩半。另外(\cite{self}, 参\ref{subsec:orderStatistics}),当m足够小时,开始使用位图法或全部载入内存。另外(\cite{self}),即使空域不缩半,如果每次迭代都将空域按一个线性因子缩小,则平均时间也为线性(无限项等比级数之和)。当发现当前值域空间存在冗余时,即可丢弃当前空域位置之后的位置。这样,不需要复制额外的文件,只要更新文件搜索范围即可。进一步,对文件的访问可以采用搅拌式。

\label{subsec:dupNumberInFile}

\subsection{约瑟夫环问题}
\cite{ms100}18,\cite{sword}45。圈有n个结点,击鼓传花式不断删除第m个结点。记该问题为f(n,m),有 $f(n,m)=[f(n-1,m)+m]\bmod n$




\section{进制表示问题举例}
本节中的问题涉及非常大的数字,普通内置数据类型容易溢出,应自定义BigInt类型。
\subsection{N!十进制/二进制中末尾的0个数}
\cite{bop}2.2。\\
对于十进制,0的个数等于所有质因子中5的个数。每隔5,产生一个因子5;每隔25,又产生一个因子5;每隔125,625,\ldots,
对于二进制,类似。另外N!中质因子2的个数等于N-N的二进制表示中1的个数。
\subsection{选择M,使N*M只包含0和1}
\cite{bop}2.8。\\转化为选择符合条件的K,使得K整除N。遍历K,值域按照模划分为N份,每份只保留最小者。K从i位数变成i+1位,相当于用$10^i$加上已访问的所有数,那么,i+1位数模N的结果依次为$10^i mod N$加上已访问的数$mod N$的和再模N。直到mod值为零的集合不再为空。遍历结束的条件,书上提到循环节的概念,不甚明了。

\subsection{Hamming Weight(Population Count)}
\cite{bop}2.1。方法一:基于\verb|b&(b-1)|的运算,除去最末bit1。

\subsection{长内存块中bit1的个数}
\cite{pp}9.5.7\\
方法一:以字节或字为单位,统计每个单位中bit1的个数,再累加。方法二:查表法,对每个单位通过查表的方式获得1的个数,再累加。方法三:统计一个单位的取值分布,对于每个值出现对次数,乘以该值对应的1的个数。

\subsection{高效实现C语言isupper函数}
\cite{pp}9.5.6\\
使用一个256元的表(可以完全纳入缓存),定义
\begin{verbatim}
#define isupper(c) (uppertable[c])
\end{verbatim}
大多数系统为表中每个元素存储几个bit,通过逻辑与操作来提取
\begin{verbatim}
#define isupper(c) (bigtable[c] & UPPER)
#define isalnum(c) (bigtable[c] & (UPPER|LOWER|DIGIT))
\end{verbatim}



\section{数论问题举例}

\subsection{最大公约数}
\cite{bop}2.7。\label{subsec:gcd}
方法0:欧几里德算法,辗转相除.
方法1:欧几里德算法减法版本,避免除法,但增加了减法.\verb|int gcd(i,j){while(i!=j){if(i>j)i-=j;else j-=i;}}|。
方法2:gcd初设为1。若两数均even,则均右移1位,gcd左移1位;若其一为even,则右移至成odd;若均为odd,则其一赋为两者差,必为even。通过最低bit快速判断奇偶性。


\subsection{自然数是否可表示为连续自然数之和}
\cite{bop}2.21。书上未给答案。可能答案是,只要不是2的幂,都可表示为连续自然数之和。可以通过试验数据找到规律并证明之。

\section{计算问题举例}

\subsection{桶计算}
将区间[0,n-1]中的若干整数按分布存放于m个桶(bin)中,计算若干整数应该放在哪个桶中(\cite{pp}13.6.9),即整数值与单桶容量$n/m$做除法,t对应桶是$t/m*n$。如果保持桶数为m,将单桶容量近似提升为2的幂次$2^s \ge n/m > 2^{s-1},s=\lceil log_{m}{n} \rceil$,则可将乘除法局限于预处理过程,在实时计算桶号的过程只使用移位\verb|t>>s|。计算s的过程为,\verb|for(s=1;2^s < n/m;s++);|。循环在$2^s \ge n/m$时停止。

\subsection{计算浮点数均值}
\cite{pp}14.6.4.c, 11.5.1。两个差异较大的浮点数相加会产生精度问题。需要每次取集合中较小的两个数相加(类似于霍夫曼编码),可以用优先级队列来完成。

\subsection{矩阵的幂}
\cite{bop}2.9提及,计算$A^n=A^{2} \cdot A^{4} \cdot A^{8}\dots$,时间复杂度为$O(logn)$。网上有说将矩阵相似对角化,可加速计算。
\label{matrixexpo}
\subsection{Fibonacci数列}
\cite{bop}2.9。方法一:求出通项公式。方法二:$(F_n,F_{n-1})=(F_{n-1},F_{n-2})\cdot A, A=  \left( \begin{array}{cc}1&1\\1&0\end{array}\right)$,进而转化为求$A^n$。

见\ref{matrixexpo}节。



\subsection{运算方式受限的连续自然数求和}
\cite{ms100}第12题。求$ \sum_{i=1}^{n}i $,不得使用乘除法、循环、和判断语句。方法:判断语句可用短路语句替代,固定循环可以展开,然而依赖
于变量的不定循环难以替代。根据$ \sum_{i=1}^{n}i = (n^{2}+n)<<1$,而$n^{2}=\sum_{k=0}^{32}I\{\textrm{n的第k位为1}\}(n<<k)$,循环展开即可。
\subsection{不定子向量频繁增减不定值}
\cite{pp}8.7.12。x[n]长度为n,并执行n次运算:\verb|x[l:u] += v|,其中l,u,v每次会变。该过程消耗平方时间。如果使用差分数组$diff[i]=x[i+1]-x[i]$,\verb|x[l:u] += v|转化为\verb|diff[l-1] += v, diff[u] -=v |,运算n次后再恢复出x,只需线性时间。

\subsection{除掉一个数,使剩余数乘积最大}
\cite{bop}2.13\\
要求不使用除法。方法一:根据所有数乘积为正、负、零三种情况讨论,会发现不需要进行任何运算,很容易可以得到解法。方法二:记数组长度为N,s[i]为前i个数乘积,t[j]为后j个数乘积,耗费O(N)的时间和空间能求出s,t两个数组,那么除去第k个数的所有数的乘积为s[k-1]t[N-k-1],总时间为线性。


\section{字符串问题举例}
\subsection{字符串循环右移K位}
\cite{bop}2.17,\cite{pp}2.1B。题目要求只允许两个附加变量。如无此要求,可以缓存最右K个元素。\\
如果K大于N,先取模,如果K大于N的一半,变换方向, K取作N-K。方法一:循环右移1位,执行K次。算法复杂度$O(N \cdot K)$,若K已被取模,算法复杂度$O(N^2)$。方法二:先整体旋转(线性时间),再分别旋转左K个和右N-K个元素。方法三:\cite{pp}2.3,杂技算法。类似于希尔排序、图书馆排序的跳跃性移动,对内存不友好,编程复杂。方法四:Gries算法\cite{pp}2.6.3,中间不动交换两侧各K个数,然后去掉已经处于最终位置的K个数,递归,编程复杂。跟欧几里德算法版本\ref{subsec:gcd}是同构的。


\subsection{字符串移位包含问题}
\cite{bop}3.1。串A看作环形,是否包含串B。判断A+A是否包含B即可。


\subsection{词典聚类:变位词和电话键盘}
\cite{pp}2.4, 2.6.1。找出词典中所有互为变位词的集合,以及给定单词找出字典中所有它的变位词。词典中所有单词计算其标识,使得所有变位词具有相同标识。标识与单词构成元组,基于该标识对各元组排序,即可将所有变位词聚类。对于后一问题,可以按照前一个问题对各元组排序,执行词典预处理,这样就能使用二分搜索。实际系统往往使用散列技术或数据库系统(\cite{pp}2.6.6答案)。标识可以是单词内对字母按照字母表排序后形成的字符串。可以推广到对数据库中元组基于某规则进行归类。例如将将名字映射为拨号按键序列的问题(\cite{bop}3.2,\cite{pp}2.6.6, \cite{pie})求解所有映射成相同序列的名字,以及给定序列返回所有对应名字,此时标识就是按键序列。
\label{problem:dictClassify}

\subsection{字串枚举:电话号码对应英文单词}
\cite{bop}3.2, \cite{pie}P119\\
问题1:枚举一个电话号码能够构成的所有字母组合,普通枚举问题。问题2:基于词典,一个给定电话号码是否构成有意义的单词。方法一:基于问题1的解答,将每个答案同词典进行匹配,需要遍历词典很多次。方法二:将可以构成单词的电话号码预先存放到数字词典中。方法三:实质可看作词典聚类问题,参\ref{problem:dictClassify},遍历一遍词典,判断每个单词是否符合这个电话号。

\subsection{文档词频统计}
\cite{pp}15.1。不基于词典,单词词形没有限制。利用散列表存储每个单词的词频,表的长度为质数29989,使哈希载荷因子接近1(圣经中有29121个单词,区分大小写),用链表解决冲突。每次查询几乎为常数时间。总时间正比于文档长度。

\subsection{至少重复两次短语最大长度}
\cite{pp}15.2。此题没有单词的概念,只有字符。基于双指针同向扫描的算法需要平方时间。可以采用后缀数组,数组中每个元素为指针,指向文档中每个字符,相当于文档的每一个后缀都在数组中有所对应。然后基于对指针指向的字符串的比较,对后缀数组排序。扫描排序后的后缀数组,查看当前字符串与下一个字符串的共同前缀长度,找最大值。如果字符串比较的时间看作常数,那么总开销取决于排序,为线性乘以对数时间。然而,如果文档所有字符都一样,或者文档是由另一个文档复制了两次而来,那么字符串比较的时间和计算共同前缀的时间可能会正比于文档长度。

扩展:至少重复K次的最长短语(\cite{pp}15.5.8)。 同样对后缀数组排序,扫描后缀数组,查看当前字符串和后面第K-1个字符串的共同前缀长度,这两个字符串的共同前缀也是中间K-2个字符串的前缀,找最大值。

\label{problem:maxlenphrase}。

\subsection{K单词马尔可夫文本预测}
\cite{pp}15.3。基于当前位置K个单词预测下一个单词。在学习阶段,对训练文档,建立后缀数组(参\ref{problem:maxlenphrase})并排序,本题后缀数组的每个指针元素指向文档的各单词,而不是指向所有字符。同时,比较函数也改为,如果前K个单词相同则断定字符串相同)扫描后缀数组。在预测阶段,对于给定K单词短语,利用二分搜索在对数时间内定位到后缀数组中有相同K单词前缀的字符串(if any),在这些字符串中随机选择一个,输出该字符串的下一个单词即可。作为优化(\cite{pp}15.5.8),不对后缀数组进行排序,而是将每个后缀的位置存放在散列表中,使得具有相同K单词前缀的后缀在同一个散列表项中。这样,在预测阶段只用常数时间即可完成对给定短语的搜索。 


\subsection{最长公共子序列LCS}
算法导论动态规划章,子序列不需连续。

$LCS[x][y]= \left\{ \begin{array}{l} LCS[x-1][y-1]+1, A[x]=B[y] \\ max\{LCS[x][y-1], LCS[x-1][y]\}, A[X] \ne B[y] \end{array}  \right. $

\label{subsec:LCS}

\subsection{最长公共子字符串}
\cite{pp}15.5.9。与子字符\ref{subsec:LCS}的区别在于子串连续。将两个源字符串连接在一起,记录分隔点的位置,转换为求解最长重复短语的问题(\ref{problem:maxlenphrase}),区别在于扫描后缀数组时,如果当前后缀和下一个后缀出自相同的源字符串,则跳过。通过比较当前后缀对应的位置和分隔点的位置,可判断出该后缀出自哪个串。使用``异或''操作判断两个后缀是否出自同串。


\subsection{字符串相似度}
\cite{bop}3.3。相似度为最小的变换次数(记$\delta$, 单字符增、删、换)的倒数。
曾认为可先求LCS(\ref{subsec:LCS}),不妥。 简记A[x\ldots A.lenght-1]为A[x], 
\begin{displaymath}
   \delta(A[x], B[y])=
   \left\{
   \begin{array}{l}
       D[x+1][y+1], A[x]=B[y] \\ 
       min\{\delta[x][y+1], \delta[x][y+1], \delta[x+1][y+1] \}, A[y] \ne B[y]
   \end{array} 
    \right.
\end{displaymath}
\cite{bop}上程序写的大概是
\begin{math}
       min\{\delta[x+1][y+2], \delta[x+2][y+1], \delta[x+2][y+2] \}, A[y] \ne B[y]
\end{math}
,我认为不太正确。存在重复计算,可优化。

\subsection{包含所有关键词的最短摘要}
\cite{bop}3.5。枚举所有报文段($O(N^2)$),枚举的过程中检查是否包含所有M个关键词($O(N^{2} \cdot M)$)。作为优化,\cite{bop}提出只考虑以关键词为开端的报文段。本题以搜索引擎为背景,个人认为不适合空间复杂度较高的算法,因此,不适合预处理报文,比如标记关键词在报文中的位置。检查任意报文分词是否为关键字的复杂度为$O(M)$。


\subsection{字符串中只出现一次的字符}
\cite{ms100}17。基于ASCII字符值域有限的特性,建立一长度255的数组,保存每个字符出现的次数。


\subsection{比特字符串排序}
\cite{pp}11.5.5。可变长位字符串排序,x[0\ldots n-1]每项包含整数length和指向数组bit[0\dots length-1]的指针,设bit[i]返回第i个bit的值(0或1)。 首先将长度小于递归深度的字符串移动到最左,再利用快速排序的Lomuto划分(参\ref{summary:quicksort}),将剩余部分划分成两部分,分布递归,检查下一位。

\begin{lstlisting}[language=C]
void bsort(l,u,depth)
    if l >= u return
    for i in [l, u]
	if x[i].length < depth
	    swap(i, l++)
    m = l
    for i in [l, u]
	if x[i].bit[depth] &=&  0
	    swap(i, m++)
    
    bsort(l, m-1, depth+1)
    bsort(m, u, depth+1)
\end{lstlisting}

一开始用bsort(0,n-1,1)调用该函数。运行时间正比与待排序数据量,即各字符串长度总和。swap移动的是字符串指针而非字符串本身。 \cite{self}认为该代码有误,除非bit的索引从1而非零开始。

\subsection{字符串排序}
\cite{self}待补充。



\section{杂问题举例}

\subsection{装箱问题的首次适应算法}
\cite{pp}14.6.5。
问题:将n个权值在(0,1)之间的数字分配到最少数目的单位容量箱。首次适应启发式算法按序考虑权值,按升序扫描箱,装入第一个合适的箱中。将箱组织成堆状结构,叶子结点为每个箱的剩余容量,其他结点的值为其子树中最空的箱的剩余容量。在选箱时,如果左右子树都合适,则选左树。选箱时间为对数,插入权值后沿插入路径更新剩余容量。

\subsection{日期计算}
\cite{pp}3.7.4,计算日期差、某天是周几以及为某月生成日历。历法常识参\ref{subsec:leapyear}。
日期差:基本思想是转换为各日期与相应元旦的差值,加上各元旦之间的差值。首先分别计算两个日期在所在年之内的编号,用后者减去前者,再加上年份差的365倍,再为每个闰年(包括起年,不包括止年)加1。求周几:需要计算给定日期和已知周日之间的天数,再做模运算。计算日历:需要知道该月1号为周几以及该月有多少天。

\subsection{大空间初始化避免}
\cite{pp}1.6.9。大数组data[0\dots n-1]需要初始化(如data[i]='a'+i\%26),采取以空间换时间的即用即初始化策略回避对整个数组的初始化。开辟长度亦n的整数辅助数组history,eventNumber,变量top置0。检查data[i]是否已经初始化(访问过):\verb|if(0<=eventNumber[i]<top AND history[eventNumber[i]] is i)|。对data[i]的首次访问使用:\verb|eventNumber[i]=top;history[top]=i;init data[i];top++|,history数组记录初始化历史,依次指向各初始化事件的依此人。eventNumber[i]表示data[i]是第几例初始化(在history数组中的位置)。\label{problem:removeInit}

\subsection{区间[0,n-1]随机取出m个整数}

问题1:输出m个有序不重复整数(\cite{pp}12.5.8)。
方法1:候选空间遍历。(\cite{acp}3.4.2算法S,\cite{pp}12.2)\verb|s=m, r=n, for i in [0,n) { w.p.s/r{select i;s--;} r--;}|。该算法可保证如果r下降到同s相等则剩余候选数以概率1全部选中。时间同n成正比,当n很大时低效。$w.p.s/r$(with probability \verb|s/r|)实现为\verb|if (bigrand()%r < s)|。r和s分布代表select(剩余名额)和remaining(剩余候选集合)。
方法2:随机抽取,排序去重(\cite{pp}12.3)。可以使用C++ set模板类自动完成排序和去重,然后输出整个set。每次执行集合插入需要$O(logm)$,总时间$O(mlog m)$。如果使用其他数据结构\cite{self},要求方便地实现排序和搜索,如BST,有序数组,有序链表。而heap不适合搜索,很难实现去重。Bob Floyd给出了改进(\cite{pp}12.5.9),使得每次的抽样都不重复(第i次抽样空间为[0,i+n-m],且如果同以往重复则取门限值i+n-m,高于往次门限,必不重复)。这样,不要求数据结构本身具有方便搜索的功能,只要便于排序即可,可以使用堆。
方法3:候选集合洗牌,头部排序。(\cite{pp}12.3)。\verb|for i in [0,n) x[i]=i; for j in [0,m) swap(j, rand(j,n-1)); sort(x,0,m);return x[0..m-1]|。空间复杂度正比于n,n很大时难以接受。初始化时间正比于n(可以避免,参\ref{problem:removeInit}),洗牌时间正比于m,排序时间$O(mlogm)$。

问题2:输出m个随机序不重复整数(\cite{pp}1.6.4)。
问题1方法3,去掉排序环节;方法2,改为排序前输出,即每次取样值(该值首次)插入数据结构时即输出,而不是输出数据结构,可用使用Floyd改进。

问题3:输出m个有序可重复整数(\cite{pp}12.5.8)。
问题1方法2, 选择集合之外的便于排序的数据结构(heap, BST, 有序数组,有序链表,C++ multimap),抽取时不判断重复性,且不使用Floyd改进方案。

问题4:输出m个随机序可重复整数(\cite{pp}12.5.8)。
随机抽取,直接输出。问题3和4要求输出可重复整数,基于候选空间的算法不可用,只能基于抽取。








\section{排序算法}

\subsection{排序算法分类}
\label{subsec:sortAthmClass}

根据\cite{acp},排序算法包括内部排序和外部排序(不完全放入RAM)。内部排序分类:插入,交换,选择,合并,分布。
根据\cite{ita},排序算法包括基于比较的排序和非基于比较的排序(可以达到线性时间复杂度)。
\cite{weijipedia}给出了丰富的讲解材料。
\begin{verbatim}
http://en.wikipedia.org/wiki/Sorting_algorithm
\end{verbatim}

\begin{description}
    \item[基于插入的排序] 
        \hfill \\
	\begin{description}
	    \item[直接插入]就地稳定。$O(N^2)$。参\cite{ita}2.2。适用于近似有序数组。快速排序和插入排序可以联合使用,性能极佳。
	    \item[Shell排序]就地,不稳定。\cite{acp}5.2.1D。也称递减增量排序算法,不需要大量的辅助空间,不稳定,复杂度取决于降序序列的选取,可能是$O(Nlg^{2}(N)),O(NlogN), N^{\frac{6}{5}}$。
	    \item[图书馆排序]有n个元素待排序,这些元素被插入到拥有(1+e)n个元素的数组中。
	    \item[(二叉)树排序]稳定。线性空间复杂度。先构造二叉查找树,再in-order遍历之。从文件中读数时比较方便。平均时间$O(NlogN),二叉树不平衡时最差O(N^2)$。
	    \item[其他]二路插入(使用二分法查找合适度位置,但未能减小移动数据度开销),表插入(以链表实现,减小数据复制开销),地址计算插入等。
	\end{description}
    \item[基于交换的排序] 
        \hfill \\
	\begin{description}
	    \item[冒泡]稳定就地排序。$O(N^2)$。数据交换过多,\cite{acp}认为不值得推荐。
	    \item[鸡尾酒排序]稳定就地排序。冒泡排序的变种。别名:双向冒泡排序, 鸡尾酒搅拌排序, 搅拌排序,涟漪排序, 来回排序。
	    \item[快速排序]就地不稳定。平价时间$O(Nlog(N))$,空间$O(logN)$。\cite{sedgewick}认为快速排序是最快的。
	    \item[奇偶排序]奇偶换位排序,或砖排序。最初发明用于有本地互连的并行计算。
	    \item[Gnome Sort]又称stupid sort。类似插入排序,但通过同前面元素交换实现插入。
	    \item[梳子排序]冒泡排序的变种。不稳定。如同快速排序和合并排序,梳排序的效率在开始时最佳,接近结束时,即进入泡沫排序时最差。如果间距变得太小时(例如小于10),改用诸如插入排序或鸡尾酒排序等算法,则可提升整体效能。
	    \item[其他]合并交换(在可并行比较计算时有用),基数交换,地址计算插入等。
	\end{description}
    \item[基于选择的排序]
        \hfill \\
	\begin{description}
	    \item[直接选择]就地不稳定。$O(N^2)$。
	    \item[堆排序]就地不稳定。时间$O(Nlog(N))$,空间$O(1)$。\cite{ita}认为堆排序不如快速排序,但有其他用途,如实现优先级队列。	
	    \item[锦标赛排序]基于堆,待排数据初存于叶子节点,决出胜者,将胜者从其上升路径中删除,在存放胜者的叶子节点放置一个大数。时间$O(Nlog(N))$,需要额外空间作为堆。主要用于外排序的多路合并步骤。?

	\end{description}
    \item[基于合并的排序] 
        \hfill \\
	\textbf{归并排序},非就地,稳定。\cite{ita}2.3节Merge例程先将待合并数组的两个有序子数组拷贝到外部,再拷回原数组使有序。
	比较操作的次数介于$(nlogn)/2$和$nlogn-n+1$,赋值操作的次数是$2nlogn$,空间复杂度为:Θ (n)\cite{weijipedia}。
	\cite{acp}提出了如下两个非递归的归并排序算法,额外分配了与原数组等长的缓冲区。
	\begin{description}
	    \item[自然的两路合并]\cite{acp}5.2.4算法N。
	    \item[直接两路合并]\cite{acp}5.2.4算法S。
	\end{description}
	另外,还有就地不稳定版本的合并排序(\cite{acp}Vol3 习题5.2.5,$O(NlgN)$时间,流程复杂)。也有就地稳定版本(\cite{acp}Vol3习题)的合并排序,流程简单,时间复杂度有所增加,失去了意义。对于就地不稳定版本,分为分块、预排序、两两并归、扫尾四部。分为m+2块,除最后一块外,其他快大小都是$sqrt(n)$。块内排序。块排序,依据块的首元素,如首元素相同则依据末尾元素。然后执行多次AUX排序。\url{http://blog.ibread.net/345/in-place-merge-sort/}.
    \item[基于分布的排序] \hfill \\

	\begin{description}
	    \item[桶排序]\cite{ita}8.4。稳定?,最差时间$O(n+k)$,平均时间$O(n+k)$,需要额外空间存放桶。It is a distribution sort, and is a cousin of radix sort in the most to least significant digit flavour. Bucket sort is a generalization of pigeonhole sort. 
	    \item[鸽巢排序]鸽巢排序(Pigeonhole sort), 也被称作基数分类\cite{weijipedia}(count sort\cite{wikipedia}), 时空复杂度均$O(n+k)$,适用于$k=O(n)$情形,否则不如桶排序。计数排序中元素移动一次,鸽巢排序则移动两次(入、出鸽巢)\cite{wikipedia}.
	    \item[比较计数]\cite{acp}5.2算法C,统计比该记录小的记录的个数。时间$O(N^2)$,空间$O(N)$。无大用。
	    \item[分布计数]\cite{acp}5.2算法D。\cite{ita}8.2。稳定排序。时间$O(n+k)a$,空间$O(k+n)$,k为分布空间大小,需要额外空间存放辅助表和排序结果。若$k=O(n)$,则时空均为$O(n)$。涉及的操作包括prefix sum(cumulative sum).
	    \item[基数排序]\cite{ita}8.3。若每个基数位使用计数排序,则时间$O(d(n+k))$,d为位数,k为基数。
	\end{description}
\end{description}

另外,bogosort(猴子排序)通过随机洗牌来排序,时间复杂度无上限,仅供了解。一个笑话说:量子计算机能够以 O(n) 的复杂度更有效地实现Bogo排序。


\subsection{空间复杂度}
总之,如果不是就地排序,则至少需要$O(N)$的额外空间,如归并排序。快速排序为就地排序,但需要$O(logN)$的空间。
归并排序的主要缺点,是在最佳情况下需要Ω(n)额外的空间\cite{weijipedia}。

\subsection{稳定性}
\cite{weijipedia}
1.稳定的排序
\begin{enumerate}
	\item 冒泡排序, 鸡尾酒排序
	\item 插入排序
	\item 归并排序
	\item 二叉树排序
	\item 鸽巢排序
	\item 桶排序
	\item 计数排序
	\item 基数排序
	\item Gnome sort
	\item 图书馆排序
	\end{enumerate}

	2.不稳定的排序

	\begin{enumerate}
	\item 选择排序
	\item 希尔排序
	\item Comb sort
	\item 组合排序
	\item 堆排序
	\item Smoothsort
	\item 快速排序
	\item Introsort
	\item Patience sort
    \end{enumerate}

\cite{sedgewick}习题2.5.18提出了强制稳定性概念,比如用额外的字段记录每个数据的位置,在不稳定排序操作之后用此字段恢复原来的相对次序。
原地分区版本的快速排序算法是不稳定的。其他变种是可以通过牺牲性能和空间来维护稳定性的\cite{weijipedia}。

\subsection{快速排序}
\label{summary:quicksort}

有许多版本,包括就地非稳定版本和稳定非就地版本。
网上有一些非权威程序员给出了非递归版本,自行实现栈结构以模拟递归调用行为。

\cite{pp}和\cite{sedgewick}提出了以下几种划分方案:
\begin{itemize}
    \item 
	Lomuto划分, x[l]存放t,[l+1,m]小于t,(m,i)大于等于t,[i,u]未知。
    \item 
	双向划分, x[l]存放t,[l+1,m]小于等于t, (m,i)未知,[i,u]大于等于t。
    \item 
	三路划分(\cite{pp}11.5.11称为宽支点划分)。[l, lt)部分小于t,[lt,i)等于t,[i,gt]未知,(gt,u]大于t。
\end{itemize}


对快速排序pivot的选择,早期常使用最左元素,导致对有序数组性能很差。R.Sedgewick提出\cite{wikipedia}选择pivot的方案:
\begin{itemize}
    \item 随机元素
    \item 中间元素
    \item 最左、最右和中间元素的均值
\end{itemize}

他同时提出了两种优化:
\begin{itemize}
    \item 先递归较短的那半数组,以保证最多使用$O(logN)$空间。较长的那半数组使用尾部递归或遍历,可能不额外增加堆栈空间。
    \item 数组段较短时不再排序,最终使用插入排序扫一遍,插入排序对于近似排好的数组很高效。
\end{itemize}

代码示例参考\ref{codes:quicksort}。

内省(introsort)排序集成了快速排序和堆排序。

\subsection{堆排序}

C++标准库提供来堆操作,包括make\_heap, push\_heap, pop\_heap,sort\_heap等。可以用两行代码实现堆排序:
\begin{lstlisting}[language=Python]
    make_heap(a, a+n)
    sort_heap(a, a+n)
\end{lstlisting}

C++标准库也提供了对优先级队列priority\_queue的支持。

基于\cite{pp}14.2给出的两个关键函数siftup和siftdown,仅用几行程序就可实现堆排序。
\begin{lstlisting}[language=Python]
    #make heap
    for i in range(2, len(x)):
	siftup(x, i) #必须是大顶堆
    
    #sort heap
    for i in range(len(x)-1, 1, -1):
	tmp = x[1]; x[1] = x[i]; x[i] = tmp
	siftdown(x, i-1)
\end{lstlisting}
对\cite{pp}14.2节siftup的优化包括x[0]置哨兵,放置一个大数;x[i]上浮时,不交换i,p两个位置,即将swap展开,只置i,不置p。siftdown很难使用哨兵,但也可以通过展开swap来进行优化。


\subsection{外部排序}
主要算法\cite{wikipedia}:
\begin{itemize}
    \item 
        \textbf{外部合并排序}\cite{wikipedia}.涉及的数量概念是pass(取决于算法设计,不宜过多)和way(取决与数据量与RAM空间比值)。pass是步骤数,pass1将子文件排序并存放于way路临时文件,pass2进行多路合并。如果way过多,pass2会因磁盘寻道频繁而效率低下。此时可增加pass数,pass2合并way路为更少的路数,pass3将现有路数合并成1路。
    \item 
	\textbf{外部桶排序},基于分布的排序。桶排序和合并排序具有数学上的对偶性\cite{weijipedia}。k趟算法可以在kn的时间开销和\verb|n/k|空间开销内完成对最多n个小于n的无重复正整数的排序(选自\cite{pp}1.5答案,每趟使用位图法进行排序)。
\end{itemize}
\cite{pp}1.3的归并排序、多趟排序分别指这里的外部合并排序和外部桶排序。

此外,还有一些不需要临时文件的原地算法。 Merge sort is used in external sorting; heapsort is not. Locality of reference is the issue.



\subsection{整数排序}
\cite{wikipedia}
题目中研究的对象如果是整数,可以在其值域做文章。计算机中的整数取值空间有限且可数。
主要算法:
\begin{itemize}
    \item 
van Emde Boas tree,\cite{ita}第3版第20章,\cite{wikipedia}。
    \item 
non-conservative "packed sorting" algorithm
    \item 
	位图排序,要求key可表示为整数且互异。bit在bit序列中的位置代表整数的值,而bit的值代表存在性。如果整数出现多次\cite{pp}1.6.6,或者需要统计其他属性(而不仅仅是存在性),可以考虑用多个bit位表示每个整数,bit值代表该属性。如\cite{pp}1.6.8,美国免费电话号(800.887.888)存储问题,可以用3个bit分别表示以800,887,888前缀的特定7位号码是否已经存在\cite{self}。
    \item 
Bucket sort, counting sort, radix sort
\end{itemize}

下面是对位向量的实现\cite{pp}1.6.2:
\begin{verbatim}
#define BITSPERWORD 32
#define SHIFT 5
#define MASK 0x1F
#define N 10000000
int a[1 + N/BITSPERWORD];

void set(int i) {        a[i>>SHIFT] |=  (1<<(i & MASK)); }
void clr(int i) {        a[i>>SHIFT] &= ~(1<<(i & MASK)); }
int  test(int i){ return a[i>>SHIFT] &   (1<<(i & MASK)); }
\end{verbatim}

Bucket sort, counting sort, radix sort, and van Emde Boas tree sorting all work best when the key size is small; for large enough keys, they become slower than comparison sorting algorithms. 






\section{编程问题总结}

本章包含了一些算法题目,几乎所有题目都可以用枚举的方法低效地完成,记作方法零。遇到题目,可以依次尝试枚举、分治和动态规划。

\subsection{空间节省技术}
\ref{problem:sparseMatrix}给出了稀疏矩阵存储方式。对于其他空间节省技术,经常是通过仅存储差量来实现,如\cite{pp}10.6.4,10.6.5。

\subsection{循环不变式}
二分搜索:待搜索值在l\dots u之间的区间中。
堆siftup:heap(1,n) except perhaps between i and its parent。
堆siftdown:heap(1,n) except perhaps between i and its (0,1,2)children
快速排序partition:
子数组最大和:
BST搜索:

\subsection{利用哨兵减少判断}
哨兵有两个意思\cite{wikipedia},一是结构化表示结尾的最末特殊值(字符串,文件,链表),二是用于消除末尾判断的特殊值。本节取第二个意思。

求最值:\cite{pp}9.5.8利用哨兵元素找出数组最大值,x[n]时刻保存当前最大值。 需要在末尾多分配一个元素对空间,即可访问x[n]。同时,如果在开头额外分配元素,则可访问x[-1]。

二分搜索:\cite{pp}9.3修改二分搜索算法,将x[-1]和x[n]看作假想的哨兵值(参\ref{codes:binsort})。

顺序搜索:\cite{pp}9.2问题3将哨兵值用在了顺序顺序搜索中,x[n]保存待搜索值。

有序链表和BST插入:参\ref{problem:listInsert}和\ref{problem:BSTInsert}。

堆siftup:首元素作哨兵,取极小值。

快速排序Partition:

\subsection{克服malloc瓶颈}
\cite{pp}9.1,9.5.2,13.6.5。如果需要分配多个相同大小的结点,可以一次性malloc来节省时间,自行管理空间。

\subsection{if-else灾难与switch膨胀}
转为为搜索问题(\cite{pp}3.7.1),或利用多态(\cite{refractor})。

\subsection{分治法时间复杂度}
\begin{displaymath}
    T(n)=aT(n/b)+f(n)
\end{displaymath}

\begin{math}
    \textrm{其中} a \ge 1, b > 1, f(n) \textrm{is asymptotically positive}。
\end{math}

三种常见情况:
\begin{enumerate}
    \item 
\begin{displaymath}
    f(n)=O(n^{log_{b}a-\epsilon}), \epsilon>0, \textrm{则 }T(n)=\Theta(n^{log_{b}a})
\end{displaymath}

    \item 
\begin{displaymath}
    f(n)=O(n^{log_{b}a}lg^{k}n), \textrm{则 }T(n)=\Theta(n^{log_{b}a}lg^{k+1}n)
\end{displaymath}

    \item 
\begin{displaymath}
    f(n)=O(n^{log_{b}a+\epsilon}), \epsilon>0, \textrm{则}T(n)=\Theta(f(n))
\end{displaymath}

\end{enumerate}

\label{divide_complex}


\section{历法常识}
\subsection{闰年的计算}
\label{subsec:leapyear}
闰年(leap year)规则为: 西元年份逢4的倍数闰、100的倍数不闰、400的倍数闰。例如:公元1992、1996 年等为4的倍数,故为闰年;公元1800、1900、2100年为100的倍数,当年不闰;公元1600、2000、2400年为400的倍数,有闰。其他年份为平年(common year)。公元前之闰年出现在1, 5, 9, 13, 因此无法以“除以4”计算。公元元年为公元1年,没有零年。有人提议4000的倍数不闰,未被采纳。


\subsection{阳历和阴历}
阳历(又称太阳历,英语:Solar Calendar),为据地球围绕太阳公转轨道位置,或地球上所呈现出太阳直射点的周期性变化,所制定的历法。在华语文化中,“阳历”一词有时会被泛指为公历,以与传统的农历有所区别。

阴历(英语:lunar calendar)又称太阴历,在天文学中与阳历对应,指主要按月亮的月相周期来安排的历法。它的一年有12个朔望月,约354或355日。主要根据月亮绕地球运行一周时间为一个月(29.5306天),大月30日,小月29日。纯粹的阴历有希腊历和伊斯兰历,而大部分通常说的阴历实际上都是阴阳历,例如全世界所有华人及朝鲜、韩国和越南及明治维新前的日本都是使用的夏历。

阴阳历(英语:lunisolar calendar),在天文学中是指兼顾月相周期和太阳周年运动所安排的历法。一年有12个朔望月,过若干年(一般为十九岁一章)安置一个闰月,使年的平均值大约与回归年相当。夏历就是阴阳历的一种,具体的历法还包括纪年(纪元)的方法。中国、日本、朝鲜、及中东以色列的传统历法也是阴阳历,其他民族如藏族、傣族也是使用阴阳历。

农历又称夏历,是现今依旧广泛使用的中国传统历法,在古代一般称“黄历”或“皇历”,近代以来又称为夏历、阴历、旧历等。事实上,夏历不是阴历,而是阴阳历;夏历不是农历,阳历才是真正的农历;夏历不是旧历,而是现今依然广泛使用的历法,因此很多人主张为夏历正名。

在农业气象学中,阴历略微不同于农历、殷历、古历、旧历,是指中国传统上使用的夏历。而在天文学中认为夏历实际上是一种阴阳历。在古代农业经济中,春天播种、秋天收耕,本来阳历应更能反映农业周期,但不少古代历法都是由月亮算起,一个推测是黑夜中的月亮特别容易观察,月亮盈亏一目了然,直至天文技术成熟后,他们才能观察到太阳在历法中的作用。

\subsection{儒略历和阴历}
Julian calendar,儒略历,是格里历的前身,由罗马共和国独裁官儒略·恺撒采纳埃及亚历山大的希腊数学家兼天文学家索西琴尼计算的历法,在公元前45年1月1日起执行,取代旧罗马历法的一种历法。一年设12个月,大小月交替,四年一闰,平年365日,闰年于二月底增加一闰日,年平均长度为365.25日。由于累积误差随着时间越来越大,1582年后被教皇格里高利十三世改善,变为格里历,即沿用至今的西历。

现行公历即格里历,又译国瑞历、额我略历、格列高利历、格里高利历,是由意大利医生兼哲学家里利乌斯(Aloysius Lilius)改革儒略历制定的历法,由教皇格列高利十三世在1582年颁行。公历是阳历的一种,于1912年开始在中国正式采用,取代传统使用的中国历法农历,而中国传统历法是一种阴阳历,因而公历在中文中又称阳历、西历、新历。格里历与儒略历一样,格里历也是每四年在2月底置一闰日,但格里历特别规定,除非能被400整除,所有的世纪年(能被100整除)都不设闰日;如此,每四百年,格里历仅有97个闰年,比儒略历减少3个闰年。

格里历的历年平均长度为365.2425日,接近平均回归年的365.242199074日,即约每3300年误差一日,也更接近春分点回归年的365.24237日,即约每8000年误差一日;而儒略历的历年为365.25日,约每128年就误差一日。到1582年时,儒略历的春分日(3月21日)与地球公转到春分点的实际时间已相差10天。因此,格里历开始实行时,将儒略历1582年10月4日星期四的次日,为格里历1582年10月15日星期五,即有10天被删除,但原星期的周期保持不变。格里历的纪年沿用儒略历,自传统的耶稣诞生年开始,称为“公元”,亦称“西元”。

格里历1582年10月15日,合儒略历1582年10月5日,只有意大利、波兰、西班牙、葡萄牙开始用格里历,日期跳过10日。由于新历法是教皇颁布的,新教国家予以抵制。直到18世纪,大英帝国,包括英格兰、苏格兰、以及现在美国的一部份才采纳格里历,也就是儒略历 1752年 9月2日星期三的次日是格里历1752年9月14日星期四,日期跳过11日。值得注意的是,1582年,罗马教廷减去的是10天,而到1752年修改历法的时候却减去了11天的原因其实很简单,这涉及到了闰年的问题。

The proleptic Gregorian calendar is produced by extending the Gregorian calendar backward to dates preceding its official introduction in 1582.

\subsection{纪年}
纪年是人们给某一年起名的方法。主要的纪年有帝王纪年、公元纪年、岁星纪年和干支纪年等。
中国在汉武帝以前用帝王纪年,从即位年用元年、二年、三年……。改元时,用“中二年”“后元年”等。从汉武帝到清末,用年号纪年;1911年推翻帝制之后采用民国诞生时间来纪年,兼或使用公元纪年。1949年中华人民共和国成立以后采用全世界通用的公元纪年。干支纪年认为兴自东汉。东汉四分历以前,用岁星纪年和太岁纪年。到现在来用干支纪年。

西元,又称耶元、西历,一种源自于西方社会的纪年方法,以耶稣诞生年作为纪年的开始。在儒略历与格里高利历中,在耶稣诞生之后的日期,称为主后(拉丁语:Anno Domini,缩写为 A.D.),而在耶稣诞生之前,称为主前(英语:Before Christ,缩写为B.C.)。现代学者,为了淡化宗教色彩,多半改采用公元(Common era,缩写为C.E.)与公元前(Before the Common Era,缩写为 B.C.E.)的说法。

公历以耶稣的出生为参照。在6世纪时,东罗马帝国为了修订历法,以替代非常混乱的罗马历法,就请当时精通天文的僧侣建议一个更合理的纪年标准。由于自君士坦丁大帝以后,罗马帝国举国改信基督教,僧侣就决定改以耶稣出世的年份为新纪元一年。当时的僧侣就基于圣经上“耶稣被处决时约三十多岁”,就在耶稣处决那一年的年份减去三十,作为新纪元的元年。


\subsection{相关历史人物和事件}

盖乌斯·尤利乌斯·恺撒(拉丁文:Gaius Julius Caesar,前100年7月13日-前44年3月15日),罗马共和国末期的军事统帅、政治家,儒略家族成员。应该注意的是,中国大陆现在通常将此人译作尤利乌斯·凯撒,但在天文学上仍沿用“儒略历”一词,而不是“尤利乌斯历”。

郭守敬(1231年-1316年),字若思,邢台人,中国元朝的天文学家、数学家和水利学家。确定了一个月为29.530593日,一年为365.2425日。正式废除以前历法积累的时差,以实际观测为准。确定以一年的1/24作为一个节气,以没有中气的月份为闰月,此原则一直采用至今。

John Herschel proposed a correction to the Gregorian calendar, making years that are multiples of 4000 not leap years, thus reducing the average length of the calendar year from 365.2425 days to 365.24225. Although this is closer to the mean tropical year of 365.24219 days, his proposal has never been adopted because the Gregorian calendar is based on the mean time between vernal equinoxes (currently 365.2424 days).


前45年1月1日,恺撒起执行儒略历,取代旧罗马历法。
前44年,盖乌斯·尤利乌斯·恺撒遭到以布鲁图所领导的元老院成员暗杀身亡。
前32年,屋大维向安东尼宣战。奥古斯都,原名盖乌斯·屋大维·图里努斯(Gaius Octavius Thurinus),是罗马帝国的开国君主,统治罗马长达43年。
前23年,屋大维辞去执政官职,接受其他二职。一为保民官职(tribunicia potestas),二为“统治大权”。普遍认为奥古斯都在前23年里披上了黄袍。然而,他仍使用第一公民这个称号。
公元元年,王莽加爵“安汉公”。汉平帝追谥孔子为褒成宣尼公,封孔子后裔均为褒成侯,从此孔子正式登上中国传统社会的神坛。世界人口达2亿。
公元14年8月19日,奥古斯都逝世。罗马元老院决定将他列入“神”的行列,并且将8月称为“奥古斯都”月,这也是欧洲语言中8月的来源。














