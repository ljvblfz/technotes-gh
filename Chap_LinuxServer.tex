\section{网络流量测量}
vnstat, sar, slurm, ifstat, system-monitor等工具可查看网卡总流量。iptraf,iftop可查看连接的流量。

\subsection{简单测量}
最原始的办法,是连续两次使用date;ifconfig命令,计算一定时间间隔内的数据量。
也可以通过查看/proc/net/dev获取数据量。
在Gnome3下,可以使用一个叫做netspeed的gnome shell插件。

\subsection{vnstat工具}
\begin{shellcmd}
#-ru 0 使其以byte为单位,1使其以bit为单位.
vnstat -l -ru 0 #持续采样 
vnstat -tr #统计网速,5秒内的采样平均计算所得。
\end{shellcmd}

\subsection{iftop工具}
显示带宽使用情况。3列显示,分别表示过去2s,10s,40s内的统计带宽。
\begin{verbatim}
iftop -h | [-nNpbBP] [-i interface] [-f filter code] [-F net/mask]
\end{verbatim}
例如:
\begin{shellcmd}
#-B表示以byte而非bit为单位,-P显示端口号
sudo iftop -B -P 
\end{shellcmd}
工具默认自动将IP地址转换为主机名,会产生一定的DNS流量,扰乱测试。为讲其关闭,可使用-n命令。

\subsection{sar工具}
也可以使用sar工具.在Fedora下,sar工具位于sysstat软件包中.
\begin{shellcmd}
#最后的数字表示刷新时间间隔,单位为秒
sar -n DEV 3 
\end{shellcmd}

\subsection{ifstat工具}
\begin{shellcmd}
ifstat -a
\end{shellcmd}
经我验证,sar统计的字节数为以太网层,包括其头部和尾部,不包括前导码和帧间隔。

\subsection{ntop工具}
Ntop是一种监控网络流量工具,用ntop显示网络的使用情况比其他一些网络管理软件更加直观、详细。Ntop甚至可以列出每个节点计算机的网络带宽利用率。它是一个灵活的、功能齐全的,用来监控和解决局域网问题的工具;尤其当ntop与nprobe配合使用,其功能更加显著。它同时提供命令行输入和web页面,可应用于嵌入式web服务。跟 top 监视系统活动状况相似,ntop 是一个用来实时监视网络使用情况的工具。由于 ntop 具有 Web 界面模式,因此无论是配置还是使用都很容易在短时间之内快速上手。

\subsection{iptraf工具}
Interactive Colorful IP LAN Monitor。可查看每条连接的信息。
\begin{verbatim}
iptraf -i eth0
\end{verbatim}


\subsection{slurm工具}
 Simple Linux Utility for Resource Management,查看网络流量的一个工具。
 \begin{verbatim}
 slurm -i eth0
 \end{verbatim}

彩色curse节目,有部分文字是白色,在浅色背景下看不清楚。




\section{磁盘速率测量}
关于磁盘负载的生成器,主要有iometer和dd。

对于速率测量,在ubuntu下需要安装sysstat软件包,它提供了iostat工具。另外ubuntu下也有sar工具。
\section{资源消耗查询}
\begin{shellcmd}
 df -h:磁盘分区使用率
 du -h:文件夹大小
 tree -h:文件夹和文件大小,树形打印
 free -m:内存使用率
 vmstat -s M:虚拟内存占用情况
\end{shellcmd}
上述free命令中,m选项设置单位为MB。因为buffer和cache是否算作已用空间有争议,故分两种算法列出。
 上述vmstat命令,M表示单位为MB,如果是m,则为1000*1000B

如果只想查看当前目录下所有子目录的大小,有
\begin{shellcmd}
du -s
\end{shellcmd}
或者
\begin{shellcmd}
du -h --max-depth=1
\end{shellcmd}

对于top命令,最重要的两个交互快捷键是h和q,分别打印帮助信息和退出top。利用好h命令就能够进行各种复杂操作。例如,F可以选择一个field列,以其为标准进行行排序,如n表示MEM。可以查出Ubuntu占用内存最剧烈的是chromium-browser等浏览器。
 




\section{dstat工具}
iostat, vmstat, ifstat 三合一的工具,用来查看系统性能。
你可以这样使用:

\begin{verbatim}
alias dstat='dstat -cdlmnpsy'
\end{verbatim}


\section{环境变量}
Linux系统里的env命令可以显示当前用户的环境变量,还可以用来在指定环境变量下执行其他命令。下面来比较一下set,env和export命令的异同:set命令显示当前shell的变量,包括当前用户的变量;env命令显示当前用户的变量;export命令显示当前导出成用户变量的shell变量。每个shell有自己特有的变量(set)显示的变量,这个和用户变量是不同的,当前用户变量和你用什么shell无关,不管你用什么shell都在,比如HOME,SHELL等这些变量,但shell自己的变量不同shell是不同的,比如BASH\_ARGC,BASH等,这些变量只有set才会显示,是bash特有的,export不加参数的时候,显示哪些变量被导出成了用户变量,因为一个shell自己的变量可以通过export “导出”变成一个用户变量。

\section{文件查找工具}
locate, find, whereis, which等
\section{主机硬件与OS信息查询}
\subsection*{查询主机名称}
hostname命令可以查询主机名称。但是很多主机的名称都取作localhost
\subsection*{Linux内核版本}
\begin{shellcmd}
cat /proc/version:查询内核信息
uname -r:查询Linux内核版本号
uname -a:查询内核信息
\end{shellcmd}
注意不要用file命令通过查看可执行文件的信息来判断当前主机的内核版本号,二者关系复杂
\subsection*{Linux发行版信息}
\begin{shellcmd}
lsb_release -a 
cat /etc/issue
cat /etc/redhat-release
rpm -q redhat-release:查看Redhat release号
\end{shellcmd}
Redhat release号和实际的版本之间存在一定的对应关系:
\begin{verbatim}
   redhat-release-3AS-1 -> Redhat Enterprise Linux AS 3
   redhat-release-3AS-7.4 -> Redhat Enterprise Linux AS 3 Update 4
   redhat-release-4AS-2 -> Redhat Enterprise Linux AS 4
   redhat-release-4AS-2.4 -> Redhat Enterprise Linux AS 4 Update 1
   redhat-release-4AS-3 -> Redhat Enterprise Linux AS 4 Update 2
   redhat-release-4AS-4.1 -> Redhat Enterprise Linux AS 4 Update 3
   redhat-release-4AS-5.5 -> Redhat Enterprise Linux AS 4 Update 4 
\end{verbatim} 

\subsection*{查看字长:32/64位}
\begin{shellcmd}
getconf LONG_BIT
getconf WORD_BIT
file /bin/ls
lsb_release -a
\end{shellcmd}
\subsection*{CPU信息}
\subsubsection*{CPU架构}
arch命令或uname -m:结果如"i386", "i486","i586", "alpha", "sparc", "arm", "m68k", 
"mips", "ppc","ia64","x86\_64"等;ia64和x86\_64就说明这台机器是64位的
\subsubsection*{CPU详细信息}
\begin{shellcmd}
cat /proc/cpuinfo
\end{shellcmd}
\subsubsection*{CPU数与核数}
Linux下可以在/proc/cpuinfo中看到每个cpu的详细信息。但是对于双核的cpu,在cpuinfo中会看到两个cpu。常常会让人误以为是两个单核的cpu。
其实应该通过Physical Processor ID来区分单核和双核。而Physical Processor ID可以从cpuinfo或者dmesg中找到.
flags如果有ht说明支持超线程技术。判断物理CPU的个数可以查看physical id 的值,相同则为同一个物理CPU。
\begin{shellcmd}
 cat /proc/cpuinfo |grep "model name" && cat /proc/cpuinfo |grep "physical id"
\end{shellcmd}
\subsubsection*{CPU型号}
\begin{shellcmd}
cat /proc/cpuinfo |grep "model name"
\end{shellcmd}

\subsection*{内存信息}
\begin{shellcmd}
cat /proc/meminfo |grep MemTotal
sudo dmidecode |grep -A16 "Memory Device$"
\end{shellcmd}

\subsection*{硬盘大小}
\begin{shellcmd}
fdisk -l |grep Disk
\end{shellcmd}
df命令主要用来查询文件系统信息,可以用来查看硬盘以挂载的分区的大小
\begin{shellcmd}
df -h
\end{shellcmd}

\subsection*{主板型号}
\begin{shellcmd}
sudo dmidecode |grep -A16 "System Information$"
cat /proc/pci
\end{shellcmd}













\section{日志信息}


\subsection{查看用户连接信息}
连接时间日志一般由/var/log/wtmp和/var/run/utmp这两个文件记录,不过这两个文件都无法使用tail或cat命令直接查看。该文件由系统自动更新。Linux提供了如 w, who, finger, id, last, lastlog,ac等命令读取这部分的信息。
\begin{description}
    \item[finger]user information lookup program.用户详细信息,甚至包括电话号码。 
    \item[w]Show who is logged on and what they are doing.
    \item[last]show listing of last logged in users.\verb+last|grep reboot+记录了开机时间。 
    \item[lastlog] reports the most recent login of all users or of a given user
    \item[users]print the user names of users currently logged in to the current host. 其实就是第一列。
    \item[ac]ac应是accounting的缩写。打印用户连接时间的总和,指多个pty
    \item[id]当前用户的id和组id
    \item[uptime]查看开机时间,其实就是w输出的第一行
\end{description}

The utmp file allows one to discover information about who is currently using the system. 

The  wtmp  file records all logins and logouts.

\subsection{命令历史记录}


.bash\_history 记录了Bash命令历史,但不包含最近执行的命令。

/var/log/apt/目录下的文件记录了apt相关的命令执行历史。

\subsection{进程清算(accounting)}

acct是一个系统调用:

\begin{verbatim}
 #include <unistd.h>
 int acct(const char *filename);
\end{verbatim}

The  acct()  system  call  enables  or  disables process accounting.  If called with the name of an existing file as its argument,
accounting is turned on, and records for each terminating process are appended to filename as it terminates.  An argument of  NULL
causes accounting to be turned off.

例如:
\verb+ acct("/var/log/pacct");+

accton为系统命令,清算一段时间内终结的进程, 记录于清算文件中, 默认为/var/log/account/pacct。lastcomm用来读取清算文件。
\verb+ accton [OPTION] on|off|filename+

例如:
\begin{verbatim}
sudo accton somefile
sudo accton off
lastcomm -f somefile
\end{verbatim}

默认的pacct清算似乎早已默认开启,使得\verb+sudo accton on+命令显得多余。


\subsection{内核日志}
dmesg: 显示内核信息

\subsection{系统与服务日志}
由syslog的服务管理,比如下面的日志文件都是由syslog日志服务驱动的:      
/var/log/lastlog :记录最后一次用户成功登陆的时间、登陆IP等信息 

/var/log/messages :记录Linux操作系统常见的系统和服务错误信息 

/var/log/secure :Linux系统安全日志,记录用户和工作组变坏情况、用户登陆认证情况 

/var/log/btmp :记录Linux登陆失败的用户、时间以及远程IP地址 

/var/log/cron :记录crond计划任务服务执行情况

syslog服务由配置文件/etc/syslog.conf或/etc/rsyslog.conf.

/etc/syslog.conf的内容格式为:
\verb+消息类型.错误级别    动作域+

\subsection{日志转储}
Linux中使用logrotate命令进行日志转储。可以配合cron计划任务轻松实现日志文件的定时转储, /etc/logrotate.conf提供了日志转储的相关配置.
包括发送电子邮件。















\section{minicom串口配置}

\begin{verbatim}
创建或编辑配置文件,保存为/etc/minirc.Filename或/etc/minicom/minirc.Filename,因系统而异。
minicom -s Filename
\end{verbatim}
配置文件指定设备名、波特率等。Dell服务器的串口设备名为/dev/ttyS0, 4个USB接口的设备名为/dev/ttyUSB0-/dev/ttyUSB3。对于Hili的串口连接,要求码率为115200, 软、硬件流控制都设置为No。



\begin{verbatim}
按照指定配置文件,登录串口设备
minicom Filename
\end{verbatim}


minirc.dfl指定了默认配置,minicom命令不加任何参数时安装该文件进行初始化。

登录后ctrl+a z显示各种快捷键选项,ctrl+a x退出并reset,ctrl+a q退出不reset。


\section{RHEL网络配置}


是否启用网络、主机名称信息于/etc/sysconfig/network文件中配置。
\begin{verbatim}
NETWORKING=yes
HOSTNAME=localhost.localdomain //修改该值作为主机名,如:rhel.lpwr.net
\end{verbatim}

配置网卡IP等信息, 编辑指定网络接口配置文件
\begin{verbatim}
/etc/sysconfig/network-script/目录下建立形如ifcfg-eth0:0的文件进行配置

DEVICE=eth0 //指定接口名称
ONBOOT=yes //系统启动时加载
BOOTPROTO=static //IP地址静态配置,若该值为“dhcp”则为动态获得
IPADDR=192.168.0.1 //设置IP地址
NETMASK=255.255.255.0 //设置子网掩码
GATEWAY=192.168.0.254 //设置默认网关
\end{verbatim}

重启网络以执行配置更新:

\begin{verbatim}
/etc/init.d/network restart
\end{verbatim}

DNS信息由/etc/resolv.conf配置
\begin{verbatim}
127.0.0.1 localhost.localdomain localhost //该行强烈建议保留
192.168.0.1 rhel.lpwr.net rhel //必须有三个字段:IP、FQDN、HOSTNAME
\end{verbatim}


\section{网络连接查询netstat}

重要选项为:
a 显示所有的socket,而不仅仅是监听套接字(默认为l)
t TCP only
u UDP only
n 不将数字解析成名字,使用该选项能加速程序输出
p 显示相关进程的PID和名字









\section{网络连接故障原因}
\begin{itemize}
    \item 物理上是否连通
        \begin{itemize}
            \item 网线、电口模块是否受损
            \item 如果多条网线直连,是否两个端一一对应
            \item 如果走交换机,交换机是否正常
            \item 有时Hili某些网口不通,reset之后也不行。需要断电才可以。

        \end{itemize}
    \item 配置是否正确
        \begin{itemize}
            \item 是否属于同一子网
            \item IP是否冲突
            \item (Hili)MAC地址是否冲突
        \end{itemize}
    \item 软件是否正常
        曾遇到Dell服务器不能ping通时,将该网口用ifconfig重启后即可用。
        
\end{itemize}

另外,不能确定某服务器是否正常时,将两台相同的服务器连接到同一个交换机内,配置为同一个子网。Hili只能ping通其中一台,说明另一台有问题。
\section{分区创建与挂载}

\subsection{分区查看与创建}
分区创建可使用fdisk和cfdisk命令,fdisk的man页自称buggy,推荐使用cfdisk等。
df可以查看分区使用率。

\subsection{分区格式化}
使用mkfs工具。

\subsection{分区挂载}
例如挂载优盘sdc,执行:
\begin{verbatim}
sudo mount -t msdos -o uid=li,gid=li /dev/sdc1 /mnt
\end{verbatim}


\section{进程查看与控制}

按照NOHUP方式执行进程:
\begin{verbatim}
nohup ./cdn > /dev/null 2>&1 &
\end{verbatim}
noup:run a command immune to hangups, with output to a non-tty

\subsection{进程查找}
\begin{verbatim}
pgrep [-u user] pattern
如 pgrep -u root evince
pstree -p
-p选项会打印出PID的值,否则只打印名字
ps -ef
\end{verbatim}

\subsection{单一进程查看}
ps, top
\begin{verbatim}
ps pid
top [-p pid]
\end{verbatim}


\subsection{信号发送}
\begin{verbatim}
kill [信号] pid
pkill [信号] pattern
\end{verbatim}

\subsection{X下杀死进程}
xkill杀死单一窗口
如需重启,可按住Alt SysRq,再依次按REISUB



\section{RHEL配置yum}

\begin{enumerate}
    \item 如果没有安装yum,先安装yum
    \item /etc/yum.repos.d目录下,删除原有的repo文件,替换为Centos的源文件
    \item /etc/yum.conf文件,添加一行\verb+timeout=120+,未验证必要性
    \item 使用rpm命令import选项导入密钥文件
    \item 运行yum update命令
\end{enumerate}

关于密钥文件,目前不是很懂其原理。网上提到的有:
\begin{verbatim}
http://ftp.sjtu.edu.cn/centos/5/os/i386/RPM-GPG-KEY-CentOS-5
http://centos.ustc.edu.cn/centos/RPM-GPG-KEY-CentOS-5
\end{verbatim}
如果不导入密钥,那么安装可能不能成功。yum有nogpgcheck选项,以及配置文件也有相关选项,可能会强制安装软件。未测试。

安装软件包时需要知道这个包在yum系统中的名字。使用yum search命令。例如,需要安装git时,键入
\begin{verbatim}
yum search git
\end{verbatim}
易知git在yum系统中可能叫做git.i386.

\section{SmartBits仪表使用}
其测试客户端称为SmartWindow, 只能安装在Windows XP系统下,驱动程序选择7.x版本。需要输入序列号。

打开SmartWindow后,选择connection set up,设置连接地址,即仪表的IP地址。其默认IP地址为192.168.1.121。然后执行连接操作,可以看到仪表正面面板图。在面板图上选择连线了的子板,执行reserve操作,相当于选中。然后可以右键进行各种测试了。

发送数据时,需要确保对方的IP,MAC,PORT等设置正确。

统计速率,可用的子工具有smart counter,或右键某module选择display counter。统计的字节数为L2层,包括以太网头和尾(FCS),不包括帧前导和帧间隔,确定。

抓包:SmartWindow界面,右击某module模块选择capture。

每次更改设置时,smartbit可能会停止发包。可以查看光电转换模块的指示等判断smartbits是否仍然在发包。如果没有,需要重启smartbits。

用完后,smartbit和光电转换模块都需关闭电源,以防损害设备。




\section{SSH免密登录}
默认情况下ssh命令的每次执行都需要输入密码以进行验证。scp命令基于ssh,也需要密码。有两种方法能够避免在使用ssh和scp时输入密码。

\subsection{sshpass}
sshpass工具可用于保存ssh密码。在ssh和scp命令前添加\verb+sshpass -p <password>+即可。

\subsection{服务器端保存客户机公钥}
首先,客户端需要生成自身的公钥和私钥,使用ssh-keygen命令,公私钥文件在生成后保存到客户机~/.ssh目录下。
\begin{verbatim}
ssh-keygen -t rsa
\end{verbatim}

其次,将公钥文件追加到服务器主机\verb+~/.ssh/authorized_keys+文件结尾。该文件模式需为600,.ssh目录需为700。
\begin{verbatim}
client> scp <pubkey_file> root@server_host:~
server> cat <pubkey_file> >> /root/.ssh/authorized_keys
server> chmod 700 /root
server> chmod 700 /root/.ssh
server> chmod 600 /root/.ssh/authorized_keys
\end{verbatim}
经实验,root家目录访问权限也可以为750, .ssh权限也可以为755, authorized\_keys权限也可为644。
Ubuntu上传客户端公钥的方法也可以是:
\begin{verbatim}
ssh-copy-id -i ~/.ssh/id_rsa.pub root@serverhost
\end{verbatim}


然后,按正常方式进行ssh登录。此时输入密码。以后则无需输入密码了。

\section{sshd服务启动}
在Ubuntu上,临时启动、停止和查看一个服务可以执行:
\begin{verbatim}
service ssh start/stop/status
/etc/init.d/ssh start/stop/status
\end{verbatim}

如果希望开机时不自动启动sshd,只有需要用时才开启sshd,目前找到的办法是:破坏\verb+/etc/init/ssh.conf+文件,如将其重命名,或将exec语言破坏。
这样,sshd将脱离service命令的管理。需要临时开启时,执行
\begin{verbatim}
/usr/sbin/sshd 或者
/usr/sbin/sshd -D
\end{verbatim}

如果遇到一个错误:
\begin{verbatim}
Missing privilege separation directory: /var/run/sshd
\end{verbatim}

只需执行:
\begin{verbatim}
mkdir -p /var/run/sshd
\end{verbatim}




\section{Ubuntu系统服务}

Ubuntu使用service命令来临时启动和停止服务。service是一个脚本,调用start,stop等可执行文件。

Ubuntu上可以使用工具update-rc.d或sysv-rc-conf来管理服务,但经尝试似乎效果不理想。


\section{RHEL系统服务}
service命令运行\verb+/etc/init.d+目录下的System V脚本或\verb+/etc/init+目录下的upstart jobs。这些脚本应至少支持start, stop参数,有些支持status, restart参数。

service runs a System V init script or upstart job in as predictable environment as possible, removing most environment variables and with current working directory set to \verb+/+.  

重启网络:
\begin{verbatim}
service network restart
\end{verbatim}

查看所有服务的状态
\begin{verbatim}
service --status-all
将所有支持status命令的服务运行一遍
\end{verbatim}

\section{Tcpdump用法举例}

\begin{verbatim}
tcpdump -i eth0 host 192.168.130.10 'tcp' -w file.pcap
tcpdump -r file.pcap >> file.txt
\end{verbatim}
注意,待写入的文件必须以pcap为后缀,否则tcpdump不能运行,报权限错误。

c选项指定抓包数量,s选项指定抓包长度(以太网层),比较实用。注意如果s选项指定的长度超过帧长,则只抓帧长。所谓帧长,如果是在Linux下运行pcap程序,是不包括L2 FCS的(以太网尾部CRC),这和Smartbit等测试仪表不一样。Smartbit等抓到FCS并统计到计数器中。sar也将FCS加入到计数器中,虽然Linux下的pcap抓不到FCS。Wireshark会判断操作系统只否能抓到FCS,但声称判断未必准确,可以手工告诉Wireshark一定有FCS。

上述pcap文件也可以用wireshark打开。


 To print the start and end packets (the SYN and FIN packets) of each TCP conversation that involves a non-local host.
\begin{verbatim}
tcpdump 'tcp[tcpflags] & (tcp-syn|tcp-fin) != 0 and not src and dst net localnet'
\end{verbatim}

待补充
\section{tftp服务器端安装}
Ubuntu下,安装tftp-hpa。在/etc/default/tftp-hpa目录下配置。
RHEL下,安装tftp-server。/etc/xinetd.d/tftp配置。守护进程名称为in.tftpd。
\section{用户与组}

\subsection{用户增加与删除}
useradd, adduser, userdel, deluser, usermod, delgroup

使用ueradd时,如果后面不添加任何参数选项,例如:\verb+#sudo useradd test+创建出来的用户将是默认“三无”用户:一无Home Directory,二无密码,三无系统Shell。
使用adduser时,创建用户的过程更像是一种人机对话,系统会提示你输入各种信息,然后会根据这些信息帮你创建新用户。
\begin{verbatim}
useradd -m -s /bin/bash user 
-m表示创建家目录,默认不创建。
注意-p选项以密文方式设置密码,似乎没什么用。
userdel -r user
删除用户及其家目录
\end{verbatim}

\subsection{设置用户为sudoer}
执行visudo命令,在打开的文件中添加一行
\begin{verbatim}
userA ALL=(ALL)  ALL
userB ALL=(ALL)  NOPASSWD: ALL
\end{verbatim}
visudo相当于修改/etc/sudoers文件。userB执行sudo的时候不需要输入密码。

可以在bashrc或profile中添加如下内容,修改PATH
\begin{verbatim}
PATH=/usr/sbin:/usr/local/sbin:/sbin:$PATH
export PATH
\end{verbatim}

\subsection{查看用户所在组}
方法一:id命令
方法二:/etc/group,格式:
\begin{verbatim}
group_name:password('x'):GID:user_list(separated by commas)
\end{verbatim}

/etc/passwd包含了家目录,shell等用户信息

\subsection{用户的shell}
查看系统安装的shell
\begin{verbatim}
cat /etc/shells
\end{verbatim}

查看当前shell
\begin{verbatim}
echo $SHELL
\end{verbatim}

更改某用户的shell
\begin{verbatim}
usermod -s /bin/zsh someuser
\end{verbatim}


\subsection{修改用户名和主机名}
用户名在/etc/shadow中修改,主机名需要在/etc目录下hostname和hosts两个文件下修改

\subsection{用户密码}
只能使用passwd命令。在创建用户后创建密码。
