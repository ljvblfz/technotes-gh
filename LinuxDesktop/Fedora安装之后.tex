\newpage
\section{Fedora16安装之后}
\subsection{配置网络}
使用图形界面配置即可。注意netmask为24,不是255.255.255.0
\subsection{配置时间和地区}
应该设置地区为上海,这样在安装很多应用程序到时候会下载中文语言
\subsection{配置更新源}
\subsubsection{手动选择源}
开源镜像网站http://mirrors.163.com/和http://mirrors.sohu.com/下载fedora的源配置文件。为了区别镜像,打开下载的镜像文件后把三个中括号[]中的fedora分别替换为fedora-163与fedora-sohu。将修改好的配置文件保存到/etc/yum.repos.d/目录下。最后在终端输入:
\begin{shellcmd}
sudo yum makecache
\end{shellcmd}

\subsubsection{安装fastest-mirror}
\begin{shellcmd}
sudo yum -y install yum-plugin-fastestmirror 
\end{shellcmd}
\subsubsection{添加rpmfusion源}
\begin{shellcmd}
sudo rpm -ivh http://download1.rpmfusion.org/free/fedora/rpmfusion-free-release-stable.noarch.rpm 
http://download1.rpmfusion.org/nonfree/fedora/rpmfusion-nonfree-release-stable.noarch.rpm
\end{shellcmd}

\subsection{升级系统}
\begin{shellcmd}
sudo yum update
\end{shellcmd}

\subsection{配置字体}
安装文泉驿字体
\begin{shellcmd}
sudo yum -y install wqy-bitmap-fonts wqy-zenhei-fonts wqy-unibit-fonts
\end{shellcmd}
安装Windows字体
\begin{itemize}
\item 将windows字体拷贝到/usr/share/fonts/某目录/下
\item chmod 755 *
\item mkfontscale;mkfontdir;fc-cache -fv
\end{itemize}

\subsection{安装系统工具}
\subsubsection{安装gnome-tweak-tool}
\begin{shellcmd}
sudo yum -y install gnome-tweak-tool
\end{shellcmd}
\subsubsection{安装gnome shell extension}
访问https://extensions.gnome.org/
\subsubsection{其他工具}
\begin{itemize}
\item xkill.需安装xkill或xorg-x11-apps
\item nautilus-open-terminal.安装完成后ctr+alt+backspace重启X
\item faenza-icon-theme.
sudo yum -y install faenza-icon-theme
\end{itemize}

\subsection{安装常用软件}
\subsubsection{安装google-chrome}
google并没有直接提供yum源,而是以sh文件的方式提供。那么就下载这个文件
\begin{shellcmd}
wget https://dl-ssl.google.com/linux/google-repo-setup.sh

sudo sh google-repo-setup.sh 

sudo yum -y install google-chrome-stable

rm google-repo-setup.sh 
\end{shellcmd}
\subsubsection{安装latex和xelatex}
\begin{shellcmd}
sudo yum -y install texlive texlive-xetex

fmtutil --enablefmt xelatex

sudo yum -y install texmaker
\end{shellcmd}
\subsubsection{安装多媒体解码器}
fedora默认没有安装视频解码器,所以不能听歌看视频,打开歌曲时会提示缺少MPEG-1 Layer3。
首先确保系统已经安装rpmfusion源,在终端中输入命令:
\begin{shellcmd}
sudo yum -y install ffmpeg ffmpeg-libs gstreamer-ffmpeg \
libmatrosca xvidcore libdvdread libdvdnav lsdvd
sudo yum -y install gstreamer-plugins-good \ 
gstreamer-plugins-bad gstreamer-plugins-ugly
\end{shellcmd}

\subsubsection{安装Office}
安装LibreOffice:
\begin{shellcmd}
sudo yum -y groupinstall "Office/Productivity"
sudo yum -y install libreoffice-langpack-zh-Hans
\end{shellcmd}
永中Office可以从官网下载

\subsubsection{Vim及其插件}
安装gvim
\begin{shellcmd}
sudo yum -y install gvim
\end{shellcmd}
安装vim-latex
\begin{shellcmd}
sudo yum -y install vim-latex
\end{shellcmd}
所谓vim-addon-manager有两个意思,一个指debian下的软件,一个是vim插件,这里指后者.从官网下载该插件后,解压,然后配置.vimrc文件指定vim-addon-manager路径和想安装的插件的名称.
例如:
\begin{shellcmd}
set runtimepath+=/PATH/TO/VIM-ADDON-MANAGER
call vam#ActivateAddons([``vim-haxe'',``snipmate''])
call vam#ActivateAddons([``OmniCppComplete''])
call vam#ActivateAddons([``The_NERD_Commenter''])
\end{shellcmd}
下一次打开vim的时候会自动提示安装相应插件.如果插件名称有微小的错误(typo),可能会得到正确提示.

\subsubsection{KVM虚拟机}
安装KVM虚拟机
\begin{shellcmd}
sudo yum -y install kvm qemu libvirt virt-manager
\end{shellcmd}
可以利用virt-manager安装Windows XP系统,然后安装360安全卫士,搜狗拼音,360浏览器,360压缩等.注意yunio不支持ie系统的浏览器,所以可以再安装一个chrome浏览器.

\subsubsection{其他}
mendeley,yozo office,fcitx输入法,unrar解压软件,qt-creator(在ubuntu下叫做qtcreator),AdobeReader\_chs,antiword

\subsection{卸载不需要的软件}
ibus等

\subsection{修改默认应用程序}
在Fedora下,有两个配置文件:
/usr/share/applications/defauts.list \\
/usr/local/share/applications/defauts.list
其关系不明

\subsection{配置硬盘自动挂载}
修改/etc/fstab

\subsection{关闭SELinux}
修改/etc/selinux/config



\newpage




