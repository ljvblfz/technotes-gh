\section{Minicom}

\subsection{Minicom配置}

创建或编辑配置文件,保存为/etc/minirc.Filename或/etc/minicom/minirc.Filename,因系统而异。
\begin{lstlisting}[language=bash]
minicom -s Filename
\end{lstlisting}
配置文件指定设备名、波特率等。Dell服务器的串口设备名为/dev/ttyS0, 4个USB接口的设备名为/dev/ttyUSB0-/dev/ttyUSB3。对于Hili的串口连接,要求码率为115200, 软、硬件流控制都设置为No。

配置文件格式如下:
\begin{lstlisting}
    # Machine-generated file - use ``minicom -s'' to change parameters. 
     pr port             /dev/ttyUSB0 
     pu baudrate         115200 
     pu bits             8 
     pu parity           N 
     pu stopbits         1 
     pu rtscts           No
\end{lstlisting}

\subsection{Minicom登录}
按照指定配置文件,登录串口设备:
\begin{lstlisting}[language=bash]
minicom Filename
\end{lstlisting}

minirc.dfl指定了默认配置,minicom命令不加任何参数时安装该文件进行初始化。

登录后ctrl+a z显示各种快捷键选项,ctrl+a x退出并reset,ctrl+a q退出不reset。

ssh远程登录并登录minicom,执行:
\begin{lstlisting}[language=bash]
ssh -t HOSTNAME minicom Filename
\end{lstlisting}
\label{sshMinicom}

