\section{printf格式化输出}
格式占位符的格式为:
\begin{lstlisting}
    %[parameter][flags][width][.precision][length]type
\end{lstlisting}

\begin{description}
    \item[paramater]
	\hfill \\
	指\verb|n$|记法,代表参数列表中第n个参数,n的范围从1开始,至参数数量,如n越界则后果不可知。
	如\lstinline!printf(``%2$d %2$#x; %1$d %1$#x'',16,17)!输出结果``17 0x11; 16 0x10''。
    \item[flags]
	\hfill \\
	\begin{description}
	    \item[+]正数须前缀加号
	    \item[-]宽度填充使用左对齐
	    \item[\#]浮点数须带小数点,标准记数法后缀0须保留,八进制、十六进制数须前缀0,0x或0X。
	    \item[0]宽度填充浮使用0而非空格。\lstinline!printf(``%05d'',3)!输出00003。
	\end{description}
    \item[width]
	\hfill \\
	最小宽度,不会截断过长的域。默认右对齐。宽度值可以用*代替,自动替换为下一个参数,如\lstinline!printf(``%*c\n'', 5, 'a');!显示为“    a”。	
    \item[precision]
	\hfill \\
	浮点数的小数位数,字符串的显示长度,后面的被截断。具体值可以用*代替,自动替换为下一个参数,如\lstinline!printf(``%.*s\n'', 3, ``abcedf'');!显示为“abc”。	
    \item[length]
	\hfill 
	\begin{description}
	    \item[hh] 期待char类型。
	    \item[h] 期待short类型。
	    \item[l] 期待long类型。
	    \item[ll] 期待long long类型。
	    \item[L] 期待long double类型。
	    \item[z] 期待size\_t类型。
	\end{description}
    \item[type]
	\hfill \\
	\begin{description}
	    \item[d,i]显示为十进制整型。 
	    \item[f,F]显示为double类型的浮点表示。F将nan,inf显示为NAN,INF等。 
	    \item[e,E]显示为double类型的标准记数法表示。E将记法中的e显示为E。
	    \item[g,G]显示形式在浮点记法和标准记法之间自动选择。
	    \item[p]期待void*类型的指针,显示形式取决于实现。
	\end{description}

\end{description}

特别地,\%m不对应参数,相当于strerror(errno),见\cite{apue2}P347。\%n表示迄今已输出的字节数,写入指针所指的无符号整型变量中。









