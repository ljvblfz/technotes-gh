\section{软件界面的绿豆沙色背景}
关于绿豆沙色的定义网上找到了多种:
1.RGB(204,232,207),RGB hex(CCE8CF)。

2.RGB hex(CCE8CF)

3.RGB(199,237,204),HSL(85,123,205)

4.HSL(84,91,205)

5.HSL(85,90,205),RGB(187,247,197)

网上有人推荐第5种.

能设置背景颜色的软件包括gnome-terminal, chromium浏览器,Adobe reader, okular等。


Adobe:Edit->Preferences->Accessibility(这里一般译作"辅助功能")->Document Color Options

okular:setting->configure okular->accessibility0->color mode

evince不能设置任意背景色,只能反色为黑背景。其他Linux上的pdf阅读器,如xournal,永中pdf阅读器等,不能设置不背景色。wine过的cajviewer,foxit reader能够设置背景色。



Chromium:
修改配置文件:
~/.config/chromium/Default/User StyleSheets/Custom.css
添加:
\begin{shellcmd}
html, body {background-color: #CCE8CC!important;}
\end{shellcmd}



关于HSL,包括:
H:Hue,色相,色调
S:Sat,Saturation,饱和度
L:Lum,Lightness,也有说Intensity,亮度

也有HSB(HSV)色彩空间,
V(Value)B(Brightness)译作明度
Adobe的软件使用HSV而非HSL
