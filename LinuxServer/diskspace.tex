\section{资源消耗查询}
\begin{shellcmd}
 df -h:磁盘分区使用率
 du -h:文件夹大小
 tree -h:文件夹和文件大小,树形打印
 free -m:内存使用率
 vmstat -s M:虚拟内存占用情况
\end{shellcmd}
上述free命令中,m选项设置单位为MB。因为buffer和cache是否算作已用空间有争议,故分两种算法列出。
 上述vmstat命令,M表示单位为MB,如果是m,则为1000*1000B

如果只想查看当前目录下所有子目录的大小,有
\begin{shellcmd}
du -s
\end{shellcmd}
或者
\begin{shellcmd}
du -h --max-depth=1
\end{shellcmd}

对于top命令,最重要的两个交互快捷键是h和q,分别打印帮助信息和退出top。利用好h命令就能够进行各种复杂操作。例如,F可以选择一个field列,以其为标准进行行排序,如n表示MEM。可以查出Ubuntu占用内存最剧烈的是chromium-browser等浏览器。
 




