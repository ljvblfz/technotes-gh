\section{字号}
px:pixel,像素,屏幕上显示的最小单位;1px没有绝对的大小,受分辨率影响。
pt:point,点,是印刷业一个标准的长度单位,1pt=1/72 inch = 0.35146mm;1 inch =2.54cm;
dpi:dots per inch,意思是指每一英吋(1inch=2.54cm)长度中,取样或可显示或输出点的数目。打印机所设定之分辨率的 DPI 值越高,印出的图像会越精细。打印机通常可以调校分辨率。例如撞针打印机,分辨率通常是60至90DPI。喷墨打印机则可达1200DPI,甚至9600DPI。激光打印机则有600至1200DPI。一般显示器为 72 dpi,印刷所需位图的 DPI 数则视印刷网线数而定。一般 150 线印刷品质需要300dpi的位图。
ppi:pixels per inch.


点数制又叫磅数制,是英文point的音译,缩写为P,既不是公制也不是英制,是印刷中专用的尺度。
中国大都使用英美点数制。
1点(1P)=0.35146mm

号数制是以互不成倍数的几种活字为标准,加倍或减半自成体系。
? 四号序列:一号、四号、小六号
? 五号序列:初号、二号、五号、七号
? 小五号序列:小初号、小二号、小五号、八号
? 六号序列:三号、六号

\begin{table}
	\centering
	\begin{tabular}{ccc}

号数&            点数&                 尺寸(mm)\\
(无名)&          72&                 25.305\\
大特号&          63&                 22.142\\
特号&            54&                 18.979\\
初号&            42&                 14.761\\
小初号&          36&                 12.653\\
大一号&          31.5&                 11.071\\
一(头)号&      28&                 9.841\\
二号&            21&                 7.381\\
小二号&          18&                 6.326\\
三号&            16&                 5.623\\
四号&            14&                 4.920\\
小四号&          12&                 4.218\\
五号&            10.5&                 3.690\\
小五号&          9&                 3.163\\
六号&            8&                 2.812\\
小六号&          6.875&                 2.416\\
七号&            5.25&                 1.845\\
八号&            4.5&                 1.581\\
	\end{tabular}
	\caption{号数与点数}
\end{table}

从上表中可以看出,常用的MS-WORD软件中字号的大小与印刷业中字号的大小是不一致的。如MS-WORD中的二号字是22磅,但在印刷业中应该是21磅。







