\section{PDF阅读、编辑和转换}

\subsection{PDF阅读与批注}
acroread软件包包含了Adobe Reader, 兼容性好,可以调节背景色。但功能相对于Acrobat十分受限, 没有批注、书签等功能,运行效率也低下。

okular可以实现pdf的批注(F6或tools->review)和背景色改变。其缺点是,如果在gnome环境下安装需要下载几十MB的包。

mendeley可以实现批注,但批注只是内部识别。不能调节背景色。

xournal添加文字注释,下划线,通过export功能保存修改结果;但是Xournal对于篇幅较长的PDF文档太耗尽内存,因为它会将PDF文档中所有的页面都转化为图像数据置于内存之中并且不再释放。我们可以首先对文档进行分割。据说只要别大范围拖动滚动条,内存占用便不大。

evince,Foxit4Linux,永中阅读器既不能标记,又不能改变背景色。

综上,对于pdf阅读,较好的有okular,wine上的foxit,cajviewer。

pdfgrep(CLI)可以从pdf中查找正则表达式,用法类似于grep。diffpdf可以比较两个pdf文件的不同。

\subsection{PDF书签编辑}
pdfmod是目前发现的唯一一个制作书签的开源工具,但编辑不便,无法调整书签的显示顺序。Windows下的Foxit Reader可以制作pdf书签,linux版本的则不可。evince制作的书签似乎只是内部识别。因此,在Linux下最好wine一个Windows版的Foxit Reader。

\subsection{PDF页面级编辑}

pdfshuffler可以合并、分裂、排序页面,在precise pangolion尝试出错,显示没有EOF标记。

pdftk为CLI工具,功能包括合并,分裂,删页,反序,旋转,加解密,其中合并pdf的方式包括连接和互插。pdfchain是pdftk的一个GUI前端。

pdfmod可以用于删页、插入其他文件、导出页面、修改各页相对顺序(通过鼠标拖动,有时比pdftk方便)、编辑索引(书签),修改文件属性。

\subsection{PDF元素级编辑}
pdfedit添加文字注释(英文),划线,删除文档元素,删加页面;感觉不太稳定,运行十分缓慢,经常在打开文件时内核转储。

可以使用inscape或gimp提取PDF中的一页,进行复杂的修改,然后使用pdf删除合并工具恢复成完整的PDF文件。

openoffice.org-pdfimport包让LibreOffice能够直接导入PDF文件,进行文字修改,再保存为odg图形文件,或者选择导出为pdf。但导入PDF时文件内的图片常常会丢失,排版可能会被破坏。所以LibreOffice目前还不能算作PDF阅读器或者编辑器。

flpsed可以添加英文文字,但可能会损坏PDF,使其不能被其他阅读器打开。

pdfstudio功能强大,但系付费软件。

\subsection{PDF元数据}
pdfmod可以修改文件元数据,如作者、标题等。pdfinfo命令行工具可以显示元数据。
pdffonts显示文件字体信息。

\subsection{PDF格式转换}
ImageMagick可以实现pdf和图片的相互转换, 使用convert命令。
\begin{verbatim}
 convert [input-options] input-file [output-options] output-file
\end{verbatim}

\begin{verbatim}
convert -density 700 -quality 100 draft.pdf draft.jpg
\end{verbatim}
详细内容参\ref{sec:imagemagick}。

cups-pdf用于将其他格式的文件如图片打印为pdf。cups-pdf打印保存位置由/etc/cups/cups-pdf.conf文件配置,一般为~/PDF,或者/var/spool/cups-pdf。

pdftotext实现pdf到文本的转换,效果一般不理想。

gpdftext是一款编辑器,可以直接导入pdf文件进行文字编辑,再保存成文本或pdf格式,但是会丢失所有除文字内容之外的信息,包括格式、排版、分页信息。

pdftohtml实现pdf到html的转换,效果往往不理想。可能需要指定编码格式,如
\verb+pdftohtml -enc GBK haha.pdf +

pdfimages提取pdf中的图片,默认保存为ppm格式。
\verb+pdfimages -j haha.pdf+。
j选项指定保存为jpg格式。

pdf2ps和ps2pdf实现pdf和ps之间的转换,基于ghostscript机制。当前ps2pdf默认使用ps2pdf14,即pdf为1.4版本。可以直接使用ps2pdf15.pdftops也能实现pdf到ps之间的转换。

pdf2djvu,pdf2dsc,pdftoppm, pdf2svg分别实现pdf到djvu,dsc,ppm, svg的转换。SVG可缩放矢量图形(Scalable Vector Graphics)是基于  svg logo可扩展标记语言(XML),用于描述二维矢量图形的一种图形格式。

chm2pdf可以将chm转换为pdf。



