\section{主机信息查询与修改}

\subsection{主机名称查询与修改}
hostname命令可以查询主机名称。但是很多主机的名称都取作localhost。
Ubuntu主机名需要在/etc目录下hostname和hosts两个文件下修改,Centos6.4主机名通过/etc/sysconfig/network文件修改。

\begin{lstlisting}
	HOSTNAME=vss-dev1   
\end{lstlisting}


\subsection{Linux内核版本}
\begin{shellcmd}
cat /proc/version:查询内核信息
uname -r:查询Linux内核版本号
uname -a:查询内核信息
\end{shellcmd}
注意不要用file命令通过查看可执行文件的信息来判断当前主机的内核版本号,二者关系复杂
\subsection*{Linux发行版信息}
\begin{shellcmd}
lsb_release -a 
cat /etc/issue
cat /etc/redhat-release
rpm -q redhat-release:查看Redhat release号
\end{shellcmd}
Redhat release号和实际的版本之间存在一定的对应关系:
\begin{verbatim}
   redhat-release-3AS-1 -> Redhat Enterprise Linux AS 3
   redhat-release-3AS-7.4 -> Redhat Enterprise Linux AS 3 Update 4
   redhat-release-4AS-2 -> Redhat Enterprise Linux AS 4
   redhat-release-4AS-2.4 -> Redhat Enterprise Linux AS 4 Update 1
   redhat-release-4AS-3 -> Redhat Enterprise Linux AS 4 Update 2
   redhat-release-4AS-4.1 -> Redhat Enterprise Linux AS 4 Update 3
   redhat-release-4AS-5.5 -> Redhat Enterprise Linux AS 4 Update 4 
\end{verbatim} 

\subsection*{查看字长:32/64位}
\begin{shellcmd}
getconf LONG_BIT
getconf WORD_BIT
file /bin/ls
lsb_release -a
\end{shellcmd}
\subsection*{CPU信息}
\subsubsection*{CPU架构}
arch命令或uname -m:结果如"i386", "i486","i586", "alpha", "sparc", "arm", "m68k", 
"mips", "ppc","ia64","x86\_64"等;ia64和x86\_64就说明这台机器是64位的
\subsubsection*{CPU详细信息}
\begin{shellcmd}
cat /proc/cpuinfo
\end{shellcmd}
\subsubsection*{CPU数与核数}
Linux下可以在/proc/cpuinfo中看到每个cpu的详细信息。但是对于双核的cpu,在cpuinfo中会看到两个cpu。常常会让人误以为是两个单核的cpu。
其实应该通过Physical Processor ID来区分单核和双核。而Physical Processor ID可以从cpuinfo或者dmesg中找到.
flags如果有ht说明支持超线程技术。判断物理CPU的个数可以查看physical id 的值,相同则为同一个物理CPU。
\begin{shellcmd}
 cat /proc/cpuinfo |grep "model name" && cat /proc/cpuinfo |grep "physical id"
\end{shellcmd}
\subsubsection*{CPU型号}
\begin{shellcmd}
cat /proc/cpuinfo |grep "model name"
\end{shellcmd}

\subsection*{内存信息}
\begin{shellcmd}
cat /proc/meminfo |grep MemTotal
sudo dmidecode |grep -A16 "Memory Device$"
\end{shellcmd}

\subsection*{硬盘大小}
\begin{shellcmd}
fdisk -l |grep Disk
\end{shellcmd}
df命令主要用来查询文件系统信息,可以用来查看硬盘以挂载的分区的大小
\begin{shellcmd}
df -h
\end{shellcmd}

\subsection*{主板型号}
\begin{shellcmd}
sudo dmidecode |grep -A16 "System Information$"
cat /proc/pci
\end{shellcmd}













