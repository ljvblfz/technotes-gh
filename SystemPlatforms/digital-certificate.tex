\section{数字证书}
1.带有私钥的证书
由Public Key Cryptography Standards \#12,PKCS\#12标准定义,包含了公钥和私钥的二进制格式的证书形式,以pfx作为证书文件后缀名。

2.二进制编码的证书
证书中没有私钥,DER 编码二进制格式的证书文件,以cer作为证书文件后缀名。

3.Base64编码的证书
证书中没有私钥,BASE64 编码格式的证书文件,也是以cer作为证书文件后缀名。
由定义可以看出,只有pfx格式的数字证书是包含有私钥的,cer格式的数字证书里面只有公钥没有私钥。
在pfx证书的导入过程中有一项是“标志此密钥是可导出的。这将您在稍候备份或传输密钥”。一般是不选中的,如果选中,别人就有机会备份你的密钥了。如果是不选中,其实密钥也导入了,只是不能再次被导出。这就保证了密钥的安全。
如果导入过程中没有选中这一项,做证书备份时“导出私钥”这一项是灰色的,不能选。只能导出cer格式的公钥。如果导入时选中该项,则在导出时“导出私钥”这一项就是可选的。
如果要导出私钥(pfx),是需要输入密码的,这个密码就是对私钥再次加密,这样就保证了私钥的安全,别人即使拿到了你的证书备份(pfx),不知道加密私钥的密码,也是无法导入证书的。相反,如果只是导入导出cer格式的证书,是不会提示你输入密码的。因为公钥一般来说是对外公开的,不用加密
