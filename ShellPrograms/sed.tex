\section{sed的使用}

用法: sed [选项]\ldots {脚本(如果没有其他脚本)} [输入文件]\ldots

\begin{verbatim}
-n, --quiet, --silent
    取消自动打印模式空间
-e 脚本, --expression=脚本
    添加“脚本”到程序的运行列表
-f 脚本文件, --file=脚本文件
    添加“脚本文件”到程序的运行列表
-i[扩展名], --in-place[=扩展名]
   直接修改文件(如果指定扩展名就备份文件)
\end{verbatim}

下面主要谈论sed脚本语言。


sed命令主要有:
\begin{description}
    \item[s/match/replace]替换
    \item[i\textbackslash\ TEXT]前插(insert)文本
    \item[a\textbackslash\ TEXT]后插(append)文本
    \item[r filename]后插文件
    \item[d]删除行
    \item[p]打印行
\end{description}

sed命令之前需要有0~2个量,被称作''地址''。addr1,addr2形式为地址范围,闭区间。addr1,addr2!后面的叹号表示取反。addr1,+N形式为addr及其后N行。addr1~N为自addr1起以N为步长的所有行。由 /regexp/表示regexp匹配到的行。

替换命令 举例:
\begin{verbatim}
echo 123456|sed -e s/123/wolf/g > haha4
more haha3|sed -e s/123/wolf/g > haha4
sed -e 's/<[^>]*>//g' haha.xml     (删除haha.xml中所有的xml标记)
\end{verbatim}

删除行命令举例:
\begin{verbatim}
sed -e /^$/d haha >> haha2  //删除文件haha中的空行
sed -e /33/d haha > haha2  //删除haha中包含33的行
sed -e 2d haha > haha2  //删除第2行
sed -e 2,4d haha > haha2  //删除第2~4行
\end{verbatim}

提取行命令:
\begin{verbatim}
sed -n '3,10p' myfile > newfile //提取从第3行到第10行
\end{verbatim}

提取不定行:
\begin{verbatim}
i=2
sed -n ${i}p myfile
\end{verbatim}


