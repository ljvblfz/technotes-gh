\section{IP-SAN连接}
加载iscsi相关模块:
\begin{shellcmd}
modprobe crc32c;
insmod iomtr_kstat.ko;
insmod scsi_transport_iscsi_dxf.ko;
insmod libiscsi_dxf.ko;
insmod iscsi_tcp_dxf.ko;
echo "127.0.0.1       localhost.localdomain   localhost" > /etc/host;
mkdir /var/lock/;
\end{shellcmd}

配置initiator端id,即IQN。主机的IQN必须经过磁盘阵列客户端的“定义”。
\begin{shellcmd}
mkdir /etc/iscsi;
echo "InitiatorName=iqn.2002-10.com.infortrend:raid.sn7905538.309" > /etc/iscsi/initiatorname.iscsi;
\end{shellcmd}

启动iscsi守护进程
\begin{shellcmd}
./iscsid_dxf;
\end{shellcmd}

寻找远程磁盘。在解压出的iscsi目录下,执行
\begin{shellcmd}
./iscsiadm_dxf -m discovery -t sendtargets -p 192.168.0.17;
\end{shellcmd}

该IP下的磁盘记录信息会被获取到本地。如果需要(如磁盘连接异常需重新连接),可用以下命令清空记录:
\begin{shellcmd}
./iscsiadm_dxf -m node -o delete
\end{shellcmd}

获取盘阵信息后,进行iscsi登录,连接磁盘:
\begin{shellcmd}
./iscsiadm_dxf -m node  -p 192.168.0.17 -l;
\end{shellcmd}

查看磁盘挂载情况:
\begin{shellcmd}
fdisk -l;
\end{shellcmd}

获取iscsi会话信息
\begin{shellcmd}
./iscsiadm_dxf -m session -P3
\end{shellcmd}




