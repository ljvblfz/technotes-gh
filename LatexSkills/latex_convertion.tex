
\subsection{Latex转rtf格式}
\begin{verbatim}
latex2rtf -C utf8 filename
\end{verbatim}
其中,filename不包含tex后缀。

公式错乱的话,可以用-M4选项将单行公式变成图片,或-M12将单行和文内公式全部变bitmap图片.转换成bitmap的过程非常耗时。

\subsection{Latex转html}
\begin{verbatim}
latex2html -html_version 4.0,latin1,unicode -split +0 haha
\end{verbatim}
其中,3.0以上的版本才能正确转换utf8编码的中文,split +0表示生成一个单一的html文件,否则会生成一堆文件。

\subsection{Latex公式转LibreOffice公式}
LibreOffice可安装Texmaths插件,即可以Latex的方式输入公式。


\subsection{LibreOffice转Latex}
安装Debian包libreoffice-writer2后,latexLibreOffice自动获得writer2latex扩展。可以以tex格式导出文件。但LibreOffice中的中文不能正确处理,会以[16进制数字串?]的方式写入tex文件。可以用perl或python等将这种串转为utf8编码,参\ref{subsec:PythonUnicode}。


