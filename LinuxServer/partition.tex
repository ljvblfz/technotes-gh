\section{分区创建与挂载}

\label{diskpartition}
\subsection{启动自动挂载设置} 
设置/etc/fstab,使得一些硬盘分区开机自动挂载。
\begin{lstlisting}
UUID=4C0E       /media/D \          
ntfs-3g defaults,nosuid,nodev,locale=zh_CN.UTF-8        0       0   
UUID=077  /media/E12    ext4    defaults        0       0   
\end{lstlisting}

然后执行

\begin{lstlisting}
mount -a
\end{lstlisting}

有个图形工具叫做pysdm;另外对于ntfs分区,可以使用ntfs-config工具。
配置完/etc/fstab后,使用mount -a命令执行该文件。


\subsection{分区查看与创建}
分区创建可使用fdisk和cfdisk命令,fdisk的man页自称buggy,推荐使用cfdisk等。
df可以查看分区使用率。

\subsection{分区格式化}
使用mkfs工具。

\subsection{分区挂载}
例如挂载优盘sdc,执行:
\begin{verbatim}
sudo mount -t msdos -o uid=li,gid=li /dev/sdc1 /mnt
\end{verbatim}


