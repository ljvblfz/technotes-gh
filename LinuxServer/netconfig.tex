\section{RHEL网络配置}


是否启用网络、主机名称信息于/etc/sysconfig/network文件中配置。
\begin{verbatim}
NETWORKING=yes
HOSTNAME=localhost.localdomain //修改该值作为主机名,如:rhel.lpwr.net
\end{verbatim}

配置网卡IP等信息, 编辑指定网络接口配置文件
\begin{verbatim}
/etc/sysconfig/network-script/目录下建立形如ifcfg-eth0:0的文件进行配置

DEVICE=eth0 //指定接口名称
ONBOOT=yes //系统启动时加载
BOOTPROTO=static //IP地址静态配置,若该值为“dhcp”则为动态获得
IPADDR=192.168.0.1 //设置IP地址
NETMASK=255.255.255.0 //设置子网掩码
GATEWAY=192.168.0.254 //设置默认网关
\end{verbatim}

重启网络以执行配置更新:

\begin{verbatim}
/etc/init.d/network restart
\end{verbatim}
或者
\begin{verbatim}
service network restart
\end{verbatim}

DNS信息由/etc/resolv.conf配置,重启后可能会被覆盖。

\begin{verbatim}
127.0.0.1 localhost.localdomain localhost //该行强烈建议保留
192.168.0.1 rhel.lpwr.net rhel //必须有三个字段:IP、FQDN、HOSTNAME
\end{verbatim}


