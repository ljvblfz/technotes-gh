\section{字体基础知识}

\subsection{字库标准}
PostScript(PS)是主要用于电子产业和桌面出版领域的一种页面描述语言和编程语言。
桌面出版系统使用的字库有两种标准: PostScript字库和truetype字库。这两种字体标准都是采用曲线方式描述字体轮廓,因此都可以输出很高质量的字形。TrueType最初是由苹果公司引入,用以回应Adobe专有的PostScript格式。早期的TrueType字体多为PostScript字体的拙劣翻版,但时至今日,各大字体公司都以TrueType格式来发布他们的产品(其中有些是有PostScript原型的,有些则原生就是TrueType,再无其他格式)。早在80年代末,苹果公司为了对抗Adobe公司的Type 1 PostScript字体,设计开发了TrueType,之后微软加入了开发,后来视窗系统的字体格式基本上都统一成TrueType,而在苹果的麦金塔系统中却成了PostScript和TrueType对立的局面。TrueType后来也被Linux等系统使用,成为标准字体。TrueType的主要强项在于它能给开发者提供关于字体显示、不同字体大小的像素级显示等的高级控制。在新开发的OpenType类型字体中,可以选择PostScript还是TrueType作为记述方式。

TrueType字体,中文名称全真字体。TrueType是由Apple公司和Microsoft公司联合提出的一种新型数学字形描述技术。它用数学函数描述字体轮廓外形,含有字形构造、颜色填充、数字描述函数、流程条件控制、栅格处理控制、附加提示控制等指令。其采用几何学中二次B样条曲线及直线来描述字体外形轮廓,特点是:TrueType既可作打印字体,又可用作屏幕显示;由于它用指令对字形进行描述,因此它与分辨率无关,输出时总是按打印机分辨率输出。无论放大或缩小,字符总是光滑的,不会有锯齿出现。但相对PostScript字体来说,其质量要差一些。特别是在文字太小时,就表现得不是很清楚。truetype字便宜,字款丰富。但一般情况厂truetype字不能直接由rip输出。需要经过特殊处理,比如转成曲线或输出时下载,使用起来较麻烦。速度也要慢一些,尤其是处理大量文字时很不方便,因此不适合用来作为页面的正文文字使用。

PostScript汉字库分为显示字库和打印字库,显示字库安装在制作计算机上,用来制作版面时显示用,通常由低分辨率的点阵字构成。打印字库要挂接在rip上,在解释页面时由rip把需要的字库调入页面并解释成记录的点阵。 PostScript汉字使用方便,输出速度快,是输出中心必备的。

TrueType vs. PostScript:TrueType有着更好的hinting属性,如果定义了hinting的话。这是一个小陷阱:必须要有人愿意为这些特性编写代码。Hinting是一种技术,用于数字化字体的平滑显示。它负责在小字号下处理矢量数据并将其转换为像素,以精确的显示字体;可以把hinting视作一种抗锯齿技术。这是另一个陷阱:hinting仅仅在屏幕小字号显示时才有用武之地。

TrueType并非不可以印刷,之所以不适合用于高端印刷,是因为打印机必须将TrueType矢量数据转换成PostScript语言。但不幸的是,PostScript不象TrueType那样支持如此多的曲线段 。
TrueType 使用二次曲线(quadratic curves)。和你在 Illustrator 和 Photoshop 中绘图时所用的曲线不同,二次曲线需要4个点来定义一条曲线。PostScript 使用三次曲线(cubic curves)。这是我们所熟悉的曲线,我们第一次在Illustrator 或 Photoshop 中理解和学习钢笔工具还是一个痛苦的回忆,但现在我们已经是轻车熟路了。三次曲线只需要3个点就可以定义一条曲线。对吧?因为三次方的意思就是自乘3次(其实应该是两次..译注),一个立方体就是3D。当TrueType 转换为 PostScript的时候,不是所有的二次曲线都能够转换为平滑的三次曲线。

OpenType,是一种可缩放字型(scalable font)电脑字体类型,采用PostScript格式,是美国微软公司与Adobe公司联合开发,用来替代TrueType字型的新字型。这类字体的文件扩展名为.otf,类型代码是OTTO,现行标准为OpenType 1.4。OpenType最初发表于1996年,并在2000年之后出现大量字体。它源于微软公司的TrueType Open字型,TrueType Open字型又源于TrueType字型。OpenType font包括了Adobe CID-Keyed font技术。Adobe公司已经在2002年末将其字体库全部改用OpenType格式。到2005年大概有一万多种OpenType字体,Adobe产品占了三分之一。

结论?使用Open Type格式,因为它可以同时嵌入TrueType 和 PostScript信息,同时它又是跨平台兼容的。全世界最棒!加油Adobe和Microsoft!听完我上面的解释,你应该能理解为什么要使用Open Type格式而不是其他格式了吧?如果你还想知道OpenType格式为什么如此优秀,看看这段引自Thomas Phinney的文章TrueType vs. PostScript Type 1中的描述:OpenType 将 PostScript 或 TrueType 轮廓都放入一个 TrueType 风格的“包装袋”中。应用程序和大多数的操作系统都在这个字体的子系统之外操作,不再关心这个“包装袋”中装的是什么类型的字体。

\subsection{ClearType技术}
ClearType,是由美国微软公司在其视窗操作系统中提供的屏幕亚像素微调字体平滑工具,让Windows字体更加漂亮。ClearType主要是针对LCD液晶显示器设计,可提高文字的清晰度。基本原理是,将显示器的R, G, B各个次像素也发光,让其色调进行微妙调整,可以达到实际分辨率以上(横方向分辨率的三倍)的纤细文字的显示效果。Windows上的像素和显示器上的像素对应的液晶显示器上效果最为明显,使用阶调控制一般CRT显示器上也可以得到一些效果。在Windows XP平台上,这项技术默认是关闭,到了Internet Explorer 7才默认为开启。而与ClearType几乎同样的技术在苹果电脑的Mac OS操作系统中,早在1998年发布的Mac OS 8.5就已经使用了。另外,依靠ClearType技术提高字体的可读性,相当程度上依赖于使用的字体,所以和原有的标准抗锯齿技术不能进行单纯比较。如果显示器不具有适用于ClearType的像素组合特性,以ClearType显示文字的实际效果会比使用前还要差。大多Windows默认的中文字体在显示小文字时使用点阵来显示,不使用ClearType。微软在Windows Vista里,新发布了两个ClearType中文字体:微软雅黑和微软正黑体。

为什么英文矢量字体就可以直接使用 ClearType 来进行平滑显示呢?这是因为大多数优秀的英文字体并不是采用内嵌点阵的方式来进行优化的,它们采用的是一种叫做 Hinting (字形微调)的技术来对小字号的显示进行优化。
  我们知道,矢量字体是可以无限平滑缩放的,在使用的时候,要通过操作系统的字体引擎自动的解析渲染为实际的像素,才能够在屏幕上显示出来。但是在字号很小的时候,由于能使用的像素非常有限,这种自动解析会出现很多问题,例如笔画粗细不匀,文字之间高低不齐,甚至笔画模糊无法识别等。因此必须由字体设计师人工干预,在矢量字库中嵌入一些附加的提示信息,来告诉字体渲染引擎在某个特定的字号下面,应该如何对这个字符的细节进行修正,才能准确的显示。这种在矢量字体中嵌入的提示信息,就叫做 Hinting 。
  对于中文字体来说,这种提示就更为重要,因为中文的笔画繁多,自动解析的错误也就更多更严重。在字号更小的情况下,根本无法显示全部的笔画,这时候还需要设计师在不影响整体的情况下,对笔画进行取舍,去掉一些不影响识别的笔画,否则这个文字就会因糊成一团无法识别。 Hinting 调整的范围需要涵盖各级小字号,一般最少要包括 9px - 18px 这个常用的字号区间。这种 Hinting ,即使是对于非常有经验的设计师,也是非常高难度而且费时费力的工作。

英文只有 26 个字母,但是对于中文的汉字情况就复杂的多了,仅仅是最常用的汉字就有 6000 个,然后为了在简繁体混排时候能完美的显示,就必须同时包含繁体和简体两套字符,再加上众多的不常用但是会在古籍文献中非常重要的生僻字,一套比较完整的大字符集字库所包含的字符数目将接近 3 万个。仅仅是这矢量造字的工作就是非常浩大的。
  这还不算,作为一套功能完整的正文字体,还需要考虑到斜体和粗体的显示。所有的斜体状态,也同样必须由设计师对不同的字号指定不同的 Hinting ,否则就会有显示问题。为了更完美的显示粗体,微软决定将标准体和粗体分开,作为两套单独的字体来设计,安装时也是两套字体,但在系统中使用时是显示为一套字体的不同状态。这套单独的黑体也同样需要单独造字,然后指定一系列的 Hinting 和斜体 Hinting 。因此要开发一套优秀的中文大型字库,耗费的人力物力是惊人的。这也正是这套字体会如此昂贵的原因之一。

Hinting信息是评价一款优秀矢量字体的一个重要指标,良好的Hinting能在小字号下面提供和内嵌点阵字一样优秀的显示质量,同时又降低内存的消耗。虽然我们现在已经拥有不少不错的矢量中文字体,但适合屏幕显示的正文字体很少,而包含完善 Hinting 信息的,一个也没有。所以,如果要在中文 Vista 平台下彻底完美的实现文本的平滑显示,微软就必须全新开发一套具备完善Hinting信息的ClearType中文字体。

\subsection{字体文件格式}

TTF(TrueTypeFont)是Apple公司和Microsoft公司共同推出的字体文件格式,随着windows的流行,已经变成最常用的一种字体文件表示方式。ttc是microsoft开发的新一代字体格式标准,可以使多种truetype字体共享同一笔划信息,有效地节省了字体文件所占空间,增加了共享性。但是有些软件缺乏对这种格式字体的识别,使得ttc字体的编辑产生困难。两者的不同处是 TTC 档会含超过一种字型,例如繁体 Windows 的 Ming.ttc 就包含细明体及新细明体两种字型 (两款字型不同处只是英文固定间距),而 TTF 就只会含一种字型。TTC是几个TTF合成的字库,安装后字体列表中会看到两个以上的字体。两个字体中大部分字都一样时,可以将两种字体做成一个TTC文件,现在常见的TTC中的不同字体,汉字一般没有差别,只是英文符号的宽度不一样,以便适应不同的版面要求。


























\subsection{常见商业字体}
维基百科有条目:\href{http://en.wikipedia.org/wiki/List_of_CJK_fonts}{List of CJK fonts}

\begin{verbatim}
中易宋体&新宋体,即通常被熟知为宋体、新宋体的字体,是由北京中易中标电子信息技术有限公司制作并持有版权的两个TrueType字体。中易宋体&新宋体是随着简体中文版Windows和Microsoft Office一起分发的字体(文件名 Simsun.ttc)。Simsun一直是简体中文版Windows XP系统及之前版本的默认字体。但由于白体的特性,在Windows Vista中已经改用支持OpenType的微软雅黑。因为对于电脑显示器来说,应该选择黑体即无衬线体作为显示器字体,才有助于显示和阅读。此字体西文的半角字符部分由于采用等宽字体设计,被指衬线和笔画的比例太夸张而不便阅读。此宋体在8pt时已经无法正常地看清(即使打开了ClearType)。此外,宋体只有区区几个最佳字体大小,在某些大小时(如14pt及以上)会发现字的笔划有残缺、断裂、粗细不均的问题(若打开了ClearType的话,“横”仍然会看到有断裂的地方),这主要是字体没有加入Hinting信息的缘故。

中易黑体,Windows XP系统及之前版本中的默认黑体。通常被熟知为``黑体''的电脑字体其实是中易黑体,是由北京中易中标电子信息技术有限公司制作并持有版权的一种TrueType字体。中易黑体是随着简体中文版Windows和Microsoft Office一起分发的字体,文件名Simhei.ttc,目前版本为5.01。此字形同中易宋体的字形设计被指太琐碎,西文部分采用等宽字体,使得美感颇为降低。

微软雅黑是美国微软公司委托中国方正集团设计的一款全面支持ClearType技术的字体。蒙纳公司(Monotype Corporation)负责了字体的修飾(Hinting)工作。它属于OpenType类型,文件名是MSYH.TTF,在字体设计上属于无衬线字体和黑体。在使用ClearType功能的液晶显示器中,微软雅黑比以前Windows XP默认的中易宋体更加的清晰易读。

微软正黑体,微软公司的一款全面支援ClearType技术的TrueType 无衬线字体,用于繁体中文系统。随Windows Vista及Office 2007(繁体中文)一起发布,是Windows Vista缺省字体,在使用的ClearType功能的液晶显示器中,其比以前Windows XP缺省的新细明体更加清晰易读。中文世界里一套合适的 ClearType 屏幕正文显示字体字体必须能解决在 ClearType 平滑显示状态下小字号正常阅读的问题。现有的所有中文字库都无法在 ClearType 平滑显示状态下完美的文本显示。之前的宋体之所以能够在小字号点阵状态下很好的显示,是由于宋体在矢量字库中嵌入了 12 、 14 、 16 、 18 等几个点阵字库,才得以比较优秀的显示。但在 ClearType 状态下,继续采用这样内嵌点阵的方式来显示汉字,就会和平滑显示的英文粗细不一致,同时风格上非常的不协调。由于当初的宋体不是为平滑显示而设计的,强制平滑显示的效果就显得纤细发虚,看起来很模糊。

对于微软雅黑和微软正黑,不好简单的用简体或者繁体来区分他们,因为这两套字体都同时包含了比较完整的简繁体汉字,以确保在简体和繁体混排的页面上都能够完美的显示。但由于两岸的文教部门在各自的文字规范中对汉字的写法规定有很多细节上的不同,所以这两套字形在正式场合是不能混淆使用的。同样的,日文的Meiryo字体中也包含了大量的繁体汉字,不过由于汉字在日本也经过了上千年的演变,日文中的汉字写法和中国大陆和台湾也有着相当的区别。

华文细黑,是由中国常州华文印刷新技术有限公司(SinoType,找不到官网,不知是否倒闭)制作并持有版权的一种TrueType电脑字体。在无中文的环境下显示的名称为STHeiti Light或STXihei,它属于华文黑体系列字体之一。在设计上,它属于黑体或无衬线体。

华文宋体,常州华文印刷有限公司推出的字体。较平常的宋体相比,华文宋体在审美上有着新的开创与突破。目前,华文宋体已广泛应用到商标设计,广告设计,报纸与图书等行业中,效果较好。

威锋数位(DynaComware),原名华康科技(DynaLab),是台湾的电脑软件公司,为少数以电脑字型为主要开发产品的公司。自Windows 3.0推出时起,华康科技与微软公司合作将Windows中文化,并开始提供字型至今。繁体中文版Windows常用的细明体、新细明体及标楷体等皆由此公司制作提供。

RedHat发布Liberation系列,包含“Liberation Serif”“Liberation Sans”“Liberation Mono”三种字体,目标非常明确,就是取代微软的“Times New Roman”“Arial”“Courier New”。字形尺寸与微软的完全一样,所以用这三种字体代替微软的那三种不会导致文档排版的偏差与错乱。

   Nimbus系列,包含“Nimbus Roman No9 L”“Nimbus Sans L”“Nimbus Mono L”三种。目标直指Adobe的“Times”“Helvetica”“Courier”,主要供打印使用。与Liberation系列不同的是,其衬线看起来和微软的很接近,但尺寸不同。因为微软的那几款字体本来也就是针对Adobe的,字形和Adobe的几乎一样而尺寸却并没做到一样。

\end{verbatim}

















