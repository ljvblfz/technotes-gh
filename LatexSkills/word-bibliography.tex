\section{Word参考文献自动修改}
写论文,参考文献的修改很麻烦,删除一个,添加一个,就需要改一长串数字。怎么办呢。本人推荐一种简单方法:尾注法。方法如下(以Word2003为例):
\begin{verbatim} 
1.光标移到要插入参考文献的地方,菜单中“插入”——“引用”-“脚注和尾注”。   
2.对话框中选择“尾注”,编号方式选“自动编号”,所在位置建议选“节的结尾”。   
3.如“自动编号”后不是阿拉伯数字,选右下角的“选项”,在编号格式中选中阿拉伯数字。   
4.确定后在该处就插入了一个上标“1”,而光标自动跳到文章最后,前面就是一个上标“1”,这就是输入第一个参考文献的地方。   
5.将文章最后的上标“1”的格式改成正常(记住是改格式,而不是将它删掉重新输入,否则参考文献以后就是移动的位置,这个序号也不会变),再在它后面输入所插入的参考文献(格式按杂志要求来慢慢输,好像没有什么办法简化)。   
6.对着参考文献前面的“1”双击,光标就回到了文章内容中插入参考文献的地方,可以继续写文章了。   
7.在下一个要插入参考文献的地方再次按以上方法插入尾注,就会出现一个“2”(Word已经自动为你排序了),继续输入所要插入的参考文献。   
8.所有文献都引用完后,你会发现在第一篇参考文献前面一条短横线(页面视图里才能看到),如果参考文献跨页了,在跨页的地方还有一条长横线,这些线无法选中,也无法删除。这是尾注的标志,但一般科技论文格式中都不能有这样的线,所以一定要把它们删除。   
9.切换到普通视图,菜单中“视图”——“脚注”,这时最下方出现了尾注的编辑栏。   
10.在尾注右边的下拉菜单中选择“尾注分隔符”,这时那条短横线出现了,选中它,删除。   
11.再在下拉菜单中选择“尾注延续分隔符”,这是那条长横线出现了,选中它,删除。   
12.切换回到页面视图,参考文献插入已经完成了。这时,无论文章如何改动,参考文献都会自动地排好序了。如果删除了,后面的参考文献也会自动消失,绝不出错。   
13.参考文献越多,这种方法的优势就体现的越大。
还没完,标号上的方括号如何加呢?很简单:
在全文中,查找尾注标记^e,然后全部替换为[^&]即可;如果用了脚注就查找脚注标记^f,再全部替换为[^&]便可以了(注意查找时让“不限定格式”按钮为灰色)
别看步骤多,操作一遍,就知道很简单了。
\end{verbatim}
