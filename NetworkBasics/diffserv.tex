\section{DiffServ}
区分服务是为解决服务质量问题在网络上将用户发送的数据流按照它对服务质量的要求划分等级的一种协议。

区分服务体系结构(DiffServ)定义了一种可以在互联网上实施可扩展的服务分类的体系结构。服务分类要求能适应不同应用程序和用户的需求,并且允许对互联网服务的分类收费。
在本体系结构,只在网络的边界节点上实现复杂的分类和调节功能,并且,通过在 IPv4 和 IPv6 包头的 DS 段做适当的标记,聚合流量,然后根据所做的标记,采取不同的每一跳转发策略。核心网络节点上,无需维护每个应用程序流或每个用户转发状态。

DittServ的最大特点就是简单有效、扩展性强。其实施特点是采用聚合的机制将具有相同特性的若干业务流聚合起来,为整个聚合流提供服务,而不再面向单个业务流。
区分服务只包含有限数量的业务级别,状态信息的数量少,因此实现简单,扩展性较好。它的不足之处是很难提供基于流的端到端的质量保证。目前,区分服务是业界认同的IP骨干网的QoS解决方案,但是由于标准还不够详尽,不同运营商的DiffServ网络之间的互通还存在困难。IETF RSVP和DiffServ两个工作组都正在研究RSVP与DiffServ相结合的问题,以进一步扩大DiffServ与现有系统的可兼容性,此外在业务分类、业务性能的量化描述以及域间业务类型映射等问题上,DiffServ模型也需进一步明确和开展的研究。



