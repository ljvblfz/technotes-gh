\section{Ubuntu/Deepin安装后的配置}

\subsection{配置网络和locale}
直接使用图形工具network-manager-gnome即可。

\subsection{硬盘设置}
设置/etc/fstab,使得一些硬盘分区开机自动挂载。
\begin{shellcmd}
UUID=4C0E	/media/D \	
ntfs-3g	defaults,nosuid,nodev,locale=zh_CN.UTF-8	0	0
UUID=077  /media/E12	ext4	defaults	0	0
\end{shellcmd}
有个图形工具叫做pysdm;另外对于ntfs分区,可以使用ntfs-config工具。
配置完/etc/fstab后,使用mount -a命令执行该文件。

\subsection{软件安装}
设置软件更新源,更新系统,并安装软件。对于deepin官方源,直接支持IPV6。对于USTC等其他源,可能需要将地址中的mirror改为mirror6。另外如果sources.list.d目录下有无法访问的地址,需要将其删除,可以将该目录重命名。

PPA清单:
\begin{verbatim}
UbuntuTweak
ppa:tualatrix/ppa
pidgin的WebQQ插件
ppa:lainme/lwqq
\end{verbatim}

包清单:
vim, vim-gnome, vim-scripts, vim-doc, vim-latexsuite,ntfs-config,pysdm,ubuntu-tweak(ppa:tualatrix/ppa),gnome-tweak-tool,nautilus-open-terminal,gdebi,Pidgin,chromium-browser, aMSN, sshpass, lcrt, texlive, texlive-xetex

需要手动下载的商用软件:
永中Office,Adobe Reader, Nutstore, Mendeley, Chrome

\subsubsection{系统工具}
Gnome桌面:gnome-shell和经典桌面gnome-panel。gnome-shell可以安装panel-dock插件。通过PPA可以安装cinnamon桌面,是LinuxMint开发的Gnome2风格的桌面。

Deb包管理器gdebi。

系统设置工具:ubuntu-tweak(ppa:tualatrix/ppa),gnome-tweak-tool。

nautilus辅助:nautilus-open-terminal

深度软件中心:deepin-softwarecenter(ppa:noobslab/deepin-sc)。

Adobe Air:官网下载安装即可。在Ubuntu12.04上安装会出错,可以执行:
\begin{verbatim}
sudo ln -s /usr/lib/i386-linux-gnu/\
libgnome-keyring.so.0.2.0 \
/usr/lib/libgnome-keyring.so.0
\end{verbatim}

\subsubsection{网络通信}
Pidgin,chromium-browser, aMSN, sshpass, lcrt, 网盘等。

\subsubsection{音视频播放}
视频播放器:Gnome-Mplayer和VLC。

视频网站客户端:ppstream。

音乐播放器:可选择banshee,audacious,rhythmbox。

歌词搜索程序:osdlyrics(ppa:osd-lyrics/ppa)为独立程序。执行osdlyrics命令,以将其打开。Ctrl+Shift+L快捷键使其解除锁定。

\subsubsection{办公软件}

VIM:vim-scripts,vim-latexsuite和vim-doc包含了Vim的一些有用插件。vim-addon-manager为Vim插件管理程序,安装该工具之后可以用vim-addons命令激活各种插件。

Latex:在texlive和texlive-xetex, texlive-latex-extra。

办公套件:永中2012青春版。

PDF文件阅读与编辑:acrobat, xournal, pdfmod。 

文献管理:mendeleydesktop。

制图工具:QtiPlot(仿制Origin), dia(模仿Visio)。论文制图也可使用办公套件中的电子表格程序,或GNUplot。 

词典: Goldendict

编程相关: ctags, cscope, doxygen, manpages-dev(Ubuntu可能已经预装), qtcreator, python-qt4, python-qt4-doc

\subsubsection{虚拟机}
KVM虚拟系统:qemu-kvm libvirt-bin virt-manager bridge-utils。
KVM制作的虚拟机保存路径一般是/var/lib/libvirt/images。
打开virt-manager可能需要root权限。


\subsection{系统设置工具一览}

系统设置:gnome-control-center

高级设置:gnome-tweak-tool

系统配置机制:gnome3用Gsettings机制取代了gconf机制。用gsettings取代了gconftool-2工具。dconf-editor是gsettings的众多后端工具之一。

\subsection{设置unity白名单}
如果是Unity桌面,设置unity白名单:
\begin{shellcmd}
gsettings set com.canonical.Unity.Panel systray-whitelist "['all']"
\end{shellcmd}

\subsection{字体配置}
文泉驿可能被默认安装了。安装少量Windows字体即可。如simsun,simfang,simhei,simkai
使用\verb+fc-list :lang=zh-cn+查看当前系统字体。

\subsection{主菜单配置}
使用alacarte创建的应用程序的快捷入口,创建在~/.local/share/applications目录下,并默认以 alacarte-made[-X].desktop 的格式命名,其中X是数字,用户可以随后重命名这个文件,菜单上显示的内容不会改变。而在alacarte工具中删除的快捷入口,也不会真的删除对应的 desktop 文件,而只是将对应文件中的Hide字段的值改为true。
以root权限安装的程序,其快捷入口大多创建在/usr/share/applications目录下,而以用户权限安装的程序,则只能将快捷入口创建在~/.local/share/applications目录下。

\subsection{默认程序配置}
有几个文件用于存储指定类型文件的关联程序,分别是 /etc/gnome/defaults.list, /usr/share/applications/defaults.list, /usr/share/applications/mimeinfo.cache, ~/.local/share/applications/mimeapps.list, ~/.local/share/applications/mimeinfo.cache。前面三个文件保存全局设置,后面两个保存用户设置。如果要修改某个类型文件的关联程序,可以通过直接修改这几个文件的方式实现。

\subsection{开机自启动程序}
gnome-session-properties工具,可以在主菜单->启动应用程序中找到。

\subsection{桌面启动器配置}
gnome-desktop-item-edit,可以创建或编辑.desktop类型的文件。创建名为my.desktop的启动器,执行:
\begin{verbatim}
gnome-desktop-item-edit --create-new my.desktop
\end{verbatim}










\section{系统备份}
需要备份的配置文件包括


\begin{verbatim}
/etc/hosts
/etc/fstab
/etc/lightdm/lightdm.conf

.bash*
.vimrc
.gitconfig
.ssh/*
.config/Chromium/User*
\end{verbatim}


\section{ISO镜像与光盘}
在Ubuntu下将文件夹做成iso镜像的工具叫做genisoimage(原名mkisofs)。可引导光盘加载后对相应文件夹使用genisoimage工具,会失去引导性。如欲保留引导性,可以使用dd工具对裸设备进行操作。如:

\verb+dd if=/dev/sr0 of=deepin.iso+

如需将iso烧录入光盘,可以使用Brasero工具。这很可能是Ubuntu系统自带的。

\section{制作USB启动盘}
可以使用dd命令或使用一些图形工具。在维基百科上有一个制作LiveUSB的软件列表。包括:
\begin{itemize}
  \item Fedora Liveusb-creator,在Windows或Linux下制作Fedora的LiveUSB
  \item Ubuntu Liveusb Creator(在命令为usb-creator-gtk或usb-creator-kde)
  \item LinuxLive USB Creator,在Windows上制作Linux LiveUSB
  \item Unetbootin,在Windows或Linux下制作UNIX LiveUSB,Ubuntu软件仓库有提供
\end{itemize}
我自己用优盘实际创建启动盘,在Ubuntu下使用Ubuntu Liveusb Creator制作失败了。用dd和Unetbootin制作成功。

如果使用dd命令,先umount这个USB设备,其名称可能是sdb1。但dd命令要作用与sdb而不是sdb1上。
\begin{verbatim}
dd if=linux_image.iso of=/dev/sdb
\end{verbatim}
尝试使用dd命令将Windows安装镜像做成启动盘,未能成功,U盘在启动时未能引导,依然由硬盘进行了引导。

制作完成后,需要更改BIOS设置,让USB-HDD(或其他USB设备类型)的设备启动优先级高于HDD。优盘可能被当作硬盘处理,如果是这种优盘,开机时需要在BIOS设置中更改的不是设备启动顺序,而是HDD启动优先级。在实验室电脑上DEL键进入BIOS设置。


\section{中文输入与显示问题}

\subsection{修改locale为GB18030}

\begin{verbatim}

/usr/share/i18n目录下为系统已经下载的字符集和locale定义文件。

在 /var/lib/locales/supported.d/local 添加两行:
zh_CN.GBK GBK和zh_CN.GB18030 GB18030 

执行sudo locale-gen 生成相应locale
在/etc/default/locale文件配置LANG,LC_*, LC_ALL等参数。其优先级递增。
也有说修改/etc/environment。但修改的结果都是只改变了root用户的locale。

普通用户的locale可以通过export LANG修改环境变量的方法进行修改。
但将locale编码改为GB码后会出现各种乱码,网上建议还是保留为UTF-8。

\end{verbatim}




\subsection{Evolution客户端没有附件}
Evolution发的邮件附件Foxmail不认,在 Evolution 的 [编辑] -> [首选项] -> [编写器首选项] 中
将“使用 Outlook / Gmail 的方式编码文件名” 前的复选框打上勾就可以了

\subsection{查看文件编码}
\begin{shellcmd}
file *
enca *
本地化编码 enconv *
\end{shellcmd}

\subsection{转换文件名由GBK为UTF8}
\begin{shellcmd}
convmv -r -f cp936 -t utf8 --notest --nosmart *
\end{shellcmd}

批量转换src目录下的所有文件内容由GBK到UTF8
\begin{shellcmd}
find src -type d -exec mkdir -p utf8/{} \;
find src -type f -exec iconv -f GBK -t UTF-8 {} -o utf8/{} \;
mv utf8/* src
rm -fr utf8
\end{shellcmd}

\subsection{文字大小不一致}
\begin{shellcmd}
sudo apt-get remove ttf-kochi-gothic ttf-kochi-mincho ttf-unfonts ttf-unfonts-core
\end{shellcmd}

\subsection{gedit中文乱码}

 命令行指定编码,例如gedit --encoding=gbk 霜冷长河.txt 。也可以配置Gsettings参数,如:
   
   \begin{shellcmd}
    gsettings set org.gnome.gedit.preferences.encodings auto-detected 
	"['UTF-8','GB18030','GB2312','GBK','BIG5','CURRENT','UTF-16']"
    gsettings set org.gnome.gedit.preferences.encodings shown-in-menu 
	"['GB18030','UTF-8','GB2312','GBK','BIG5','CURRENT','UTF-16']"
   \end{shellcmd}  

	相关图形工具叫做dconf-editor,在dconf-tools包中

  
  


\subsection{vim字符编码}
涉及3个参数:enc,fenc,fencs.
fenc为当前文件的属性,可以对当前文件set fenc,在保存,从而设置文件的编码方式。在vimrc中设置fenc,就是设置以后写入的所有文件的编码方式。
fencs为读入文件时的编码猜测列表。enc为vim工作时的编码方式,但最终写入文件时采取的编码仍取决于fenc。
 set fencs=utf-8,gbk
 set fenc=gb18030

\subsection{PDF 文件乱码}
PDF中文乱码有两种
1.汉字显示为方块:如各种从中国知网下载的论文。网上的方法多针对此种乱码

2.汉字显示为各种欧洲字母:如我电脑中的"西游记.pdf", 基于popplar的阅读器都不能搞定,foxit4linux和Adobe Reader也都失败。永中office的pdf阅读器可以搞定。foxit for linux在2009年烂尾,目前仍为1.1版。
\begin{shellcmd}
sudo apt-get install xpdf-chinese-simplified xpdf-chinese-traditional poppler-data
\end{shellcmd}

\begin{shellcmd}
sudo gedit /etc/fonts/conf.d/49-sansserif.conf 
将倒数第四行 <string>sans-serif</string>
改为 <string>sans</string>
保存即可,重启firefox
\end{shellcmd}

\subsection{unzip中文文件名乱码}
\begin{shellcmd}
sudo apt-get install p7zip-full
export LANG=zh_CN.GBK  #临时在控制台修改环境为zh_CN.GBK,然后解压缩即可
7za e docs.zip
\end{shellcmd}

\subsection{输入法问题}
ibus跟随
\begin{shellcmd}
安装ibus-gtk即可,最好另外安装:ibus-qt4
\end{shellcmd}
输入法切换
\begin{shellcmd}
im-switch -c
\end{shellcmd}









\section{中文字体的安装使用}
\subsection{查看已安装的中文字体}
\begin{shellcmd}
fc-list :lang=zh|sort
\end{shellcmd}

\subsection{安装文泉驿字体}
在Ubuntu软件仓库中,文泉驿点阵宋体和矢量正黑体分别包含于xfonts-wqy,ttf-wqy-zenhei软件包,不过很可能已经预装。
在Fedora中,点阵宋体和矢量正黑体分别位于wqy-bitmap-fonts,wqy-zenhei-fonts

\subsection{安装文鼎字体}
在Ubuntu软件仓库中,we文鼎楷体和明体分别位于ttf-arphic-ukai,ttf-arphic-uming软件包,不过很可能已经预装。
新发布的有文鼎PL报宋二GBK

\subsection{安装Windows字体}

将simsun等字体复制到目录/usr/share/fonts/TTF

在字体所在目录执行命令:
\begin{shellcmd}
mkfontdir  生成font.dir
ttmkfdir   为ttc字体生成font.dir
mkfontscale 为simsun.ttc生成font.scale
\end{shellcmd}
生成encoding文件:
\begin{shellcmd}
sudo cp /usr/share/fonts/encodings/encodings.dir ./
\end{shellcmd}
以上命令为X.org 字体系统服务。以下命令为xft字体系统服务:
\begin{shellcmd}
sudo fc-cache -fv   刷新xft字体库
\end{shellcmd}
重启X即可。

\subsection{字体大小}
1em=16px

\subsection{字体与文件名对照}
\begin{verbatim}
方正舒体              FZSTK.TTF

方正姚体              FZYTK.TTF

微软雅黑              MSYH.TTF

微软雅黑 Bold         MSYHBD.TTF

仿宋体               simfang.ttf

黑体                 simhei.ttf

楷体                 simkai.ttf

隶书                 SIMLI.TTF

宋体 & 新宋体        simsun.ttc

幼园                 SIMYOU.TTF

华文彩云            STCAIYUN.TTF

华文仿宋            STFANGSO.TTF

华文琥珀            STHUPO.TTF

华文楷体            STKAITI.TTF

华文隶书            STLITI.TTF

华文宋体            STSONG.TTF

华文细黑            STXIHEI.TTF

华文行楷            STXINGKA.TTF

华文新魏            STXINWEI.TTF

华文中宋            STZHONGS.TTF

宋体-方正超大字符集 SURSONG.TTF
\end{verbatim}
\section{LinuxDeepin 12.06源地址}

\begin{verbatim}

#北京交通大学
deb http://mirror.bjtu.edu.cn/deepin/ precise main non-free
deb-src http://mirror.bjtu.edu.cn/deepin/ precise main non-free

deb http://mirror.bjtu.edu.cn/deepin/ oneiric-updates main non-free
deb-src http://mirror.bjtu.edu.cn/deepin/ oneiric-updates main non-free

#中国科技大学
deb http://mirrors.ustc.edu.cn/deepin/ precise main non-free
deb-src http://mirrors.ustc.edu.cn/deepin/ precise main non-free

deb http://mirrors.ustc.edu.cn/deepin/ precise-updates main non-free
deb-src http://mirrors.ustc.edu.cn/deepin/ precise-updates main non-free

#清华大学
deb http://mirrors.tuna.tsinghua.edu.cn/deepin/ precise main non-free
deb-src http://mirrors.tuna.tsinghua.edu.cn/deepin/ precise main non-free

deb http://mirrors.tuna.tsinghua.edu.cn/deepin/ oneiric-updates main non-free
deb-src http://mirrors.tuna.tsinghua.edu.cn/deepin/ oneiric-updates main non-free


#天津大学(仅IPv4)
deb http://mirror.tju.edu.cn/deepin/ precise main non-free
deb-src http://mirror.tju.edu.cn/deepin/ precise main non-free

deb http://mirror.tju.edu.cn/deepin/ precise-updates main non-free
deb-src http://mirror.tju.edu.cn/deepin/ precise-updates main non-free

#天津大学ftp(仅IPv4)不用
deb ftp://mirror.tju.edu.cn/deepin/ precise main non-free
deb-src ftp://mirror.tju.edu.cn/deepin/ precise main non-free

deb ftp://mirror.tju.edu.cn/deepin/ precise-updates main non-free
deb-src ftp://mirror.tju.edu.cn/deepin/ precise-updates main non-free

#天津大学ftp(IPv4/IPv6)
deb http://jx.tju.zyrj.org/deepin/ precise main non-free
deb-src http://jx.tju.zyrj.org/deepin/ precise main non-free

deb http://jx.tju.zyrj.org/deepin/ precise-updates main non-free
deb-src http://jx.tju.zyrj.org/deepin/ precise-updates main non-free
\end{verbatim}


\section{常见错误与异常}
更新源下载目录时,提示没有与主机相关联的地址,可能是网络不通

系统死机时,如键盘还有效,可以执行CtrlAlt+f切换到其他tty,然后通过top和kill杀死一些占用资源较大的进程。如果键盘无效,应尝试安全重启。即按住Alt+SysRq,然后依此执行R,E,I,S,U,B等6个命令。

lightDM登录失败,输入用户密码执行登录崩溃又回到登录页面。通过删除~目录下Xauthority文件或许可以解决。网上其他解决方法包括:安装gdm取代lightDM,或放弃当前用户新建用户。甚至有人干脆提出重装系统。
\section{软件界面的绿豆沙色背景}
关于绿豆沙色的定义网上找到了多种:
1.RGB(204,232,207),RGB hex(CCE8CF)。

2.RGB hex(CCE8CF)

3.RGB(199,237,204),HSL(85,123,205)

4.HSL(84,91,205)

5.HSL(85,90,205),RGB(187,247,197)

网上有人推荐第5种.

能设置背景颜色的软件包括gnome-terminal, chromium浏览器,Adobe reader, okular等。


Adobe:Edit->Preferences->Accessibility(这里一般译作"辅助功能")->Document Color Options

okular:setting->configure okular->accessibility0->color mode

evince不能设置任意背景色,只能反色为黑背景。其他Linux上的pdf阅读器,如xournal,永中pdf阅读器等,不能设置不背景色。wine过的cajviewer,foxit reader能够设置背景色。



Chromium:
修改配置文件:
~/.config/chromium/Default/User StyleSheets/Custom.css
添加:
\begin{shellcmd}
html, body {background-color: #CCE8CC!important;}
\end{shellcmd}



关于HSL,包括:
H:Hue,色相,色调
S:Sat,Saturation,饱和度
L:Lum,Lightness,也有说Intensity,亮度

也有HSB(HSV)色彩空间,
V(Value)B(Brightness)译作明度
Adobe的软件使用HSV而非HSL
\newpage
\section{Fedora16安装之后}
\subsection{配置网络}
使用图形界面配置即可。注意netmask为24,不是255.255.255.0
\subsection{配置时间和地区}
应该设置地区为上海,这样在安装很多应用程序到时候会下载中文语言
\subsection{配置更新源}
\subsubsection{手动选择源}
开源镜像网站http://mirrors.163.com/和http://mirrors.sohu.com/下载fedora的源配置文件。为了区别镜像,打开下载的镜像文件后把三个中括号[]中的fedora分别替换为fedora-163与fedora-sohu。将修改好的配置文件保存到/etc/yum.repos.d/目录下。最后在终端输入:
\begin{shellcmd}
sudo yum makecache
\end{shellcmd}

\subsubsection{安装fastest-mirror}
\begin{shellcmd}
sudo yum -y install yum-plugin-fastestmirror 
\end{shellcmd}
\subsubsection{添加rpmfusion源}
\begin{shellcmd}
sudo rpm -ivh http://download1.rpmfusion.org/free/fedora/rpmfusion-free-release-stable.noarch.rpm 
http://download1.rpmfusion.org/nonfree/fedora/rpmfusion-nonfree-release-stable.noarch.rpm
\end{shellcmd}

\subsection{升级系统}
\begin{shellcmd}
sudo yum update
\end{shellcmd}

\subsection{配置字体}
安装文泉驿字体
\begin{shellcmd}
sudo yum -y install wqy-bitmap-fonts wqy-zenhei-fonts wqy-unibit-fonts
\end{shellcmd}
安装Windows字体
\begin{itemize}
\item 将windows字体拷贝到/usr/share/fonts/某目录/下
\item chmod 755 *
\item mkfontscale;mkfontdir;fc-cache -fv
\end{itemize}

\subsection{安装系统工具}
\subsubsection{安装gnome-tweak-tool}
\begin{shellcmd}
sudo yum -y install gnome-tweak-tool
\end{shellcmd}
\subsubsection{安装gnome shell extension}
访问https://extensions.gnome.org/
\subsubsection{其他工具}
\begin{itemize}
\item xkill.需安装xkill或xorg-x11-apps
\item nautilus-open-terminal.安装完成后ctr+alt+backspace重启X
\item faenza-icon-theme.
sudo yum -y install faenza-icon-theme
\end{itemize}

\subsection{安装常用软件}
\subsubsection{安装google-chrome}
google并没有直接提供yum源,而是以sh文件的方式提供。那么就下载这个文件
\begin{shellcmd}
wget https://dl-ssl.google.com/linux/google-repo-setup.sh

sudo sh google-repo-setup.sh 

sudo yum -y install google-chrome-stable

rm google-repo-setup.sh 
\end{shellcmd}
\subsubsection{安装latex和xelatex}
\begin{shellcmd}
sudo yum -y install texlive texlive-xetex

fmtutil --enablefmt xelatex

sudo yum -y install texmaker
\end{shellcmd}
\subsubsection{安装多媒体解码器}
fedora默认没有安装视频解码器,所以不能听歌看视频,打开歌曲时会提示缺少MPEG-1 Layer3。
首先确保系统已经安装rpmfusion源,在终端中输入命令:
\begin{shellcmd}
sudo yum -y install ffmpeg ffmpeg-libs gstreamer-ffmpeg \
libmatrosca xvidcore libdvdread libdvdnav lsdvd
sudo yum -y install gstreamer-plugins-good \ 
gstreamer-plugins-bad gstreamer-plugins-ugly
\end{shellcmd}

\subsubsection{安装Office}
安装LibreOffice:
\begin{shellcmd}
sudo yum -y groupinstall "Office/Productivity"
sudo yum -y install libreoffice-langpack-zh-Hans
\end{shellcmd}
永中Office可以从官网下载

\subsubsection{Vim及其插件}
安装gvim
\begin{shellcmd}
sudo yum -y install gvim
\end{shellcmd}
安装vim-latex
\begin{shellcmd}
sudo yum -y install vim-latex
\end{shellcmd}
所谓vim-addon-manager有两个意思,一个指debian下的软件,一个是vim插件,这里指后者.从官网下载该插件后,解压,然后配置.vimrc文件指定vim-addon-manager路径和想安装的插件的名称.
例如:
\begin{shellcmd}
set runtimepath+=/PATH/TO/VIM-ADDON-MANAGER
call vam#ActivateAddons([``vim-haxe'',``snipmate''])
call vam#ActivateAddons([``OmniCppComplete''])
call vam#ActivateAddons([``The_NERD_Commenter''])
\end{shellcmd}
下一次打开vim的时候会自动提示安装相应插件.如果插件名称有微小的错误(typo),可能会得到正确提示.

\subsubsection{KVM虚拟机}
安装KVM虚拟机
\begin{shellcmd}
sudo yum -y install kvm qemu libvirt virt-manager
\end{shellcmd}
可以利用virt-manager安装Windows XP系统,然后安装360安全卫士,搜狗拼音,360浏览器,360压缩等.注意yunio不支持ie系统的浏览器,所以可以再安装一个chrome浏览器.

\subsubsection{其他}
mendeley,yozo office,fcitx输入法,unrar解压软件,qt-creator(在ubuntu下叫做qtcreator),AdobeReader\_chs,antiword

\subsection{卸载不需要的软件}
ibus等

\subsection{修改默认应用程序}
在Fedora下,有两个配置文件:
/usr/share/applications/defauts.list \\
/usr/local/share/applications/defauts.list
其关系不明

\subsection{配置硬盘自动挂载}
修改/etc/fstab

\subsection{关闭SELinux}
修改/etc/selinux/config



\newpage




\section{Grub与系统登录}
\subsection{修改Grub配置}
配置/etc/default/grub文件,然后运行update-grub,会自动生成/boot/grub/grub.cfg文件
\subsection{安装Grub}
如果安装了多个Linux系统,最后一个安装的系统的GRUB会覆盖之前安装的GRUB。如果这不是所希望的,可以进入心目中的默认系统重新安装GRUB。如果最后安装的是Windows,
则GRUB会被破坏,不会在开机时显示GRUB界面,也需要设法进入Linux后重新安装GRUB。
进入所希望的Linux系统后,执行grub-install. grub-install copies GRUB images into /boot/grub, and uses grub-setup to install grub into the boot sector.
\begin{verbatim}
sudo grub-install /dev/sda
\end{verbatim}
可以添加选项,\verb+boot-directory=DIR+
install GRUB images under the directory DIR/grub instead of the /boot/grub directory。
如果是从LiveCD进入的系统,执行grub-install前先chroot。
如果没有LiveCD,在进入系统前停留在了如下界面上:
\begin{verbatim}
GRUB loading
error:unknow filesystem
grub rescue>
\end{verbatim}
可以执行ls命令,找到Linux被安装在了哪个分区。如果确定了分区,比如,(hd0,5),则执行\verb+ls (hd0, 5)+时会看到许多mod文件。
安装normal.mod:
\begin{verbatim}
grub rescue>set root=(hd0,5)
grub rescue>set prefix=(hd0,5)/boot/grub
grub rescue>insmod /boot/grub/normal.mod
\end{verbatim}
执行normal命令,可以恢复Grub界面。进入Linux后重新安装GRUB即可。


\subsection{登陆密码恢复}
方法如下:

1、重新启动,按ESC键进入Boot Menu,选择recovery mode(一般是第二个选项)。

2、在\#号提示符下用cat /etc/shadow,看看用户名。

3、输入passwd "用户名"(引号要有的哦)。

4、输入新的密码.

5、重新启动,用新密码登录。 
\section{lightdm登录配置}
配置文件在\verb+/etc/lightdm/+目录下,修改lightdm.conf文件
\begin{verbatim}
[SeatDefaults]
greeter-session=unity-greeter
user-session=ubuntu
greeter-show-manual-login=true #手工输入登陆系统的用户名和密码
allow-guest=false   #不允许guest登录
\end{verbatim}


\section{主机网络配置}
\subsection{配置IP地址和DNS}
可以使用图形工具network-manager-gnome.重启网络:系统设置-网络-关闭
\begin{shellcmd}
sudo vi /etc/network/interfaces 
\end{shellcmd}
\begin{verbatim}
auto lo
iface lo inet loopback
auto eth0 eth0:1
iface eth0 inet static
address 210.75.225.165
netmask 255.255.255.0
gateway 210.75.225.254
dns-nameservers 159.226.59.158 159.226.8.6
iface eth0:1 inet static
address 192.168.1.224
netmask 255.255.255.0
gateway 192.168.1.1
\end{verbatim}

新系统版本不希望手动更改/etc/resolv.conf文件,重启后会重置。修改/etc/resolvconf/tail文件,然后执行/etc/init.d/resolveconf restart,会自动生成/etc/resolve.conf文件。或者在上述/etc/network/interfaces添加dns-nameservers配置。

对于旧版系统,编辑/etc/resolv.conf文件,加入2行: 
\begin{verbatim}
nameserver 159.226.59.158
nameserver 159.226.8.6
\end{verbatim}
其他的DNS服务器包括:
谷歌:8.8.8.8
中国科大:202.38.64.1。


重启网络
\begin{shellcmd}
sudo /etc/init.d/networking restart
或者 sudo ifdown -a;sudo ifup -a;
\end{shellcmd}
如果配置了interfaces文件,那么network-manager-gnome无用,可以卸载

\subsection{代理设置}
使用tsocks这个软件。配置文件在/etc/tsocks.conf,修改local使其包含202.38.95.110。
其他工具还包括proxychains

\subsection{备注}
2012年5月9日我调动工位,网址由192.168.10.224改为同MAC绑定的210.75.225.165,结果网络不通。后来放弃network-manager-gnome工具,改用/etc/network/interface来配置,方可正常使用网络。原因不明。





\section{工作打印机配置}
首先确保能够连接打印机所在网络:192.168.192.0/24

\subsection{Windows XP下连接打印机}

“通过驱动连接打印机”的方式,打印机IP为:192.168.192.1。我不太明白这句话的意思。通过SMB方式连接打印机,在XP系统开始菜单,选择“添加打印机”,地址为:
\begin{verbatim}
//192.168.192.235/3nmedia_tzb_printer
\end{verbatim}

\subsection{Gnome桌面连接打印机}
网上的帖子大多是通过系统设置->打印机方式获得添加打印机的界面,但Linux Deepin 12.06下该界面无法解锁。使用另一套工具:system-config-printer-gnome软件包,该软件包包含了system-config-printer.py程序,执行它,就得到了一个添加打印机的界面。该程序由/usr/bin/system-config-printer调用。

执行system-config-printer命令后,选择以SMB方式添加打印机,等待程序搜索SMB网络,可以找到上述打印机。需要安装驱动程序,可以在给出的列表中选择Laser 1015驱动。



\section{Ubuntu快捷键}

Gnome快捷键参考系统设置->键盘->快捷键,其他应用的快捷键以该应用的具体设置为准。

\subsection{Gnome}

显示桌面:Ctrl+Alt+D 或者 Super+D,Ctrl+Super+D

前往其他工作区:Ctrl+Alt+Arrowkey

切换同一程序的不同窗口 :Alt+`

移动到其他工作区 :Ctrl+Alt+Shift+Arrowkey

打开帮助:F1

打开主菜单:Alt+F1

启动程序:Alt+F2

全屏 :F11

锁屏:Ctrl+Alt+L

注销:Ctrl+Alt+Del

\subsection{General Frame Window}

关闭Tab Ctrl+W

关闭Frame Alt+F4

恢复Frame原始大小  Alt+F5

最小化 Alt+F9

最大化 Alt+F10

\subsection{Nautilus}

显示隐藏的内容 : Ctrl+H

将路径从按钮变成文字 :  Ctrl+L(如需恢复回按钮,点击文字,按Esc)


\subsection{Terminal}

(Add a shift to normal operation)

清屏: Ctrl + L

暂停并置之后台: Ctrl+Z(fg to recover),may be a SIGTSTP signal,SUSP character

发送SIGQUIT信号: Ctrl + \

冻结终端显示: Ctrl + S(Ctrl+Q to recover,maybe a SIGCONT),NOT a SIGSTOP

向上滚屏一行:Ctrl+Shift+UpArrow

向上滚动一屏:Shift+PageUp

向上滚动至顶:Shift+Home


\subsection{tty切换}
Ctrl+Alt+F1~F6 从X切换到tty1~tty6

Ctrl+Alt+F7 返回X视窗系统











\section{Ubuntu软件安装与管理}

查看已经安装了哪些包:
\begin{verbatim}
dpkg -l
\end{verbatim}

查看软件包版本和其他信息
\begin{verbatim}
 apt-cache showpkg mendeleydesktop
\end{verbatim}

查找官方源软件包:
\begin{verbatim}
apt-cache search 正则表达式
aptitude search 软件包
\end{verbatim}

如果官方源中没有相关的包,可以查看PPA仓库:

\url{https://launchpad.net/ubuntu/+ppas}

安装ppa软件的一般方法有两种:
\begin{verbatim}
sudo add-apt-repository source_line
sudo add-apt-repository ppa:<user>/<ppa-name>
\end{verbatim}

查询软件xxx依赖哪些包:
\begin{verbatim}
apt-cache depends xxx
\end{verbatim}

查询软件xxx被哪些包依赖:
\begin{verbatim}
apt-cache rdepends xxx
\end{verbatim}

查找文件属于哪个包
\begin{verbatim}
apt-file search filename
\end{verbatim}

编译时缺少h文件的自动处理
\begin{verbatim}
sudo auto-apt run ./configure
\end{verbatim}

查看安装软件时下载包的临时存放目录
\begin{verbatim}
ls /var/cache/apt/archives
\end{verbatim}

备份当前系统安装的所有包的列表
\begin{verbatim}
dpkg --get-selections | grep -v deinstall > ~/somefile
\end{verbatim}

从上面备份的安装包的列表文件恢复所有包
\begin{verbatim}
dpkg --set-selections < ~/somefile
sudo dselect
\end{verbatim}

清理旧版本的软件缓存
\begin{verbatim}
sudo apt-get autoclean
\end{verbatim}

清理所有软件缓存
\begin{verbatim}
sudo apt-get clean
\end{verbatim}

删除系统不再使用的孤立软件
\begin{verbatim}
sudo apt-get autoremove
\end{verbatim}

查看包在服务器上面的地址
\begin{verbatim}
apt-get -qq --print-uris install ssh | cut -d\' -f2
\end{verbatim}

除所有已删除包的残馀配置文件
\begin{verbatim}
dpkg -l |grep ^rc|awk '{print $2}' |sudo xargs dpkg -P 
\end{verbatim}
如果你的系统中没有残留配置文件了,会报错。

\section{软件包安装与管理}

一般而言,aptitude命令优于apt命令。

查看已经安装了哪些包:
\begin{verbatim}
dpkg -l
\end{verbatim}

查看软件包版本和其他信息
\begin{verbatim}
 aptitude show mendeleydesktop
 apt-cache showpkg mendeleydesktop (详细信息,包括依赖关系)
 apt-cache show mendeleydesktop (概要信息)
 apt-cache depends xxx
 apt-cache rdepends xxx(被依赖)
\end{verbatim}

搜索软件包:
\begin{verbatim}
 aptitude search pkgname (推荐)
apt-cache search 正则表达式
\end{verbatim}

如果官方源中没有相关的包,可以查看PPA仓库:
\url{https://launchpad.net/ubuntu/+ppas}

安装ppa软件的一般方法有两种:
\begin{verbatim}
sudo add-apt-repository source_line
sudo add-apt-repository ppa:<user>/<ppa-name>
\end{verbatim}

\begin{verbatim}
查找文件属于哪个包
apt-file search/find filename
查询软件包包含的文件
atp-file list/show pkgname
\end{verbatim}

编译时缺少h文件的自动处理
\begin{verbatim}
sudo auto-apt run ./configure
\end{verbatim}

清理下载包的临时存放目录/var/cache/apt/archives
\begin{verbatim}
apt-get clean
apt-get autoclean (只删除部分无用的包)
\end{verbatim}

备份当前系统安装的所有包的列表
\begin{verbatim}
dpkg --get-selections | grep -v deinstall > ~/somefile
\end{verbatim}

从上面备份的安装包的列表文件恢复所有包
\begin{verbatim}
dpkg --set-selections < ~/somefile
sudo dselect
\end{verbatim}

删除系统不再使用的孤立软件
\begin{verbatim}
sudo apt-get autoremove
\end{verbatim}

查看包在服务器上面的地址
\begin{verbatim}
apt-get -qq --print-uris install ssh | cut -d\' -f2
\end{verbatim}

除所有已删除包的残馀配置文件
\begin{verbatim}
dpkg -l |grep ^rc|awk '{print $2}' |sudo xargs dpkg -P 
\end{verbatim}
如果你的系统中没有残留配置文件了,会报错。

\section{网络对时}
\begin{shellcmd}
date -s 20110813
ntpdate 210.72.145.44
\end{shellcmd}

国家授时中心的NTP服务器地址:210.72.145.44
教育网的ntp:
\begin{verbatim}
s1a.time.edu.cn 北京邮电大学
s1b.time.edu.cn 清华大学
s1c.time.edu.cn 北京大学
 s1d.time.edu.cn 东南大学
 s1e.time.edu.cn 清华大学
 s2a.time.edu.cn 清华大学
 s2b.time.edu.cn 清华大学
 s2c.time.edu.cn 北京邮电大学
s2d.time.edu.cn 西南地区网络中心
s2e.time.edu.cn 西北地区网络中心
s2f.time.edu.cn 东北地区网络中心
 s2g.time.edu.cn 华东南地区网络中心
 s2h.time.edu.cn 四川大学网络管理中心
 s2j.time.edu.cn 大连理工大学网络中心
 s2k.time.edu.cn CERNET桂林主节点
 s2m.time.edu.cn 北京大学
\end{verbatim}
