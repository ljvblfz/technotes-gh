\section{GDB调试器使用要略}


如果想对elfname程序进行调试,则:
\verb+gdb elfname+或\verb+gdb --args elfname arg1 arg2+\ldots
也可以只输入gdb,在交互界面上设置:
file elfname;
set args argv1 argv2 \ldots 

该程序在编译时需使用-g选项。

基本的用法,可以在进入gdb后执行help。

每次执行run,会从头开始运行程序。run简称r。

\subsection{查看源码}
list 显示当前位置10行代码。list简称l。

list funcname:显示某函数附近的10行代码

list lineno: 显示第lineno行及其上下文10行代码


\subsection{断点}
help b可以查看断电的用法。

首先需要明确,如果在第18行设置断点,指的是在第18行执行之前中止,而非之后。

设置断点:breakpoint命令,简称b,参数为[文件名:]行号或函数名,可以用list命令辅助b命令的参数选择。如无参数,为在当前行设置断点。

此外,断点还可以enable, delete, disable。

delete b会删除所有断点。简称del b或del或d。del 2会删除2号断点。

info breakpoints,简称info b,显示当前断点。

遇到断点时执行cont或c时程序继续执行直至结束或受阻。

commands 断点号:
用于修改断点行为。如 
\begin{verbatim}
commands 2
>display
>end
\end{verbatim}
程序在2号断点不中止,而是执行display;end用于退出对commands的定义。

\subsection{步进}
next命令简称n,用于执行下一条语句。执行完后显示的代码行为尚未执行的下一条语句。

n 5可以前进5行。

Return键用于重复执行上一条命令。

step命令简称s,与next的区别是会进入函数体。

\subsection{查看上下文信息}

backtrace, 简称bt,查看堆栈层次信息。最内层的frame号为零;

命令 frame frameno会跳到frameno指定的frame层次并打印相关信息。

在函数堆栈中,info args为函数参数;在main函数中则为程序参数。

info line 显示当前行号。

info source 查看当前源文件信息。info sources查看所有源文件信息。

info locals 局部变量信息查看。

info variables 查看全局变量和静态变量。

info registers,info frame顾名思义。

info macro macroname和info macros查看宏定义。


\subsection{查看指定变量}
print varname 查看变量。print简称p。

例如,对于char c[5] = {97,98,99,100,101},
执行print c[2]显示 99 'c';print /x c[2]会显示为16进制0x63。
print /x c[2]@3会显示c[2]开始的3个连续变量{0x63, 0x64, 0x65}。

display varname 添加自动显示变量。
display添加的变量会在每次程序中止时再次显示。

\subsection{Patching:就地修补}
set variable varname = varvalue命令用于就地修改变量的值。
那么,下面的命令在断点处修改变量的值,下次run时生效。
\begin{verbatim}
commands 2
>set variable n = n+1
>cont
>end
\end{verbatim}


\subsection{观察值}
watch varname命令:当程序执行时发现varname被修改时就中止;
rwatch varname:当varname被读时中止;而awatch是在varname被读或者写时都中止。

注意watch只能观察当前堆栈frame的变量。所以一般要配合断点来使用。








































