\section{C/C++中的日期和时间}

\subsection{基本常识}

世界标准世界(Coordinated Universal Time, UTC):协调世界时,又称格林威治标准时间(Greenwich Mean Time,GMT)。中国内地的时间与UTC的时差为+8,也就是UTC+8。美国是UTC-5。 

日历时间(Calendar Time):代表从一个日历参考点到此时的时间经过的秒数。日历参考点(Epoch)因编译器而异,如Unix系统通常使用UTC时间1970年元旦零点作为参考点,称作Unix时间或POSIX时间。

时钟计时单元(clock tick, 而不把它叫做时钟滴答次数):一个时钟计时单元的时间长短是由CPU控制的,但一个clock tick未必是CPU的一个时钟周期,而是C/C++的一个基本计时单位。


\subsection{程序时间:毫秒级精度}
time.h提供了如下计时函数clock(),返回从开启这个程序进程到程序中调用该函数时之间的CPU时钟计时单元(clock tick)数,称为挂钟时间(wall-clock):
\begin{lstlisting}[language=C]
clock_t clock(void);
\end{lstlisting}

函数其中clock\_t定义在time.h文件中,经常是long类型。time.h文件还定义了一个常量表示一秒钟会有多少个时钟计时单元: 
\begin{lstlisting}[language=C]
#define CLOCKS_PER_SEC ((clock_t)1000) 
\end{lstlisting}
表达式\verb|clock()/CLOCKS_PER_SEC|返回一个进程自身的运行时间。

\subsection{日历时间:秒级精度}
time.h通过time()函数返回日历时间,为从日历参考点到此时的秒数。
\begin{lstlisting}[language=C]
time_t time(time_t * timer); 
\end{lstlisting}

time\_t定义于time.h中,可能是long型。如果time\_t类型占32位,且日历参考点为Unix时间,则其表示的时间不能晚于2038年1月18日19时14分07秒。一些编译器厂商引入了64位甚至更长的整形数来保存日历时间。比如微软在Visual C++中采用\verb+__time64_t+数据类型来保存日历时间,并通过\verb|_time64()|函数来获得日历时间(而不是通过使用32位字的time()函数),这样就可以通过该数据类型保存3001年1月1日0时0分0秒(不包括该时间点)之前的时间。 

time.h提供difftime函数,实现两个日历时间值的简单相减(虽然返回值被转换为double型),无大用:
\begin{lstlisting}[language=C]
time_t difftime(time_t t1, time_t t2); 
\end{lstlisting}


\subsection{分解时间}

标准C/C++通过time.h定义tm结构保存时间结构: 
\begin{lstlisting}[language=C]
#ifndef _TM_DEFINED 
struct tm { 
int tm_sec; /* 秒 – 取值区间为[0,59] */ 
int tm_min; /* 分 - 取值区间为[0,59] */ 
int tm_hour; /* 时 - 取值区间为[0,23] */ 
int tm_mday; /* 一个月中的日期 - 取值区间为[1,31] */ 
int tm_mon; /* 月份(从一月开始,0代表一月) - 取值区间为[0,11] */ 
int tm_year; /* 年份,其值等于实际年份减去1900 */ 
int tm_wday; /* 星期 – 取值区间为[0,6],其中0代表星期天,1代表星期一,以此类推 */
int tm_yday; /* 从每年的1月1日开始的天数 – 取值区间为[0,365],其中0代表1月1日,1代表1月2日,以此类推 */ 
int tm_isdst; /* 夏令时标识符,实行夏令时的时候,tm_isdst为正。不实行夏令时的进候,tm_isdst为0;不了解情况时,tm_isdst()为负。*/ 
}; 
#define _TM_DEFINED 
#endif 
\end{lstlisting}

ANSI C标准称使用tm结构的这种时间表示为分解时间(broken-down time)。

time.h提供了以下函数实现日历时间和分解时间的相互转换: 
\begin{lstlisting}[language=C]
struct tm *gmtime(const time_t *timer); 
struct tm *localtime(const time_t *timer); 
time_t mktime(struct tm *timeptr); 
\end{lstlisting}

\subsection{时间显示}
asctime函数将tm结构转换为字符串:
\begin{lstlisting}[language=C]
char *asctime(const struct tm *timeptr);
\end{lstlisting}

asctime通过以下格式显示时间:
\begin{verbatim}
星期几 月份 日期 时:分:秒 年\n\0 
例如:Wed Jan 02 02:03:55 1980\n\0 
\end{verbatim}

ctime函数将日历时间转换为字符串,相当于嵌套调用localtime和asctime:
\begin{lstlisting}[language=C]
char * ctime(const time_t *timer); 
\end{lstlisting}

time.h还提供strftime实现更灵活的时间显示格式:
\begin{lstlisting}[language=C]
size_t strftime(char *strDest, size_t maxsize, const char *format, const struct tm *timeptr); 
\end{lstlisting}

strftime()根据format指向字符串中指定的格式命令把timeptr中保存的时间信息放在strDest指向的字符串中,最多向strDest中存放maxsize个字符。该函数返回向strDest指向的字符串中放置的字符数,使用方式类似于snprintf()。关于显示格式,strftime用\%B表示月份,\%G表示年份,等等。

\subsection{POSIX下的时间获取与设置:微秒精度}
sys/time.h提供了以下函数用于时间获取和设置:
\begin{lstlisting}[language=C]
int gettimeofday(struct timeval *tv, struct timezone *tz);
int settimeofday(const struct timeval *tv, const struct timezone *tz);
\end{lstlisting}

其中timeval结构代表自日历参考点以来的时间:

\begin{lstlisting}[language=C]
    struct timeval {
    time_t      tv_sec;     /* seconds */
    suseconds_t tv_usec;    /* microseconds */
    };
\end{lstlisting}

timezone结构定义如下:
\begin{lstlisting}[language=C]
    struct timezone {
    int tz_minuteswest;     /* minutes west of Greenwich */
    int tz_dsttime;         /* type of DST correction */
    };
\end{lstlisting}

timezone结构的使用已经过时,应该设置为NULL。尤其是tz\_dsttime字段,其在内核中的使用被视作bug。

这两个函数调用成功则返回零,失败则返回-1,同时设置errno。


















