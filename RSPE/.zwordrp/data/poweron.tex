\section{开机运行}
\subsection{CPUA启动}
串口线连接到RSPE的COM-A上。执行reboot命令重启RSPE,当它进入UBoot后输入Ctrl+C,使其停留在UBoot命令行界面。
下载Linux镜像:
\begin{verbatim}
tftpboot 0 vmlinux.64.iscsi.order
\end{verbatim}

其中,vmlinux.64.iscsi.order是Linux镜像(Linux内核需要在编译时选择支持crc32c),目前已经被预先放置在文件服务器的usr/local/nginx/html目录下。
如果上述过程不成功,需要确保物理连接畅通,同时SPI6被设置为UBoot默认端口,且SPI6地址和文件服务器在同一网段内:
\begin{verbatim}
setenv ethact spi6
setenv ipaddr 192.168.10.224
\end{verbatim}

下载镜像成功后,加载镜像:
\begin{verbatim}
bootoctlinux 0 coremask=0xffff mem=0
\end{verbatim}

coremask对应16个CPU的掩码,0xffff表示16个CPU全部加载,mem后为内存大小(MB),mem=0表示全部内存。
进入linux后,配置IP:
\begin{verbatim}
ifconfig spi6 192.168.10.224
ifconfig spi7 192.168.0.224
\end{verbatim}

从文件服务器下载iscsi相关软件包(目前已经被预先放置在文件服务器)并解压:
\begin{verbatim}
tftp -g -r iscsi.tar.gz 192.168.10.233
tar xzvf iscsi.tar.gz
cd iscsi
chmod +x *
\end{verbatim}

