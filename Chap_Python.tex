

\section{Python版本查询}
\begin{verbatim}
shell执行:python -V
Python脚本:import sys;print sys.version
\end{verbatim}

\section{Python额外包安装}
\begin{verbatim}
pip install package
\end{verbatim}


\section{Python类型判断}
\begin{verbatim}
>>>a=1
>>> isinstance(a,int)
True
>>> isinstance(a,(int,float))
True
>>> type(a)
<type 'int'>
>>> type(int())
<type 'int'>
>>> class AClass():
...     pass
... 
>>> a=AClass
>>> type(a)
<type 'classobj'>
>>> isinstance(a, AClass)
False
>>> b=AClass()
>>> isinstance(b, AClass)
True
\end{verbatim}

\section{Python特殊矩阵}
\begin{verbatim}
buckets = [0] * 100 #这种矩阵可能会出现赋值异常,即会保持所有值恒等。
buckets = [[0 for col in range(5)] for row in range(10)]

w, h = 100, 100
bucket = [[None] * w for i in range(h)]

import numpy
zarray = numpy.zeros(100)


\end{verbatim}











\section{Python时间与日期} 

Python的datetime包提供了日期与时间相关的操作。
\begin{verbatim}
#datetime提供了date,datetime,timedelta等类
from datetime import * 
#日期差计算
date(2012,7,28)-date(2012,7,25)
date.today()-date(2012,7,25)
#时间差计算
#datetime参数中,时、分、秒、微秒可选
datetime.now()-datetime(2012,7,25,09,23,45,23333)
#日期推算
#timedelta参数依次为天、秒、微秒,后两者可选
datetime(2012,7,25)+timedelta(2) #2天后的日期
date(2012,7,25)+timedelta(2) #2天后的日期
\end{verbatim}

两个date对象相减,得到的是timedelta类型的对象,如果想返回整数,则有:
\begin{verbatim}
(date.today()-date(2012,7,25)).days
\end{verbatim}

calendar模块提供了查询平闰年和星期的功能,如
\begin{verbatim}
calendar.isleap(2000)
calendar.weekday(2000, 1, 1) #周一是0,周日是6
\end{verbatim}
calendar也能产生日历字符串如:
\begin{verbatim}
print calendar.month(2000, 1) #月历
print calendar.calendar(2000) #年历
\end{verbatim}

\label{sec:pythonTimeCalc}


\section{Python读写Excel文件}
\begin{verbatim}
xlrd,xlwt, xlutils,Python-xlsx and PyXLSX,openpyxl
\end{verbatim}

\section{Python脚本}
常用对象和函数
\begin{verbatim}
sys.argv
len(sys.argv)
os.system()
time.sleep()
sys.exit()
os.path.exists()
os.path.isfile()
os.path.split()
os.path.basename()
\end{verbatim}

\subsection{检查文件是否存在}
os.path.isfile(filename)

检查文件是否可执行:
\begin{verbatim}
fpath = commands.getoutput('which %s'% handler)
if not (os.path.isfile(fpath) and os.access(fpath, os.X_OK)):
\end{verbatim}
\subsection{当前脚本名字}
os.path.basename(sys.argv[0])
os.path.split(sys.argv[0])[1]
\subsection{执行外部程序}
\begin{verbatim}
os.system(cmd)
commands.getstatusoutput(cmd)
subprocess.Popen([],...)
pexpect.spawn()
\end{verbatim}
subprocess模块定义了Popen类,试图取代os.system,os.spawn,os.popen, popen2,commands等模块。除Popen外,还定义了一些简洁函数call, check\_call, check\_output.详见subprocess文档。

pexpect包含的spawn类具有强大的交互功能,适用于ssh,scp等工具。

如果需要非阻塞式地获取进程的输出,似乎只能用subprocess.Popen或pexpect.spawn。


\section{Python字符编码问题}
\subsection{Python中文字符输出}
\begin{verbatim}
#!/usr/bin/python
# -*- coding: utf-8 -*-

u"中文¥ chinese"
\end{verbatim}

\subsection{Python Unicode序列与字符串}
unicode(str,encoding)函数将字符串str转换为unicode序列。

str.encode(encoding)将某种编码的字符串转为另一种编码的字符串。如:
\begin{verbatim}
str = u'你'
str2 = str.encode('gbk')
str3 = str.encode('utf8')#此处也作'utf-8'
\end{verbatim}


一个unicode数值可表示为字符串\verb|u'%c'%i|.中文“你”的unicode序列为0x4f60,unicode字符串为\verb|u'\u4f60'|。假设unicode数字序列用链表uarray表示,则其转换为unicode编码的字符串的过程为:
\begin{verbatim}
str = ''
for ucode in uarray: str += '%c'%ucode
\end{verbatim}

\label{subsec:PythonUnicode}




