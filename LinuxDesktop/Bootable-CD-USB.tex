\section{ISO镜像与光盘}
在Ubuntu下将文件夹做成iso镜像的工具叫做genisoimage(原名mkisofs)。可引导光盘加载后对相应文件夹使用genisoimage工具,会失去引导性。如欲保留引导性,可以使用dd工具对裸设备进行操作。如:

\verb+dd if=/dev/sr0 of=deepin.iso+

如需将iso烧录入光盘,可以使用Brasero工具。这很可能是Ubuntu系统自带的。


\section{USB启动盘}

可以使用dd命令或使用一些图形工具。在维基百科上有一个制作LiveUSB的软件列表。包括:
\begin{itemize}
  \item Fedora Liveusb-creator,在Windows或Linux下制作Fedora的LiveUSB
  \item Ubuntu Liveusb Creator(在命令为usb-creator-gtk或usb-creator-kde)
  \item LinuxLive USB Creator,在Windows上制作Linux LiveUSB
  \item Unetbootin,在Windows或Linux下制作UNIX LiveUSB,Ubuntu软件仓库有提供
\end{itemize}
我自己用优盘实际创建启动盘,在Ubuntu下使用Ubuntu Liveusb Creator制作失败了。用dd和Unetbootin制作成功。

如果使用dd命令,先umount这个USB设备,其名称可能是sdb1。但dd命令要作用与sdb而不是sdb1上。
\begin{verbatim}
dd if=linux_image.iso of=/dev/sdb
\end{verbatim}
尝试使用dd命令将Windows安装镜像做成启动盘,未能成功,U盘在启动时未能引导,依然由硬盘进行了引导。

制作完成后,需要更改BIOS设置,让USB-HDD(或其他USB设备类型)的设备启动优先级高于HDD。优盘可能被当作硬盘处理,如果是这种优盘,开机时需要在BIOS设置中更改的不是设备启动顺序,而是HDD启动优先级。在实验室电脑上DEL键进入BIOS设置。


